 \documentclass{article}
\usepackage[utf8]{inputenc}
\usepackage[english]{babel}
\usepackage{graphicx}
\usepackage{amsfonts}
\usepackage{amsmath}
\usepackage{amsthm}
%\usepackage[a4paper, total={7in, 10in}]{geometry}
\newtheorem{theorem}{Theorem}
\newtheorem{definition}{Definition}
\newtheorem{lemma}[theorem]{Lemma}

\newcommand{\notimplies}{
  \mathrel{{\ooalign{\hidewidth$\not\phantom{=}$\hidewidth\cr$\implies$}}}}
\begin{document}

\begin{definition}[Total Variation]
  $TV[f, T] = \sup_{0 < t_1 < \ldots < t_n=T, n \ge 1} \sum_{k=1}^n |f(t_i) - f(t_{i-1})|$
\end{definition}

\begin{theorem}[TV of of a Brownian Motion]
  $(X_t)_t$ BM $\Rightarrow TV[X, T] = \infty$ ass
\end{theorem}
\begin{proof}
  Eough to show that: $\lim_{n\infty} \sum_{k=1}^{2^n} |X_{Tk2^{-n}} - X_{T(k-1)2^{-n}}| = \infty$
  \begin{align*}
    \mathbb P(\lim\sup_{n\infty} \sum_{k=1}^{2^n} |X_{Tk2^{-n}} - X_{T(k-1)2^{-n}}|) < M)
    &\le \lim\inf \mathbb P(\sum_{k=1}^{2^n} \underbrace{|X_{Tk2^{-n}} - X_{T(k-1)2^{-n}}|}_{\mathcal N(0, T2^{-n})} < M) \\
    &\le \lim\inf \mathbb P(\frac1 {2^n} \sum_{k=1}^{2^n} |\epsilon_k| < \frac M {\sqrt{T2^n}}) \\
    & \rightarrow 0
  \end{align*}
\end{proof}

\section{Continuous Time theory}
$(\Omega, \mathcal F, (\mathcal F_t), \mathbb P)$ filtered probability space.

$(X_t)$ continuous time process

\begin{definition}[Continuous - Adapted]
  \begin{itemize}
  \item $X$ is $\mathcal F_t$-adapted (nonanticipating) if $X_t$ is $\mathcal F_t$-measurable $\forall t$
  \item $X$ is continuous if $t \rightarrow X_t(\omega)$ is continuous $\forall \omega$.
  \item $X$ is measurable if $(t, \omega) \rightarrow X_t(\omega)$ is $\mathcal B(\mathbb R) \times \mathcal F$-measurable.
  \item $X$ is progressively measurable if $X: [0, T] \times \Omega \rightarrow \mathbb R; (t, \omega) \rightarrow X_t(\omega)$ is $\mathcal B([0, T]) \times \mathcal F_t$-measurable.
  \end{itemize}
  Beware: measurable and adapted $\notimplies$ progressively measurable.
  But continuous + adapted $\implies$ progressively measurable
\end{definition}
\begin{definition}[Usual conditions]
  $(\mathcal F_t)_t$ is set to satisfy the \emph{usual conditions} if
  \begin{itemize}
  \item[a)] $\forall B \subseteq A \in \mathcal F \;   P(A) = 0 \implies B \in \mathcal F_0$ (completeness)
  \item[b)] $\mathcal F_t = \cup_{\varepsilon > 0} \; \mathcal F_{t+\varepsilon}$ (right-continuity)
  \end{itemize}
  Why?
  \begin{itemize}
  \item (a) $\implies$: if $(X_t)$ progressively measurable, $B$ borel set in $\mathbb R$, $\tau = \inf \{ t: X_t \in B \}$ is a stopping time.
    
    But if $(X_t)$ is continuous adapated, $B$ closed, $\tau$ is always a stopping time. (doesn't need a) ).
  \item $X$ BM, $\mathcal F_t^0$ the natural filtration. $\{ t \rightarrow X_t \text{ has a local maximum at } t \} \in \mathcal F_{t^+}^0$ and not in $\mathcal F_t^0$. But one can prove $\mathcal F_t^0 = \mathcal F_{t^+}^0$ a.s (for every set  $A \in \mathcal F_{t^+}^0$, there exist $B \in \mathcal F_{t}^0$ such that $\mathbb P(A \setminus B) = 0$).
  
  $\mathcal N = \{ B: B \subseteq A \in \mathcal F, \mathbb P(A) = 0 \}$, $\mathcal F_t = \cap_{\varepsilon > 0} \mathcal (F^0_t \vee N)$ is called the standard Brownian filtration, it satisfies usual conditions.
  
    b) is important if processes have jumps.
  \end{itemize}
\end{definition}

\section{Continuous Time Martingale}
\begin{definition}[Martingale]
  $(M_t)$ is a martingale if
  \begin{enumerate}
  \item $M$ is adapted
  \item $\mathbb E[|M_t|] < \infty \forall t$
  \item $\mathbb E[M_t | \mathcal F_s] = M_s \forall t \ge s$
  \end{enumerate}
\end{definition}

\begin{definition}[Stopping time]
  A r.v $\tau :\Omega \rightarrow [0, \infty]$ is a stopping time if $\{ \tau \le t \} \in \mathcal F_t \forall t$.
\end{definition}
\begin{definition}[Stopped filtration]
  $\mathcal F_{\tau} = \{ A \in \mathcal F_{\infty} : A \cup \{ \tau \le t \} \in \mathcal F_t \; \forall t \}$
\end{definition}
\begin{theorem}[Optional Stopping]
  $M$ is a \textbf{continuous} martingale.
  $\tau$ is a stopping time.

  Then $(M_{t \wedge \tau})_t$ is a continuous martingale
\end{theorem}
\begin{proof}
  \begin{enumerate}
  \item[Step 1)]
    Suppose $\tau \in \underbrace{\{t_1, t_2, \ldots\}}_{\Pi}$ as. Let's show that for $s < t$
    \begin{enumerate}
    \item $M_{t\wedge \tau}$ is $\mathcal F_t$-measurable
    \item $\mathbb E |M_{t\wedge \tau}| < \infty$
    \item $\mathbb E(M_{t\wedge \tau} |F_s) = M_{s \wedge \tau}$
    \end{enumerate}

    Without loss of generality let's assume that $s < t$ both in $\Pi$.
    Let $X_n = M_{t_n}$. Then $X_n$ is a discrete time martingale with respect to the filtration $(\mathcal G_n = \Mathcal F_{t_n})_n$.
  \item[Step 2)]
    $\tau$ any stop time.
    Introduce discrete step times $\tau_n \downarrow \tau$, take limits.
    $\Pi_n := \{ s, t, \infty, k2^{-n}, k \in \mathabb Z\}$
    $\tau_n = \inf \{ r \in \Pi_n r \ge \tau \}$ (We round up so it stays a stopping time)
    \begin{lemma}
      $\tau_n$ is a stopping time beacause $$\{ \tau_n \le t \} = \{ \tau \le \underbrace{\max \{s \in \Pi_n : s \le t \}}_{\le t} \}$$
    \end{lemma}
    So:
    \begin{itemize}
    \item[a)] $M_{t\wedge \tau} = \lim_n M_{t \wedge \tau_n}$ (because $M$ is continuous) all $\mathcal F_t$-measurable
    \item[b)] $\mathbb E|M_{t \wedge \tau}|  = \mathbb E \lim\inf |M_{t \wedge \tau_n}| \underbrace{\le}_{\text{fatou}} \lim\inf \mathbb E \underbrace{|M_{t \wedge \tau_n}|}_{\text{(submartingale)}} \le E|M_t| < \infty$
    \end{itemize}

  \end{enumerate}
\end{proof}
\end{document}

  



























