\documentclass[12pt]{article}

% packages
\usepackage{geometry}
\usepackage{amsmath}
\usepackage{amsfonts}
\usepackage{enumitem}

\newcommand{\Q}[1]{\subsubsection*{Q.#1}}
\newcommand{\union}[1]{\underset{#1}{\cup} }
\newcommand{\bigunion}[1]{\underset{#1}{\bigcup} \, }
\newcommand{\inter}[1]{\underset{#1}{\cap} }
\newcommand{\biginter}[1]{\underset{#1}{\bigcap} }

\newcommand{\minimize}[3]{\optimize{#1}{#2}{#3}{min}}
\newcommand{\maximize}[3]{\optimize{#1}{#2}{#3}{max}}

\DeclareMathOperator{\cov}{cov}
\DeclareMathOperator{\var}{var}


% parameters
\geometry{hmargin=1cm,vmargin=1cm}
\title{ORF527 - Problem Set 2}
\author{Bachir EL KHADIR }

\begin{document}
\maketitle
\Q{1}
$(W_{t_1}, \ldots W_{t_n})$ and $(W_{t_1}', \ldots W_{t_n}')$ are both guassian with the same mean and covariance matrix, so they do have the same carateristic function and therefore the same distribution.
So $$\mathbb E f(W_{t_1}, \ldots W_{t_n}) = \mathbb E f(W_{t_1}', \ldots W_{t_n}') $$

\Q{2}

\begin{align*}
  t \le \tau &\iff \forall \varepsilon > 0 \; \exists s \le t + \varepsilon \; X_s \in B
  \\&\iff \exists s \le t \; X_s \in B
\end{align*}
The last equivalence follows from the following:
\begin{itemize}
\item[$\Rightarrow$]  Trivial
\item[$\Leftarrow$] Suppose $\forall n \in \mathbb N; \exists s_n \le t + \frac1n \; X_{s_n} \in B$
  Without loss of generality, we can assume that $s_n$ converges to $s$ (by taking a subsequence)
  It is easy to see that $s \le t$, $X$ that $X_{s_n} \rightarrow X_s$ (by continuity), and that $X_s \in B$ (because $B$ is closed).
\end{itemize}
\begin{align*}
  t \le \tau &\iff \exists s \le t \; X_s \in B
  \\&\iff \inf_{s \le t} d(X_s, B) = 0
\end{align*}
Where the last equivalence is due to the fact that when $B$ is closed $x \in B \iff d(x, B) = 0$

Let's consider $f: x \rightarrow d(x, B)$. $f$ is continuous, $X_t$ is also continuous, therefore $\inf_{s \le t} f(X_s) = \inf_{s \le t, s \in \mathbb Q} f(X_s)$, and $\inf_{s \le t} f(X_s)$ is $\mathcal F_t$ measurable as a limit a countable number of $\mathcal F_t$-measurable functions.

\Q{3}
\begin{enumerate}
\item 
  \begin{enumerate}
  \item
    $\{ \tau \le t \} = \cap_{\varepsilon > 0} \underbrace{\{ \exists s \le t + \varepsilon, X_s \in B \}}_{U_{t+\varepsilon}}$,
    $U_t = \{ \exists s \le t, X_s \in B \} \in \mathcal F_t$:
    \begin{itemize}
    \item $U_t = \cup_{s \le t} \{X_s \in B \} = \cup_{s \le t, s \in \mathbb Q} \{ X_s \in B \}$. Indeed,
      Let $\omega \in \{ X_s \in B \}$ for some $s \le t$. Since $B$ is open there exist $\alpha > 0$
      such that $\mathcal B(X_s(\omega), \alpha) \subset B$.
      By continuity of $X_s$, there exist $s' \in \mathbb Q$ smaller than $t$ such that $|X_s(\omega) - X_{s'}(\omega)| < \alpha$,
      therefore $\omega \in \{ X_{s'} \in B\}$
    \item $\cup_{s \le t, s \in \mathbb Q} \{ X_s \in B \} \in \mathcal F_t$ as a countable union of sets in $\mathcal F_t$.
    \end{itemize}
    Let $a > 0$
    $(U_{t + \varepsilon})_{\varepsilon > 0}$ is non-decreasing sequence. So $\cap_{\varepsilon > 0} U_{t + \varepsilon} \subseteq U_{t + a} \in \mathcal F_{t+a}$,
    Therefore $\cap_{\varepsilon > 0} U_{t + \varepsilon} \cap_{\varepsilon > 0} \mathcal F_{t+\varepsilon}$
    Since the filtration satisfies the usual conditions, $\{\tau \le t\} = \cap_{\varepsilon > 0} U_{t + \varepsilon}  \in \mathcal F_t$ and $\tau$ is a stopping times.
  \end{enumerate}
\item
  $\Omega = \{ \omega_1, \omega_2 \}$,
  $X_t(\omega_1) = t$,
  $X_t(\omega_2) = -t$,
  $B = (0, \infty)$, $\tau = \inf \{ t , X_t \in B\}$,
  $\{\tau \le 0\} = \{ \omega_1 \}$.

  
  $\mathcal F_0 = \{ \emptyset, \Omega \}$
  so $\{\tau \le 0 \} \not \in \mathcal F_0$
  
  si $\tau$ is not stopping time.
  
\end{enumerate}

\Q{4}
\begin{itemize}
\item Without loss of generality we can assume
  $i \in J \Rightarrow i+t \in J$.  $(W_i)_{i \in J}$ is a discrete
  markov chain.  Let $A \in \mathcal F_{\tau}$, so that
  $A \cap \{\tau \le t\} \in \mathcal F_t$.

  Let's prove that  $E[1_A f(W_{t + \tau})] = E[1_A \int f(x)
  \frac{e^{-(x-W_{\tau})^2/2t}}{\sqrt{2\pi t}} dx]$

\begin{align*}
  E[1_A \int f(x) \frac{e^{-(x-W_{\tau})^2/2t}}{\sqrt{2\pi t}} dx]
  &= \sum_J E[1_{A, \tau=i} \int f(x) \frac{e^{-(x-W_i)^2/2t}}{\sqrt{2\pi t}} dx] &\text{Fubini for bounded r.v}
  \\&= \sum_J E[1_{A, \tau=i} E[f(W_{i+t}) | \mathcal F_i]] &\text{(Markov Property)}
  \\&= \sum_J E[ E[ 1_{A, \tau=i}f(W_{i+t}) | \mathcal F_i]]&\text{(Because $A \in \mathcal F_{\tau}$)}
  \\&= \sum_J E[ E[ 1_{A, \tau=i} f(W_{\tau+t}) | \mathcal F_i]]
  \\&= \sum_J E[ 1_{A, \tau=i} f(W_{\tau+t})]
  \\&= E[ 1_A f(W_{\tau+t})] &\text{Fubini for bounded r.v}
\end{align*}

$W_\tau$ is $\mathcal F_\tau$-measurable. Indeed, $\{ W_{\tau} \le a, \tau \le t \} = \cup_{j \in J, j \le t} \{W_j \le a, \tau = j\} \in \mathcal F_t$

By fubini, $\int \frac{e^{-(x-W_{\tau})^2/2t}}{\sqrt{2\pi t}}$  is $F_{\tau}$ measurable as integral of $\mathcal F_{\tau}$-measurable function.
So:
$$E[W_{t+\tau} | \mathcal F_{\tau}] = \int f(x) \frac{e^{-(x-W_{\tau})^2/2t}}{\sqrt{2\pi t}}$$

\item
  Let $\tau_k \downarrow \tau$ a sequence of discrete stopping times, then

  $$E[f(W_{t+\tau_k}) | \mathcal F_{\tau_k}] = \int f(x) \frac{e^{-(x-W_{\tau_k})^2/2t}}{\sqrt{2\pi t}}$$

  \begin{itemize}
  \item By continuity of $W_t$, $f(x) \frac{e^{-(x-W_{\tau_k})^2/2t}}{\sqrt{2\pi t}} \rightarrow f(x) \frac{e^{-(x-W_{\tau})^2/2t}}{\sqrt{2\pi t}}$ everywhere.
  \item $f$ is bounded, by dominated convergence theorem: $\int f(x) \frac{e^{-(x-W_{\tau_k})^2/2t}}{\sqrt{2\pi t}} \rightarrow \int f(x) \frac{e^{-(x-W_{\tau})^2/2t}}{\sqrt{2\pi t}}$
  \end{itemize}
  For $A \in \mathcal F_{\tau} \subset \mathcal F_{\tau_k}$
  $E[f(W_{\tau+t}) 1_A] = \lim E[f(W_{\tau_k + t}) 1_A] = \lim \int f(x) \frac{e^{-(x-W_{\tau_k})^2/2t}}{\sqrt{2\pi t}} = \int f(x) \frac{e^{-(x-W_{\tau})^2/2t}}{\sqrt{2\pi t}}$
\end{itemize}
\end{document}

%%% Local Variables:
%%% mode: latex
%%% TeX-master: t
%%% End:






















