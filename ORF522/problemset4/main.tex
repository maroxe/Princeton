\documentclass[12pt]{article}

% packages

\usepackage{geometry}
\usepackage{amsmath}
\usepackage{amsfonts}
\usepackage[]{mcode}



% custom commands
\newcommand{\Q}[1]{\subsubsection*{Problem #1}}

% parameters
\geometry{hmargin=1cm,vmargin=1cm}
\title{ORF524 - Problem Set 4}
\author{Bachir EL KHADIR }

\begin{document}

\maketitle

\Q{1}
The KKT conditions we will need:

\begin{itemize}
\item $\exists \lambda_j, \ge 0,  j= 1..n$ s.t $\nabla f(x^*) = - \sum_j \lambda_j \nabla g_j(x^*)$
\item $\lambda_j g_j(x^*) = 0$ for $j = 1..n$
\end{itemize}

Let $x$ be a feasible solution to the second optimization problem. We have that

\begin{align*}
\nabla f(x^*)^T(x-x^*) &= - \sum_j \lambda_j \nabla g_j(x^*)^T(x-x^*)
\\& \ge - \sum_j \lambda_j g_j(x^*) &\text{by feasibility of $x$}
\\& \ge 0 &\text{By complementarity condition}
\end{align*}

As a result, $f(x^*) + f(x^*)^T(x^*-x^*) \le f(x^*) + f(x^*)^T(x-x^*)$, and since $x^*$ trivially verifies the feasibility conditions of the second problem, $x^*$ is a global optimal.


\Q{2}


\begin{enumerate}
\item First order conditions:
\[
0 = \nabla f = \left( 
\begin{array}{c} 
4x + y - 6\\
x + 2y + z - 7 \\
y + 2z - 8
\end{array}
\right)
\]

Or 

\[
\left( 
\begin{array}{ccc} 
4&1&0\\
1&2&1 \\
0&1&2
\end{array}
\right)
\left( 
\begin{array}{c} 
x\\y\\z
\end{array}
\right)
=
\left( 
\begin{array}{c} 
6\\7\\8
\end{array}
\right)
\]


Which can be resolved:
\[
X^* = 
\left( 
\begin{array}{c} 
x\\y\\z
\end{array}
\right)
=
\frac15
\left( 
\begin{array}{c} 
6\\6\\17
\end{array}
\right)
\]


\item 

\[
\nabla^2 f(x,y,z) = \left(
\begin{array}{ccc}
4 & 1 & 0 \\
1 & 2 & 1 \\
0 & 1 & 2
\end{array}
\right)
\]
is symetric.


Let $X = (x, y, z)^T \in R^3$, 
$X^T\nabla^2fX = 4x^2 + 2x^2 + 2z^2 + 2 xy + 2 yz = 2(  x^2 + \frac12 y^2 + (x+\frac12 y)^2 + (z+\frac12 y)^2 ) \ge 0$ 
So $\nabla^2 f$ is positive semi-definite.

\item 

We proved that:
\[
\nabla^2 f(x,y,z) = \left(
\begin{array}{ccc}
4 & 1 & 0 \\
1 & 2 & 1 \\
0 & 1 & 2
\end{array}
\right) > 0
\]

So $f$ is convexe, and every local minimum is also a global minimum. Since $f$ admits only one local minimum $X^*$, it is the global minimum of $f$, and $\min f = f(\frac{6}{5}, \frac65, \frac{17}5) = -30.4$


\iffalse
\begin{align*}
f(x,y,z) &=  x^2 + \frac12 y^2 + (x+\frac12 y)^2 + (z+\frac12 y)^2- 6x -7y - 8z - 9
\\&= x^2 - 6x + \frac12 (y^2 - 6 y) + (x+\frac12 y)^2 + ( (z+\frac12 y)^2  - 8(z+\frac12y) )  - 9
\\&= (x-3)^2 + \frac12 (y-3)^2 + (x+\frac12 y)^2 + (z+\frac12y - 4)^2 - 9 - 9 - \frac12 9 - 16
\\&= (x-3)^2 + \frac12 (y-3)^2 + (x+\frac12 y)^2 + (z+\frac12y - 4)^2 - 38.5 \ge -38.5
\end{align*}
$f$ is a continous function bounded from below, and is coercive, it has a global minimum, which is also a local minimum. 
\fi


\item 

\lstinputlisting[language=Matlab, frame=single, breaklines=true]{problem2.m}

\end{enumerate}

\Q{3}

\begin{enumerate}

\item
Let's consider the problem (P):

$$\max_{2\pi(x^2 + y) \le C, x \ge 0, y \ge 0} \pi xy$$

The feasible set is bounded and closed. The objective function is continuous. So it has an optimal solution $(x^*, y^*)$. 
$x^* y^* \ne 0$ because $x = \sqrt {\frac{C}{4\pi}}, y = \frac{C}{4\pi}$ is a better feasible solution.

Let $h^* = \frac {x^*}{y^*}, r^* = x^*$. $(r^*, h^*)$ is optimal because:

\begin{itemize}
\item It is feasible: $2\pi({r^*}^2, + r^*h^*) = 2\pi({x^*}^2 + y^*) \le C$, $r > 0, h > 0$
\item If $r, h$ another feasible solution, then $x = r, y = rh$ is feasible for the problem (P), and therefore: $\pi x y \le \pi x^* y^*$, ie $\pi r^2h \le {r^{*}}^2h^*$
\end{itemize}

\item 
the objective function $(r,h) \rightarrow r^2 + rh$ is not convexe. Indeed its hessian \[\left(\begin{array}{cc}2h&2r\\2r&0\end{array} \right) \]
has negative derterminant($-4r^2$), so one of the eigen values have different signs, and therefore the objective function is neither convexe nor concave.

\item
Let's look for local optimal solutions. 

Let $(r, h)$ be an optimal solution. We know that $r \ne 0$ because $(\epsilon, h)$ is a better feasible solution for $\epsilon > 0$ small enough so that $2\pi(\epsilon^2 + \epsilon h) < C$

The lagrangian is $\mathcal L(r, h, \lambda) = \pi r^2 h + \lambda (C - 2\pi r^2 - 2\pi r h)$

KKT:

\begin{itemize}
\item $0 = \frac{\partial}{ \partial r}\mathcal L = 2\pi (rh - \lambda (2r + h)) \Rightarrow \lambda = \frac{rh}{2r + h}$
\item $0 = \frac{\partial}{ \partial h}\mathcal L = \pi (r^2 - \lambda 2 r) \Rightarrow \lambda = \frac r 2 \ne 0$
\item $\lambda = \frac r 2 = \frac{rh}{2r + h} \Rightarrow h = 2r $

\item Complementary condition $\lambda \ne 0 \Rightarrow C = 2\pi (r^2 + rh) = 6\pi r^2 \Rightarrow r = \sqrt{\frac{C}{6\pi}}$
\end{itemize}

Conclusion: $r = \sqrt{\frac{C}{6\pi}}$, $h =  \sqrt{\frac{2C}{3\pi}}$, $\pi r^2 h = \frac{C^{\frac32}}{3\sqrt{6\pi}}$

Since there is only one local optimum, and the problem admits an global optimum, then the local optimum is also global.


\end{enumerate}


\Q{4}

\begin{enumerate}
\item $f(x) = -x$, $g(x) = x^2$ are both convexe but $fog(x) = -x^2$ is not convexe.

\item 
Let $\lambda \in [0, 1]$, since $g$ is convexe:
$g(\lambda x + (1-\lambda)y) \le \lambda g(x) + (1-\lambda)g(y)$

Since $f$ is non-decreasing:
$f(g(\lambda x + (1-\lambda)y)) \le f(\lambda g(x) + (1-\lambda)g(y))$

Since $f$ is convexe:
$f(\lambda g(x) + (1-\lambda)g(y)) \le \lambda fog(x) + (1-\lambda)fog(y)$


as a conclusion 
$$fog(\lambda x + (1-\lambda)y) \le \lambda fog(x) + (1-\lambda)fog(y)$$

and $fog$ is convexe

\item

Take $f(x) = -e^x$, $g(x) = x$, but $fog(x) = -e^x$ is not convexe.



\item

Let's take $f = \frac 1 {1 + e^{-x}}$. 

$f'(x) = \frac{e^x}{1+e^{-x}} > 0$

$(xf)''(x) = \frac{e^x}{(1+e^{-x})^3} (2 + x - e^x(x-2)) \sim_{x +\infty} - x e^{2x} $


$f$ is increasing and non negative. 
$(xf)''$ goes to $-\infty$ as $x$ grow larger, so there is an $x > 0$ where it is negative, and as a result $xf$ is not convexe on $R^+$.
 



\end{enumerate}
\end{document}

%%% Local Variables:
%%% mode: latex
%%% TeX-master: t
%%% End:

