\documentclass[12pt]{article}

% packages
\usepackage{geometry}
\usepackage{amsmath}
\usepackage{amsfonts}


\newcommand{\Q}[1]{\subsubsection*{Problem #1}}
\newcommand{\optimize}[4]
{
\begin{align*}
& \underset{#1}{\text{#4}}
& & #2 \\
& \text{subject to}
& & #3
\end{align*}
}



\newcommand{\union}[1]{\underset{#1}{\cup} }
\newcommand{\bigunion}[1]{\underset{#1}{\bigcup} \, }
\newcommand{\inter}[1]{\underset{#1}{\cap} }
\newcommand{\biginter}[1]{\underset{#1}{\bigcap} }

\newcommand{\minimize}[3]{\optimize{#1}{#2}{#3}{min}}
\newcommand{\maximize}[3]{\optimize{#1}{#2}{#3}{max}}



% parameters
\geometry{hmargin=1cm,vmargin=1cm}
\title{ORF524 - Problem Set 2}
\author{Bachir EL KHADIR }

\begin{document}

\maketitle

\Q{1}

\[
\begin{array}{l|r|r|r|r}
\min \epsilon &    & 2 x_1 & -6 x_2 & 0x_3\\
         &    &    +\mu x_1 &  +\mu x_2 & +\mu x_3\\
\hline
w_1      & -1+\mu & 1 & 1 & 1\\
w_2      &  1+\mu & -2 & 1 & -1\\
\end{array}
\]

Feasibility and optimality conditions:
$ -6 + \mu \geq 0$,
$ -1 + \mu \geq 0$. $\Rightarrow \mu \geq 6$ 

for $\mu = 6$
\[
\begin{array}{l|r|r|r|r}
\min \epsilon &    & 8 x_1 & 0 x_2 & 6 x_3\\
\hline
w_1      & 5 & 1 & 1 & 1\\
w_2      & 7 & -2 & 1 & -1\\
\end{array}
\]

Entering basis: $x_2$. The ratio test fails $\Rightarrow$ The problem is unbounded.

If we go back to the original problem: 
\begin{itemize}
\item It is feasible. ($x^* = (0, 0, 1)$ is a feasible solution for example)
\item It is unbounded, because $\forall x_2 > 0 \,  x^*+x_2 e_2 \text{ is feasible, and }  c^T(x^*+x_2 e_2) = -6 x_2 \rightarrow -\infty$
\end{itemize}


\Q{2}
Let $A$ be the col player and $B$ the row player.

The payoff matrix is: (we set the value of a nickel to 1)
\[ M = \left(\begin{array}{cc}N&-D\\-N&D \end{array} \right)
= \left(\begin{array}{cc}1&-2\\-1&2 \end{array} \right)
 \]
The game can be written as $$\max_{a_1 + a_2 = 1} \min_{b_1 + b_2 = 1} b^T M a $$

Or $$ \min_{b_1 + b_2 = 1} \max_{a_1 + a_2 = 1} b^T M a $$

And the equivalent LP form:

$\max_{v,a} v$ subject to: $ v \leq e_i M a$ , $a_1 + a_2 = 1$, $a \geq 0$



\begin{align}
b^T M a &= (b_1 - b_2, -2(b_1 - b_2)) a \\&= (b_1-b_2)(1, -2)a \\&= (b_1 - b_2)(a_1 -2a_2) \\&= (2b_1 -1)(3a_1 - 2)\\&:=f(a_1, b_1)
\end{align}

Using the equiavlent LP form, we have that:

$$a_1^* = \arg\max_{a_1} \left[\min(f(a_1, 0), f(a_1, 1))\right] = \arg\max_{a_1} \min(3a_1 - 2, -(3a_1, 2)) = \arg\max_{a_1} - |3a_1 - 2| = \frac 2 3$$

$$b_1^* = \arg\min_{b_1} \left[\max(f(0, b_1), f(1, b_1))\right] = \arg\min_{b_1} \max(2b_1-1, -(2b_1-1)) = \arg\min_{b_1}  |2b_1-1| = \frac 1 2$$


The Nash equilibrium is thus attained for 
$a = (\frac 2 3, \frac 1 3), b = (\frac1 2, \frac 1 2)$



\Q{3}

By strong duality, if the (P) has a solution $x$, so does (D). (let's call it $p$)

Let's assume that $x$, the solution to (P), is unique and non degenerate, and prove that the solution to (D) is unique.
If $x$ is unique, $x$ is a BFS. Let $B$ be the associated basis. 
By complementary slackness, we have that $(p^TA_B - c_B^T) x_b = 0$, since $x_b > 0$: $p = (A_B^T)^{-1}c_B$.


Let's now prove that: $p$ degenrate and $x$ unique $\Rightarrow$ $x$ degenrate.

If $p$ is degenerate, there exist an entry that is $0$, meaning that the reduced cost of one of the non basic variables in the primal problem is $0$, and we can enter this variable without changing the objective function.As a result, we get two different basis $B_1$, $B_2$ that are associated with optimal solutions. If we assume the uniqueness of $x$, they must be both associated to $x$.
Let $i \in B_1 \setminus B_2$, $x_i = 0$, so $x$ is degenerate.


Conclusion: The solution to (P) is unique and non degenerate $\Rightarrow$ The solution to (D) is unique and non degenerate.
\Q{4}



\begin{enumerate}
\item Let $x$ and $y$ be in the solution set, and $\lambda \in [0,1]$, then $A (\lambda x + (1-\lambda y)) = \lambda b + (1-\lambda)b = b$, and $c^T(\lambda x + (1-\lambda)y) = \lambda c^T x + (1-\lambda)c^T y =\lambda v^* + (1-\lambda)v^* = \lambda c^T x + (1-\lambda)c^T y = v^*$, so the solution set is convex.
\item 

Let's consider the problem:
$$\max_{x \geq 0} \,  e_j^Tx \, \text{ s.t. } Ax = b, c^Tx \leq v^*$$

Its dual is
$$\min_{p, z \geq 0} \, b^Tp + zv^* \text{ s.t. } e_j - A^Tp \leq zc$$
can also be written as :
$$\max_{p, z \geq 0} \,  b^Tp - zv^* \text{ s.t. } e_j + A^Tp \leq zc$$


We know from the question that the primal problem has an optimal value of 0, and so does the dual. Let $(p, z)$ be a solution for the dual. 


\begin{enumerate}
\item
If $z \neq 0$:

And let's note $u = \frac p z$. We have that:

$b^Tu - v^* = 0$, and $\frac{e_j}{z} + A^Tu \leq c$
which means that $u$ is optimal for $(D)$, and

$$A_j^Tu \leq c_j - \frac{e_j}{z} < c_j$$


\item 
If $z = 0$, we have the existence of $p$ such that $b^Tp = 0$ and $A^Tp + e_j \leq 0$

$A^T(p^* + p) \leq c - e_j$, and $b^T(p^* + p) = v^*$
we can take $u = p^* + p$ to be a solution to $(P)$ that satisfies $A_j^Tu < c_j$.


\end{enumerate}


\end{enumerate}



\end{document}

%%% Local Variables:
%%% mode: latex
%%% TeX-master: t
%%% End:

