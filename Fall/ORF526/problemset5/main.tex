\documentclass[12pt]{article}

% packages
\usepackage{geometry}
\usepackage{amsmath}
\usepackage{amsfonts}

% custom commands
\newcommand{\Q}[1]{\subsubsection*{Question #1}}
\newcommand{\salgebra}{$\sigma$-algebra }

\newcommand{\union}[1]{\underset{#1}{\cup} }
\newcommand{\bigunion}[1]{\underset{#1}{\bigcup} \, }
\newcommand{\inter}[1]{\underset{#1}{\cap} }
\newcommand{\biginter}[1]{\underset{#1}{\bigcap} }

\newcommand{\norm}[1]{||{#1}|| }
\newcommand{\abs}[1]{|{#1}| }

% parameters
\geometry{hmargin=1cm,vmargin=1cm}
\title{ORF526 - Problem Set 1}
\author{Bachir EL KHADIR }

\begin{document}

\maketitle



$\Omega = [0, 1]$, $\mathcal F = \mathcal B([0, 1])$, $\mathbb{P}$ is the restriction of the lebesgue measure on $\Omega$. This is a probability space.


Let's consider the sequence:

$$X_k = k 1_{\{0 < x < \frac 1 k\}}$$


\begin{itemize}
\item $\mathbb{E}[X_k] = 1$
\item $X_k \rightarrow_{k \infty} 0$ a.s., because for all $x \in (0,1)$, $X_k(x) = 0$ for all $k > \frac{1}{x}$
\end{itemize}

\Q{1} 

\begin{itemize}
\item $\sup_k ||X_k||_1 = 1 < \infty$
\item For any $C > 0$, for any $k > C$, $\mathbb{E}[|X_k| 1_{\{X_k > C\}}] = \mathbb{E}[X_k] = 1$. Which means the sequence is not uniformly integrable.
\end{itemize}

\Q{2}
the $(X_k)$ satisfy the conditions: $\sup E[|X_n|] = 1$, $X_n \rightarrow 0$ and $E[|X_n|] = 1$

\Q{3}
$\mathbb{E}(\lim \inf X_k) = \mathbb{E}(\lim X_k) = \mathbb{E}(0) = 0 < 1 = \lim_k \mathbb{E}(X_k) = \lim \inf \mathbb{E}(X_k)$


\Q{4}

Let's define:
$\mu_1(A_1, ..., A_m) = \prod_i \mathbb{P}(X_i \in A_i)$,
$\mu_2(A_1, ..., A_m) = \mathbb{P}(X_1 \in A_1,...,X_m \in A_m)$

Let's prove by induction that for all $i = 0, .. m$:

$$\forall A_1, ... A_{i-1} \in B(R) \, \mu_1(A_1, ... A_{i-1}, (-\infty, x_i], ... (-\infty, x_n]) = \mu_2(A_1, ... A_{i-1}, (-\infty, x_i], ... (-\infty, x_n])$$

The propertie holds for $i=0$.

Let's suppose the property holds for $i \leq m$

Let's define: $$S(A_1, ... A_i) = \{ A_{i+1} \in B(R) | \forall x_{i+2} ... x_m\in \mathbb{\bar R} \, \mu_1(C) = \mu_2(C) \text{where } C = (A_1, ... A_{i+1}, (-\infty, x_{i+2}], ... (-\infty, x_n]\}$$

$S(A_1, ... A_i)$ is a Dynken system because:

\begin{itemize}
\item Let $A_{i+1} \in S(A_1, ... A_i)$, and $x_{i+2} ... x_m \in\mathbb{ \bar R}$, then
\begin{align*}
\mu_1(A_1, ... A_{i+1}^C, (-\infty, x_{i+2}], ... (-\infty, x_n]) 
&= \prod_{k \leq i} P(X_k \in A_k) P(X_{i+1} \in A_{i+1}^c) \prod_{k > i+1} P(X_k \in  (-\infty, x_k])
\\&= \prod_{k \leq i} P(X_k \in A_k) (1 - P(X_{i+1} \in A_{i+1})) \prod_{k > i+1} P(X_k \in  (-\infty, x_k])
\\&= \prod_{k \leq i} P(X_k \in A_k) P(X_{i+1} \in (-\infty, \infty)) \prod_{k > i+1} P(X_k \in  (-\infty, x_k]) 
\\&- \prod_{k \leq i} P(X_k \in A_k) P(X_{i+1} \in A_{i+1}) \prod_{k > i+1} P(X_k \in  (-\infty, x_k])
\\&= P(\{X_k \in A_k, k < i\}, X_{i+1} \in \mathbb{R}, \{X_k \in  (-\infty, x_k], k > i+1\}) 
\\&- P(\{X_k \in A_k, k < i\}, X_{i+1} \in A_{i+1}, \{X_k \in  (-\infty, x_k], k > i+1\}) 
\\&= P(\{X_k \in A_k, k < i\}, X_{i+1} \in A_{i+1}^c, \{X_k \in  (-\infty, x_k], k > i+1\}) 
\\&=\mu_2(A_1, ... A_{i+1}^C, (-\infty, x_{i+2}], ... (-\infty, x_n]) 
\end{align*}
\item \item Let $A_{i+1}^j \in S(A_1, ... A_i)$ a sequence of disjoint sets, and $x_{i+2} ... x_m \in\mathbb{ \bar R}$, and let's call $A = \union{j} A_{i+1}^k$ then
\begin{align*}
\mu_1(A_1, ... A, (-\infty, x_{i+2}], ... (-\infty, x_n]) 
&= \prod_{k \leq i} P(X_k \in A_k) P(X_{i+1} \in A) \prod_{k > i+1} P(X_k \in  (-\infty, x_k])
\\&= \prod_{k \leq i} P(X_k \in A_k) (\sum_j P(X_{i+1} \in A_{i+1}^j)) \prod_{k > i+1} P(X_k \in  (-\infty, x_k]) \\& \text{(because disjoint)}
\\&= \sum_j \prod_{k \leq i} P(X_k \in A_k) P(X_{i+1} \in A_{i+1}^j) \prod_{k > i+1} P(X_k \in  (-\infty, x_k]) 
\\&= \sum_j P(\{X_k \in A_k, k < i\}, X_{i+1} \in A_{i+1}^j, \{X_k \in  (-\infty, x_k], k > i+1\}) 
\\&= \sum_j P(\{X_k \in A_k, k < i\}, X_{i+1} \in A, \{X_k \in  (-\infty, x_k], k > i+1\}) 
\\& \text{(because disjoint)}
\\&=\mu_2(A_1, ... A, (-\infty, x_{i+2}], ... (-\infty, x_n]) 
\end{align*}
\end{itemize}



$S(A_1, ... A_i)$ contains the $\pi$-system $\{(-\infty, x_1 \times ... \times (-\infty, x_m] | \, x_1..., x_m \in \mathbb{\bar R} \}$  by induction hypothesis, so it contains the sigma algebra generated by them, which is $B(R)$. So $B(R) \subseteq S(A_1, ... A_i)$, $S(A_1, ... A_i) \subseteq B(R)$ trivially and therefore the induction hypothesis holds for $i+1$.


\Q{5}
\begin{enumerate}
\item $i \Rightarrow iii$

Let $\epsilon > 0$,$A_n = \union{m \geq n} \{ \omega , |X_n(\omega) - X(\omega)| > \epsilon\}$
and $A_{\infty} = \inter{n} A_n$.
$A_n$ is a decreasing sequence.

If $\omega \in A_\infty$, 
for infinitely many $m \in \mathbb{N}$, 
$|X_n(\omega) - X(\omega)| > \epsilon$. 
Which means that $\omega \in N$. Therefore $\mathbb{dP}(A_\infty) \leq \mathbb{P}(N) = 0$


By continuty from above:
$$\mathbb{P}(|X_n - X| > \epsilon) \leq \mathbb{P}(A_n) \rightarrow \mathbb{P}(A_\infty) = 0$$


\item $ii \Rightarrow iii$
By Markov inequality
$\mathbb{P}(|X_n - X| > \epsilon) \leq \mathbb{P}(|X_n - X|^p > \epsilon^p) \leq \frac{E{|X_n - X|^p}}{\epsilon^{2p}} \rightarrow 0$

\item $iii \Rightarrow iv$ 

\subsubsection*{Lemma}

For two rv $X, Y$ and $a \in \mathbb{R}$:
\begin{align}
  P(Y\leq a) &= P(Y\leq a,\ X\leq a+\varepsilon) + P(Y\leq a,\ X>a+\varepsilon) \\
  &\leq P(X\leq a+\varepsilon) + P(Y-X\leq a-X,\ a-X<-\varepsilon) \\
  &\leq P(X\leq a+\varepsilon) + P(Y-X<-\varepsilon) \\
  &\leq P(X\leq a+\varepsilon) + P(Y-X<-\varepsilon) + P(Y-X>\varepsilon)\\
  &= P(X\leq a+\varepsilon) + P(|Y-X|>\varepsilon)
\end{align}


Let $F$ be the cdf of $X$, and $a$ be a point of continuety of $F$.

Using the previous lemma:
$$P(X \leq a- \epsilon) - P(|X_n - X| > \epsilon) \leq P(X_n \leq a) \leq P(X_n \leq a + \epsilon) + P(|X_n - X| > \epsilon)$$


By taking the $\lim\sup$ and $\lim\inf$
$ P(X \leq a- \epsilon) \leq \lim\inf P(X_n \leq a)\leq \lim\sup P(X_n \leq a) \leq P(X_n \leq a + \epsilon)$
And by continuity of $F$ in $a$

$ F(a) \leq \lim\inf P(X_n \leq a)\leq \lim\sup P(X_n \leq a) \leq F(a)$

And therefore $F_n(a) \rightarrow F(a)$. By using question 6, we have the convergence in distribution.


\item 

Let $\epsilon > 0$
$$A_n^{\epsilon} = \{|X_n- X| > \epsilon \}$$
For every $n$, there exist infinitely many $m$ such that $P(A_m^{\frac 1 n}) \leq \frac1 {2^n}$, and let $\phi(n)$ be one of the $m$ such that $\phi(n) > \phi(n-1)$ when $n > 0$. 

$$\sum_n \mathbb{P}(A^{\frac1 n}_{\phi(n)}) \leq 1 < \infty$$

By Borel Cantelli, $\mathbb{P}(\lim \sup_n A^{\frac1 n}_{\phi(n)}) = 0$

If $\omega \not \in \lim\sup_n A^{\frac 1 n}_{\phi(n)}$, there is at most finitely many $n$ s.t $\omega \in A^{\frac 1 n}_{\phi(n)}$. Which means that there exist $N > 0$ s.t $\forall n > N \, |X_{\phi(n)}(\omega) - X(\omega)| \leq \frac{1}{n}$


We conclude that $X_{\phi(n)} \rightarrow X$ on $\left(\lim \sup_p A^{\frac 1 p}_{\phi(p)})\right)^c$, which is of measure 1.
\end{enumerate}


\Q{6}
\begin{enumerate}

\item
Every cdf is right continous and admits$F$ a left limit everywhere. (Let's call it $F(x-)$)

A point of discontinuty is where $F(x-) \neq F(x)$.


Let $A$ be the set of discontinuties of $F$.
\[
f: \left\{ 
\begin{array}{ll}
      A \longrightarrow &\mathbb{Q}\\
      x \longrightarrow &\text{some arbitrary } r \in (F(x^-), F(x))\\
\end{array}
\right.
\]

This application is an injection. So $A$ is countable.

\item


Let 
\[ f_k^a(x) = 
\left
\{ \begin{array}{ll}
1 & \text{if $x \leq a-\frac1 k$}\\
1 - k(x-(a-\frac1 k))&\text{if $a-\frac1 k  < x \leq a$}\\
0&\text{else}\\
\end{array}\right. \]
\[ g_k^a(x) = 
\left
\{ \begin{array}{ll}
1 & \text{if $x \leq a$}\\
1 - k(x-a)&\text{if $a < x \leq a + \frac 1 k$}\\
0&\text{else}\\
\end{array}\right. \]


\begin{itemize}
\item $f_k^a, g_k^a$ is continuous.
\item $f_k^a \uparrow_k 1_{x \leq a}$ pointwise.
\item $g_k^a \downarrow_k 1_{x \leq a}$ pointwise.
\item $f_k^a, g_k^a$ is positive and bounded.
\end{itemize}

Let $x$ be a point of continuty of $F$.

$$E[f_k^x(X_n)] \leq F_n(x) \leq E[g_k^x(X_n)]$$


so by right continuity: $$\lim \sup_n F_n(x) \geq \lim\sup_nE[f_k^x(X_n)] = F(x-\frac 1 k) \rightarrow_k F(x)$$
and by left continuty $$\lim \inf_n F_n(x) \geq \lim\inf_nE[g_k^x(X_n)] = F(x+\frac 1 k) \rightarrow_k F(x)$$

We proved that:
$$F(x) \leq \lim \inf_n F_n(x) \leq \lim \sup_n F_n(x) \leq F(x)$$
so the $F_n(x) \rightarrow F(x)$.

For the other direction:
Let $g$ be continuous and bounded.


Let first suppose that the support of $g$ is compact. Then $g$ is uniformly continuous. Given $\epsilon > 0$, there is $\delta > 0$ s.t $|x-y| < \delta \Rightarrow |g(x) - g(y)| < \epsilon$.
Let $C$ be a closed interval containing the support of $g$. This interval being bounded, we can find a partition of $C$ into intervals $(a_i, a_{i+1}]$, of size at most $\delta$, and we can also assume that the boundaries of the intervals are points of continuity of $F$ (the point of discontinuity being countable, we can avoid them)


We can construct a simple function $h = \sum_{i=1..p} g_i 1_{(a_i, a_{i+1}]}$,  we chose arbitrarly a value $g_i$ from the image of $g$ on each interval), so we have that $\sup |h - g| < \epsilon$.

We can always rewrite $h$ as $\sum_{i=1..r} h_i 1_{(-\infty, a_i]}$ .

$$\lim_n E[h(X_n)] = \lim_ n\sum_i h_i E[1_{(-\infty, a_i]}(X_n)] = \lim_n \sum_i h_i F_n(a_i) = \sum_i h_i F(a_i) = E[h(X)]$$

Let $n$ be large enough so that $|E[h(X_n)] - E[h(X_n)]| < \epsilon$

\begin{align}
|E[g(X_n)] - E[g(X)]| &\leq |E[g(X_n) - h(X_n)]| + |E[h(X_n) - h(X)]| + |E[h(X) - g(X)]|
\\&\leq 3\epsilon
\end{align}
Which conclude the proof.


If $g$ is only continous and bounded:

$X \neq \infty$ a.s, let $M > 0$ be so that $P(|X| > M) \leq \epsilon$

Let $f$ be continous function equal to $g$ on $[-M, M]$, equal to 0 when $|x| > M+\alpha$, and linear on the remaining intervals to make the function continous. 

$f$ has a compact support. Let $n$ be large enough so that $|E[f(X_n) - f(X)]| < \epsilon$

$C = \sup g \geq \sup f$
\begin{align}
|E[g(X_n) - f(X_n)]| 
&\leq  |E[(g(X_n) - f(X_n))1_{|X_n| < M}]|
\\&+ |E[(g(X_n) - f(X_n))1_{|X_n| \geq M }]|
\\&= 0 + 2 C P(M \leq |X_n|)
\\&\rightarrow_n 2C  P(M \leq |X|)\leq 2C \epsilon
\end{align}
So for $n$ large enough,  $|E[g(X_n) - f(X_n)]| \leq (2C+1) \epsilon$.


Using the same calculations:
$|E[g(X_n) - f(X_n)]| \leq 2C\epsilon$


\begin{align}
|E[g(X_n)] - E[g(X)]| &\leq |E[g(X_n) - f(X_n)]| + |E[f(X_n) - f(X)]| + |E[f(X) - g(X)]|
\\&\leq (2C+1) \epsilon + \epsilon + 2C \epsilon \\&\leq C' \epsilon
\end{align}

And we conclude the proof.

\end{enumerate}


\end{document}

%%% Local Variables:
%%% mode: latex
%%% TeX-master: t
%%% End:



