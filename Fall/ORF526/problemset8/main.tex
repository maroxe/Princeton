\documentclass[12pt]{article}

% packages
\usepackage{geometry}
\usepackage{amsmath}
\usepackage{amsfonts}
\usepackage{enumitem}
\usepackage{amsthm}





% custom commands
\newcommand{\Q}[1]{\subsubsection*{Question #1}}
\newcommand{\OB}[1]{\overbrace{#1}^{\in F_n}}
\newtheorem{thm}{Theorem}
\newtheorem{lem}[thm]{Lemma}

% parameters
\geometry{hmargin=1cm,vmargin=1cm}
\title{ORF526 - Problem Set 8}
\author{Bachir EL KHADIR }


\begin{document}

\maketitle

\Q{1}
$X_n$ is a martingale because
\begin{itemize}
\item It is adapted to $\mathcal F_n$, and $L_1$ by definition of conditional expectation.
\item $E[X_{n+1}|\mathcal F_n] = E[ E[X | \mathcal F_{n+1}] | \mathcal F_n] = E[X|F_n] = X_n$ because $F_n \subseteq F_{n+1}$
\end{itemize}

$X_n$ are ui because:
\begin{itemize}
\item
  By Jensen inequality: $E[|X_n|] = E[|E[X|\mathcal F_n]|] \le E[E[|X| |\mathcal F_n]] \le
  E[|X|] < \infty$

\item
  For $\epsilon > 0$,
  Since $X \in L_1$, there exist $\delta > 0$ st $\forall A \in \mathcal F, P(A) < \delta \Rightarrow E[|X|1_A] < \epsilon$.
  
  Let $c$ be large enough so that $\frac{E[|X|]}{c} \le \delta$.
  \begin{align*}
    E[|X_n| 1_{|X_n| > c]}] &\le E[ E[|X| | F_n] 1_{|X_n| > c]}]
    \\& \le E[E[ E[|X|  1_{|X_n| > c]} | F_n]] &\text{because $1_{|X_n| > c]}$ is $F_n$ measurable}
    \\& \le E[|X|1_{|X_n| > c]}]
    \\& \le \epsilon &\text{because $P(|X_n| > c) \le \frac{E[|X_n|]}{c} \le \frac{E[|X|]}{c} \le \delta$}
  \end{align*}
\end{itemize}

\Q{2}
$F_{\tau}$ is a $\sigma$-algebra because:
($n$ denotes a natural number)
\begin{itemize}
\item $\{\tau = n\} \cap \emptyset = \emptyset \in F_n$ so $\emptyset \in F_{\tau}$, and $F_{\tau} \ne \emptyset$
\item Let $(A_i)_{i \in \mathbb{N}} \in F_{\tau}^{\mathbb{N}}$,
  $(\cup_{i \in \mathbb{N}} A_i) \cap \{ \tau = n \} = \cup_{i \in \mathbb{N}} (\OB{A_i \cap \{ \tau = n \}}) \in F_n$, so $\cup_{i \in \mathbb{N}} A_i \in F_{\tau}$
\item Let $A \in F_{\tau}$, then $(A^c \cap \{\tau = n\})^c = A \cup \{\tau \ne n\} = (\OB{A \cap \{\tau = n\}}) \cup \OB{\{\tau \ne n\}} \in \mathcal F_n$, so $A^c \cap \{\tau = n\} \in F_n$ so $A^c \in F_{\tau}$

\end{itemize}

\Q{3}
\begin{lem}
  if $\tau$ a stopping time adapted to $(F_n)$, then for $k \le n$, $\{ \tau = k \} \in F_n$ and $\{ \tau \le n \} \in F_n$.
\end{lem}
proof: $F_n$ is increasing and $\{ \tau \le n \} =  \cup_{k = 0..n} \OB{\{ \tau = k \}} \in F_n$
\begin{lem}
  if $A \cap \{\tau \le n\} \in F_n \forall n$ then $A \in F_{\tau}$
\end{lem}
proof: $A \cap \{\tau = n\} = A \cap \{\tau \le n\} \cap \{\tau \le n-1\}^c \in F_n$ for all $n$, so $A \in F_{\tau}$

$n$ is an arbitrary natural number:
\begin{enumerate}[label=\alph*)] 
\item $\{ \tau + \sigma = n \} = \cup_{k=0..n} (\OB{\{ \tau = k \}} \cap \OB{\{ \sigma = n-k \}}) \in F_n$, so $\tau + \sigma$ is a stopping time.
  \item
  $\{ \tau \vee \sigma \le n\} =
\{ \tau \le n\} \cup \{ \sigma \le n\} \in F_n$,
by lemma 2 $\tau \vee \sigma$ is a stopping time.
\item
  $\{ \tau > k \} \in F_k$ because $\{ \tau > k \} = \{ \tau \le k \}^c \in F_k$. Same for $\{ \sigma > k \}$
  
  $\{ \tau \wedge \sigma > k \} = \{ \tau > k \} \cap \{ \sigma > k \} \in F_k$
  
  But $\{ \tau \wedge \sigma \le n \} =  \{ \tau \wedge \sigma > n  \}^c \in F_n$, 
  so by lemma 2 $\tau \wedge \sigma$ is a stopping time.

\item 
  \begin{itemize}
  \item[$\Rightarrow$]Let $A \in F_{\tau} \cap F_{\sigma}$, then
    $A \cap \{ \tau \le n \}$ and $A \cap \{ \sigma \le n \}$ are in $F_n$,
    so is their intersection $A \cap \{ \tau \wedge \sigma \le n\}$.
    c/c: $F_{\tau} \cap F_{\sigma} \subset F_{\tau \wedge \sigma}$
    \item[$\Leftarrow$] Let $A \in F_{\tau \wedge \sigma}$,
  then $A \cap \{ \tau \wedge \sigma = n \} \in F_n$

  \begin{align*}
    A \cap \{ \tau = n \}
    &= \left(\cup_{k \le n} A \cap \{ \tau = n, \sigma = k \}\right)
      \bigcup \left(\cup_{k > n} A \cap \{ \tau = n, \sigma = k \} \right)
    \\&= \left(\cup_{k \le n} A \cap \{ \tau = n, \tau \wedge \sigma = k \} \right)
        \bigcup \left(\cup_{k > n} A \cap \{ \sigma = k, \tau \wedge \sigma = n \} \right)
    \\&= \left( \OB{A \cap \{ \tau = n \}} \cap \OB{\{\tau \wedge \sigma \le n \}} \right)
        \bigcup \left(\OB{A  \cap \{ \tau \wedge \sigma = n \}} \cap \OB{\{ \sigma \le n \}^c} \right)
    \\& \in F_n
  \end{align*}
  c/c: $ F_{\tau \wedge \sigma} \subset F_{\tau} \cap F_{\sigma}$
  \end{itemize}
As a conclusion $F_{\tau \wedge \sigma} = F_{\tau} \cap F_{\sigma}$
\end{enumerate}

\Q{4} Let $M_n$ be such a martingale, then
\begin{itemize}
  \item $\forall n, \,  M_n \in L_1$ and is $G_n$ adapated trivially.
\item 
Since $E[ M_{n+1} | F_n] = M_n$, we have that $E[E[ M_{n+1} | F_n]|G_n] = E[M_n|G_n] = M_n$, so that
$E[ M_{n+1} | G_n] = M_n$ because $G_n \subset F_n$.
\end{itemize}
\Q{5}
\begin{enumerate}[label=\alph*)]
\item
  Let $N \in mathbb{N}^*$, we have that
  $$ \cup_{n=1..N} \{ Y_{(A+B)n + k} = 1, k=1..(A+B) \} \subset \{ \tau <  \infty \}$$
\begin{align*}
  P(\tau < \infty) &\ge P(\cup_{n=1..N} \{ Y_{(A+B)n + k} = 1, k=1..(A+B) \})
  \\&= 1 - P(\cap_n \{ Y_{(A+B)n + k} = 1, k=1..(A+B) \})^c)
  \\&= 1 - \prod_{n=1..N} P(\{ Y_{(A+B)n + k} = 1, k=1..(A+B) \})^c) &\text{independence}
  \\&= 1 - \prod_{n=1..N} (1-P(\{ Y_{(A+B)n + k} = 1, k=1..(A+B) \}))&\text{independence}
  \\&= 1 - \prod_{n=1..N} (1-p^{A+B})
  \\&= 1 - (1-p^{A+B})^N \rightarrow 0 &\text{because $0 < p < 1$}
\end{align*}
so $\tau < \infty$ a.s.
\item
  $$E[X_N | G_{N-1}] - X_{N-1} = X_N - X_{N-1} = Y_N \ne 0$$
  So $(X_n)$ is not martingale with respect to $(G_n)$.
\end{enumerate}
\end{document}

%%% Local Variables:
%%% mode: latex
%%% TeX-master: t
%%% End:



