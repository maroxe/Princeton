\documentclass[12pt]{article}

% packages
\usepackage{geometry}
\usepackage{amsmath}
\usepackage{amsfonts}


\geometry{hmargin=1cm,vmargin=1cm}


% custom commands
\newcommand{\Q}[1]{\subsubsection*{Question #1}}
\newcommand{\salgebra}{$\sigma$-algebra }

\newcommand{\union}[1]{\underset{#1}{\cup} }
\newcommand{\bigunion}[1]{\underset{#1}{\bigcup} \, }
\newcommand{\inter}[1]{\underset{#1}{\cap} }
\newcommand{\biginter}[1]{\underset{#1}{\bigcap} }

\newcommand{\norm}[1]{||{#1}|| }
\newcommand{\abs}[1]{|{#1}| }



% parameters
\title{ORF526 - Problem Set 4}
\author{Bachir EL KHADIR }

\begin{document}

\maketitle

\Q{1}
\begin{enumerate}

\item
Let $\mathbb{F}$ be a field which is either $\mathbb{R}$ or $\mathbb{C}$.  A normed vector space over $\mathbb{F}$ is a pair $(V,\norm{\cdot})$ where $V$ is a vector space over $\mathbb{F}$ and $\norm{\cdot}\colon V\to\mathbb{R}$ is a function such that
\begin{enumerate}
\item $\norm{v}\geq 0$ for all $v\in V$ and $\norm{v}=0$ if and only if $v=0$ in $V$ (\emph{positive definiteness})
\item $\norm{\lambda v} = \abs{\lambda} \norm{v}$ 

for all $v\in V$ and all $\lambda\in\mathbb{F}$
\item $\norm{v+w}\leq\norm{v}+\norm{w}$ for all $v,w\in V$ (the \emph{triangle inequality})
\end{enumerate}


\item Inner product space is a vector space with an inner product

\item A metric space $M$ is a set with a distance.
It is called complete if every Cauchy sequence of points in $M$ has a limit that is also in $M$.

\item A Banach space is a vector space X over the field $R$ of real numbers, or over the field $C$ of complex numbers, which is equipped with a norm and which is complete with respect to that norm.

\item A Hilbert space is a vector space $H$ with an inner product $<f,g>$ such that the norm defined by
 $\norm{f}=\sqrt{<f,f>}$
turns $H$ into a complete metric space. 
\end{enumerate}

\Q{2}

In the following we use these properties: For $A, B$ two measurable sets:

\begin{enumerate}
\item if $A \subseteq B$, $\mu(B \setminus A) = \mu(B) - \mu(A)$
\item $\mu(A \union{} B) = \mu(A \union{} (B \setminus A \inter{}B)) = \mu(A) + \mu(B) - \mu(A \inter{} B)$
\end{enumerate}


\begin{itemize}
\item
$$(a_1, b_1] \times (a_2, b_2] = (-\infty, b_1] \times (-\infty, b_2]
 \setminus 
\left( (-\infty, b_1] \times (-\infty, a_2] 
 \union{} (-\infty, a_1] \times (-\infty, b_2] \right)$$ 


\begin{align*}
\mu (a_1, b_1] \times (a_2, b_2] 
&= \mu (-\infty, b_1] \times (-\infty, b_2]
 -
\mu \left( (-\infty, b_1] \times (-\infty, a_2] 
 \union{} (-\infty, a_1] \times (-\infty, b_2] \right)\\
&= \mu (-\infty, b_1] \times (-\infty, b_2] 
-\mu  (-\infty, b_1] \times (-\infty, a_2] 
- \mu ((-\infty, a_1] \times (-\infty, b_2]) \\
&+ \mu \left( (-\infty, b_1] \times (-\infty, a_2] 
 \inter{} (-\infty, a_1] \times (-\infty, b_2] \right) \\
&= F(b_1, b_2) - F(b_1, a_2) - F(b_2, a_1) + F(a_1, a_2)
\end{align*}

\item
The sequence $x_k$ is decreasing so the following intersection is decreasing:
$$(-\infty, x_1] \times (-\infty, x_2] = \biginter{k \in \mathbb{N}} (-\infty, x^k_1] \times (-\infty, x^k_1]$$
By continuity from above $F(x_k) \rightarrow F(x)$

\item

$$\mathbb{R} = \union{k \in \mathbb{N}} (-\infty, x_1^k] \times (-\infty, x_2^k]$$

The union is increasing because $x_{k}$ is increasing, by continuity from below we have the equality.


\item If any of the $x_i^k$ go to $-\infty$, we have the following decreasing intersection:
$$\emptyset = \inter{k} (-\infty, x^k_1) \times (-\infty, x^k_2) $$
We conclude by continuety at 0.


\item By monotonicity of the measure $(-\infty, x_1] \times (-\infty, x_2] \subseteq (-\infty, y_1] \times (-\infty, x_2] $ so $F(x_1, x_2) \leq F(y_1, x_2)$
By symetry of $(x_1, y_1)$ and $(x_2, y_2)$, we prove the other inequality.

\item Let G be function definied on $R^2$ such that 

\[ 
G(x,y) = \left\{
\begin{array}{cc}
    1 & \text{when $x, y \geq 0$ and $(x, y) \neq (0,0)$} \\
      0 & \text{otherwise}
  \end{array}
  \right.
\]
$G(x,y)$ is non decreasing in $x$ and $y$, but
$$G(1, 1) - G(1, 0) - G(0, 1) + G(0, 0) = 1 - 1 - 1 + 0 = -1$$

\end{itemize}

\Q{3}
Let's write $f$ and $g$ as:
$f = \sum_i a_i 1_{A_i}$, $g = \sum_k b_k 1_{B_k}$,
$f+g = \sum_i a_i 1_{A_i} + \sum_k b_k 1_{B_k}$


$$\int (f+g) = \sum_i a_i \mu(A_i) + \sum_k b_k \mu(B_k) = \int f + \int g$$, 

General case:
If $f, g \geq 0$
Let $(f_n)_n$ and$ (g_n)_n$ two sequences of simple functions 

$0 \leq f_n \uparrow f$, 

$0 \leq g_n \uparrow g$

Then $0 \leq f_n + g_n \uparrow f+g$
By monotone convergence theorem:
$$\int f+g = \lim_n \int  f_n + g_n = \lim_n \int f_n + \lim_n \int g_n = \int f + \int g$$

If $f$ and $g$ measurable, we write $f = f^+ - f^-$ and $g = g^+ - g^-$ and we apply the precedent result.

\Q{4}
\begin{itemize}
\item
If $f = \sum a_i 1_{A_i}$ a simple function, then $cf = \sum (c a_i) 1_{A_I}$, $\int c f = \sum c a_i \mu(A_i) = c \sum a_i \mu(A_i) = c \int f$.



If $f \geq 0$
If $f_n$ a sequence of non negative increasing simple function converging to $f$, then $(c f_n)$ is an monotonous sequence converging to $c f$, and therefore by monotnous convergence, $\int c f = \lim \int c f_n = c \lim \int f_n = c \int f$.

General case $f = f^+ - f^-$, so by linearity $$\int cf = \int c f^+ + \int (-c) f^- = c(\int f^+ - \int  f^-) = c \int f$$

\item $h := g-f$, $\mu{h < 0} = 0$

\begin{align*}
\int g - \int f  &= \int (g-f)  \\
&= \int h \\
&= \int_{\{h \geq 0\}} h + \int_{\{h < 0\}} h \\
&= \int_{\{h \geq 0\}} h^+ + \int_{\{h < 0\}} h
\end{align*}

It is easy to see that the integral of constant functions on a set of measure $0$ is $0$ , by linearity it extends to simple functions, and it holds for measurable positive functions because of the definition of the integral as the sup of a set that contains only $0$ and for measurable function by linearity. Therefore $\int_{\{h < 0\}} h = 0$


Moreover, the integral of positive functions is definied as the sup of the integral of positive simple functions, and that integral is positive as the sum of positive terms. So $\int_{\{h \geq 0\}} h^+ \geq 0$

We conclude that $\int g - \int f \geq 0$.


\item 
$$\int |f| = \int_{f = 0} |f| + \int_{f \neq 0} |f| = \int_{f \neq 0} |f| = 0$$

$$\{f \neq 0\} = \union{n \geq 1} \{ |f| \geq \frac1 n\}$$

Let's call $A_n := \{ |f| \geq \frac1 n\}$ for $n \geq 1$, so that $A_n$ is increasing. By using the last question:
$$ 0 \geq \int_{f \neq 0} |f| \geq  \int_{\{f \neq 0\} \inter{} A_n} |f| + \int_{A_n} |f| \geq  \int_{A_n} |f| \geq \frac{\mu A_n}{n} \geq 0$$

so $\mu A_n = 0$. and subsequently by continuty from below $$\mu \{f \neq 0\} = 0$$

\end{itemize}


\Q{5}
$\mu$ is a measure because:
\begin{itemize}
\item $\mu_f(\emptyset) = \mu f^{-1}(\emptyset) = \mu \emptyset = 0$
\item $\mu_f(B^c) = \mu f^{-1}(B^c) = \mu (f^{-1}B)^c = 1 - \mu(f^{-1}B) = 1 - \mu_f(B)$
\item if $\{B_k | k \in \mathbb{N}\}$ a set of pairwise disjoint sets, so is$\{ f^{-1} B_k | k \in \mathbb{N}\}$ and therefore $$\mu_f(\union{k} B_k) = \mu(f^{-1}\union{k}B_k) = \mu(\union{k}f^{-1}B_k) = \sum_k \mu_f(B_k)$$
\end{itemize}


If $g$ is simple, eg $g = \sum a_i 1_{A_i}$: $gof = \sum a_i 1_{f^{-1}(A_i)}$

$$\int_{\Omega} g o f \rm{d}\mu = \sum_i a_i \mu(f^{-1}A_i) = \sum_i a_i \mu_f(A_i) = \int_E g  \rm{d}\mu_f$$

If $g \geq 0$, and $0 \leq g_n \uparrow g$, $g_nof \uparrow gof$  by monotone convergence:
$$\int gof \rm d \mu= \lim\int g_nof  = \lim \int g_n \rm{d}\mu_f = \int g \rm{d}\mu_f$$
If $g$ is measurable such that the integral exist, we write $g = g^+ - g^-$,$gof = g^+of - g^-of$, one of the integrals of $g^+, g^-$ is finite, and we have the result by linearity of the integral. 

\Q{6}

\begin{enumerate}
\item Let $(x_k)_k$ be a sequence s.t. $x_k \downarrow x$, then by continuety from above of the probability measure:

$F(X \leq x) = P( \inter{k}\{X \leq x_k\} ) = \lim_k P( \{X \leq x_k\} ) = \lim_K F(x_k)$ and therefore $F$ is right continuous.

First we notice that by definition of $q$ and $F$, for every $u$ and $\epsilon > 0$:

$$F(q(u)+\epsilon) > u \geq F(q(u) - \epsilon)$$

$q$ is non-decreasing, so the right limit exists at every point u. Let's note it $q(u^+)$. 
Let's suppose $q(u^+) > q(u)$ and note $\alpha = q(u^+) - q(u)$.
Let $(u_n)$ a decreasing sequence s.t $u_n \downarrow u$ so that $q(u_n) \downarrow q(u)$
For every $\epsilon > 0$ we have:$u_n \geq F(q(u_n) - \epsilon) $


For $\epsilon  = \frac{\alpha}2$, we have
$$u_n \geq F(q(u_n) - \epsilon) \geq F(q(u^+) - \epsilon) \geq F(q(u) + \epsilon) > u$$
and by going to the limit:
$$u \geq F(q(u)+\epsilon) > u$$
Contradiction. $q$ is right continous.


\item 

Let $g_n(x) = x$, by question $5$:
\begin{align*}
\mathbb{E}[X] &= \int_{\Omega} goX(\omega) \rm{d}P \\
&= \int_{\mathbb{R}} g \, \rm d \mu_{X}
\end{align*}


Let's consider the application from the Borel to $\mathbb{R}^+$: $\alpha: A \rightarrow \int_\mathbb{R} 1_A \rm dF_X$.

Since $F_X$ is right continous, we can prove that $\alpha$ is a measure that agrees with $\mu_X$ on every interval (so they agree on a semi-ring that generates the Borel $\sigma$-algebra), so by  unicity garanteed by Carathéodory extension theorem for $\sigma$-finite measures , $\mu_x = \alpha$. so that

$$E[X] = \int_R g \rm d \mu_X = \int_R g \rm d F_X$$



\item
Let's first prove the statement of bounded functions. Let $X$ be integrable such that $|X| < a$

\begin{align*}
\int_{-\infty}^0 (P[X > x] - 1) \rm dx 
&= -  \int_{-a}^0 F(x) \rm dx \\
&=  \int_{-a}^0 x \rm d F(x) - F(0)*0 - F(-a)a\\
&= \int_{-a}^0 x \rm d F(x) 
&= \int_{-\infty}^0 x \rm d F(x) 
\end{align*}


\begin{align*}
\int^{\infty}_0 P[X > x] \rm dx 
&= \int_0^a 1 - F(x) \rm dx \\
&= -\int_0^a x \rm d(1- F) (x) + a(1-F(a)) \\
&= \int_0^a x \rm d F(x) \\
&= \int_0^{\infty} x \rm d F(x) \\
\end{align*}
$F$ being constant for $x > a$ and $x < -a$, the Stieltjes sum and there for the $\int$ is 0 on those interval.



By summing both:

$$\int_{-\infty}^0 (P[X > x] - 1) \rm dx + \int^{\infty}_0 P[X > x]  \rm dx = \int_{-\infty}^{\infty} x \rm d F(x) = E[X]$$

Let $X$ be just integrable now, $X = X^+ - X^-$, and $X^+$, $X^-$ are both integrable.
If we prove the statement for non negative functions, then we can conclude by linearity because
$E[X] = E[X^+] - E[X^-]$ and when the integrals exist we have:
\begin{align*}
\int_{-\infty}^0 (P[X > x] - 1) \rm dx + \int^{\infty}_0 P[X > x]  \rm dx &= \int_{-\infty}^0 (P[-X^- > x] - 1) \rm dx + \int^{\infty}_0 P[X^+ > x]  \rm dx 
\\&=
-\int_0^{\infty} P[X^- \geq x]  \rm dx + \int^{\infty}_0 P[X^+ > x]  \rm dx 
\\&=-\int_0^{\infty} P[X^- > x]  \rm dx  - \int_0^{\infty} P[X^- = x]  \rm dx + \int^{\infty}_0 P[X^+ > x]  \rm dx
\\&=
-\int_0^{\infty} P[X^- \geq x]  \rm dx + \int^{\infty}_0 P[X^+ > x]  \rm dx 
\\&=-\int_0^{\infty} P[X^- > x]  \rm dx + \int^{\infty}_0 P[X^+ > x]  \rm dx \\&\text{because $P[X^- = x]$ is non zero on at most a countable set}
\end{align*}


Let's now suppose $X \geq 0$
Let $X_n := X1_{|X|<n}$, so $X_n \uparrow X$ which is integrable.
\begin{align*}
E[X] &= \lim_n E[|X|1_{|X|<n}] &\text{by monotone convergence}\\
&=\int^{\infty}_0 P[X_n > x]  \rm dx \\
&= \int^{\infty}_0 P[X_n > x]1_{x \geq 0}  \rm dx \\
&= \lim_n \int_{\mathbb{R}} P[X_n > x]
\end{align*}


\begin{align*}\phi_n &:= P[X_n > x] \\
&\leq P[X > x] &\text{because $X_n$ is increasing}\\
&\leq (1 - F(X \leq x))\\
\end{align*}

and we have by right continuety of $F$ and the fact that $X_n \uparrow X$ that:
$\phi_n \uparrow P[X > x]$
By monotne convergence we can swap limit and integral, and we have the equality.

$E[X] = \lim_n \int_{\mathbb{R}} P[X_n > x] = \lim_n \int \phi_n = \int P[X > x]$

\item
Let's show that 

$$\{ u \in (0, 1): u < F(x) \} \subseteq\{ u \in (0, 1): q(u) \leq x \} \subseteq \{ u \in (0, 1): u \leq F(x) \} $$
Let's note that sets $A, B, C$ (it's clear that they are all measurable sets)

Let's prove $A \subset B$.
Let $u \in A$, then $u < F(x)$. Let $y \in \mathbb{R}$ s.t $F(y) \leq u$, then $F(y) < F(x)$, then $y < x$ because $F$ is non-decreasing. By taking the $\sup$, $q(u) \leq x$.


Let's first prove that $F(q(u)) \geq u$. 
Let $\epsilon > 0$ we know that $F(q(u)+\epsilon) \geq u$,  because otherwise $q(u) + \epsilon \in \{x | F(x) \leq u\}$ and $q(u) + \epsilon $ is greater than the sup of this set.
We now take the limit since $F$ is right continous, and we have $F(q(u)) \geq u$. 


Let's now prove that $B \subseteq C$ 
Let $u \in C$, ie $F(x) \geq u$, so $q(F(x)) \leq q(u)$ so $x \leq q(u)$.

We conclude by noting that $\mu(C) = \mu(A)$ because $\mu(C \setminus A) = \mu(\{F(x)\}) = 0$.
And therefore $\mu(A) = \mu(B) = \mu(C)$

But $\mu(C) = \mu((0, F(x)]) = F(x)$


\item 
The distribution of $X$ and $q$ are the same, so the distribution of  $F(X)$ and $F(q)$ are the same.

In the case where $F$ is increasing, $F$ is a bijection and $F(q)$ is the identity of $(0, 1)$, so $F(q) \sim \mathcal{U}\text{nif}(0, 1)$

\end{enumerate}
\end{document}

%%% Local Variables:
%%% mode: latex
%%% TeX-master: t
%%% End:

