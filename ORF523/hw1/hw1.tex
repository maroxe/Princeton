\documentclass[12pt]{article}

% packages
\usepackage{geometry}
\usepackage{amsmath}
\usepackage{amsfonts}
\usepackage{enumitem}

\newcommand{\Q}[1]{\subsubsection*{Q.#1}}
\newcommand{\union}[1]{\underset{#1}{\cup} }
\newcommand{\bigunion}[1]{\underset{#1}{\bigcup} \, }
\newcommand{\inter}[1]{\underset{#1}{\cap} }
\newcommand{\biginter}[1]{\underset{#1}{\bigcap} }

\newcommand{\minimize}[3]{\optimize{#1}{#2}{#3}{min}}
\newcommand{\maximize}[3]{\optimize{#1}{#2}{#3}{max}}

\DeclareMathOperator{\cov}{cov}
\DeclareMathOperator{\var}{var}


% parameters
\geometry{hmargin=1cm,vmargin=1cm}
\title{ORF523 - Problem Set 1}
\author{Bachir EL KHADIR }

\begin{document}
\maketitle
\Q{1}
\begin{enumerate}
\item Let $\lambda$ be an eigen value of $A^TA$ corresponding to an eigen vector $u \ne 0$, then $0 \le ||Au||^2 = u^TA^TAu = \lambda ||u||^2$, therefore $\lambda > 0$.
\item Let $\lambda$ be an eigen value of $A$ corresponding to an eigen vector $u$, the $A^TAu = A(Au) = \lambda^2 u$, so $\lambda^2$ is an eigen value of $A^TA$. Since $A$ has $n$ eigen values (accounting for multplicity), the eigen values of $A^TA$ are exactly the squares of the eigen values of $A$, and therefore the singular values of $A$ are the absolute values of the eigen values of $A$.

\item
  $$u_i^Tu_j = u_i^T\frac{A^TAu_j}{\lambda_j} = \frac{u_i^TA^TA}{\lambda_j}u_j$$
  Since $\lambda_i \ne \lambda_j$, $u_i^Tu_j = 0$

\end{enumerate}
\Q{2}
\begin{itemize}
\item The $L_2$ norm for vectors is unitarly invariant:
  Let $O$ unitary matrix and $X$ a vector, then $||OX||^2  = X^TO^TOX = X^TX = ||X||^2$.
\item Since $O$ is invertible, the application $S \rightarrow S, X \rightarrow OX$, where $S$ is the $L_2$ sphere, is a bijection. So   $\{ x, ||x||_2 = 1\} = \{ Ox, ||x||_2 = 1\}$
\item The $L_2$ norm for matrices is unitarly invariant.
  If $A$ a matrix, then $||AO|| = \max_{||x||_2 = 1} ||AOx|| = \max_{||Ox||_2 = 1} ||AOx|| = \max_{||y||_2 = 1} ||Ay|| = ||A||$ and
  $||OA|| = \max_{||x||_2 = 1} ||OAx|| = \max_{||x||_2 = 1} ||Ax|| = ||A||$.
\item
  Let $B$ be a matrix of rank at most $k$.
  $||A - B|| = ||U(\Sigma - U^TBV)V^T|| = ||\Sigma - U^TBV||$
  Let $U'\Sigma'V'$ be the SVD of $B$, by a similar argument: $||A - B|| = ||\Sigma - \Sigma'||$.
  
  $rank(B) = rank(\Sigma') \le k$, so $\Sigma'$ can be written as $\Sigma'^{(k)} = diag(\sigma_1', \ldots, \sigma'_k, \0 ldots )$.
  $||A - B|| = \sqrt{\sum_{i = 1 ... k} (\sigma_i - \sigma'_i)^2 + \sum_{i = k+1 \ldots n} \sigma_i^2}
  = \sqrt{\sum_{i = 1 ... k} (\sigma_i - \sigma'_i)^2 + ||A - A^{(k)}||^2} \ge ||A - A^{(k)}||$,
  Since $rank(A^{(k)}) = k$ we have proved the result.
\end{itemize}

\Q{3}
\begin{enumerate}
\item
\item To store $A^{(k)}$ we need only to store $\Sigma^{(k)}$ ($k$ parameters), $U^{(k)}, V^{(k)}$ (each of them has only at most $k$ column not set to zero, so they take $nk + mk$), while $A$ takes $nm$ numbers to store.
  In conclusion, we save ($nm - k(n+m+1)$).
\item 
\end{enumerate}
\end{document}









