\documentclass[12pt]{article}

% packages
\usepackage{geometry}
\usepackage{amsmath}
\usepackage{amsfonts}


\geometry{hmargin=1cm,vmargin=1cm}


% custom commands
\newcommand{\Q}[1]{\subsubsection*{Question #1}}
\newcommand{\salgebra}{$\sigma$-algebra }

\newcommand{\union}[1]{\underset{#1}{\cup} }
\newcommand{\bigunion}[1]{\underset{#1}{\bigcup} \, }
\newcommand{\inter}[1]{\underset{#1}{\cap} }
\newcommand{\biginter}[1]{\underset{#1}{\bigcap} }

% parameters
\title{ORF526 - Problem Set 3}
\author{Bachir EL KHADIR }

\begin{document}

\maketitle

\Q{1}

\begin{itemize}
\item [a)] $\cap_{n \in \mathbb{N}} A_n^c$
\item [b)] $\cap_{m \in \mathbb{N}} \cup_{m \leq n} A_n$
\item [c)] That's the opposite of $b)$, $\left(\cap_{m \in \mathbb{N}} \cup_{m \leq n} A_n \right)^c = \cup_{m \in \mathbb{N}} \cap_{m \leq n} A_n^c $
\item [d)] $\omega$ has to be in exactly two of the $A_i$, ie $\cap_{ i,j \in \mathbb{N}, i \neq j} \left(A_i \cap A_j \cap (\cup_{n \neq i, n \neq j} A_n^c) \right)$
\item [e)] This event can be expressed as "$\Phi$ nevr occurs at even times", ie $\cap_{n \in \mathbb{N}} A_{2 n}^c$
\end{itemize}

\Q{2}
\begin{itemize}
\item $\varepsilon \subseteq \sigma(\varepsilon)$, so $\{ f^{-1}(A) : A \in \varepsilon \} \subseteq \{ f^{-1}(A) : A \in \sigma(\varepsilon) \}$. since the RHS is already a $\sigma$-algebra (showed in class), $\sigma \{ f^{-1}(A) : A \in \varepsilon \} \subseteq \{ f^{-1}(A) : A \in \sigma(\varepsilon) \}$
\item Let's note $B := \sigma \{ f^{-1}(A) : A \in \varepsilon \}$, and $C := \{ A: f^{-1}(A) \in B \}$.
\begin{itemize}
\item $C$ is a \salgebra containing $\varepsilon$, so $\sigma(\varepsilon) \subseteq C$
\item As a consequence, for every $A \in \sigma(\varepsilon)$, $A \in C$, ie $f^{-1}(A) \in B$.
\end{itemize}
We have just proved that $\{ f^{-1}(A) : A \in \sigma(\varepsilon) \}  \subseteq B$
\end{itemize}


\Q{3} 
\begin{itemize}
\item $X = \lim_n X_n = \lim_n \text{sup}_{k \geq n} X_k$
\item For $x \in  \mathbb{R} \cup \{ \infty \}$, $X \leq x$ is quivalent to $\exists n \, \forall k \, \geq n \, X_k \leq x$
\item $\{X \leq x\} = \cup_{n \in \mathbb{N}} \cap_{k \geq n} \{X_k \leq x \}$ is then measuable.
\end{itemize}

\Q{4}

\begin{itemize}

\item
Let's call $Q = P(\{1,..,n\})$. For $i = 1 .. n$:
$$A_i = \bigunion{I \subseteq Q, i \in I} (\inter{k \in I} A_k) \inter{} (\inter{k \in I^c} A_k^c) $$
Note that this a union of disjoint sets.


Let's call $I_i := \{ I \in Q: \sum_{i \in I} a_i = x_i\}$, ie the different possible combinations for the $A_i$ where $\omega \in \Omega$ can be so that its image by $f$ equals $x_i$. Note that $\omega$ can not be  in any other set $A_i$, for $i \not \in I_i$ bacause $a_i > 0$.


Written differently, $\{ f = x_i \} = \bigunion{I \in I_i} \inter{k \in I_i} A_k \inter{} \inter{k \in I_i^c} A_k^c$. And as result of the sets being disjoint: 
$$ \mu(f = x_i) = \sum_{I \in I_i} \mu(\inter{k \in I_i} A_k \inter{} \inter{k \in I_i^c} A_k^c)$$


Note that any sum index by some $I \in Q$ in finite because $|I| \leq n$, and this we can rearrange the sums in any order.

\begin{align*}
\sum_{i=1}^m x_i \mu(f = x_i) 
&= \sum_{i=1}^m x_i \sum_{I \in I_i} \mu(\inter{k \in I} A_k \inter{} \inter{k \in I^c} A_k^c) \\
&= \sum_{i=1}^m  \left(\sum_{I \in I_i} (\sum_{k \in I} a_k) \mu(\inter{k \in I} A_k \inter{} \inter{k \in I^c} A_k^c)\right) \\
&= \sum_{I \in Q} (\sum_{k \in I} a_k) \mu(\inter{k \in I} A_k \inter{} \inter{k \in I^c} A_k^c) 
& \text{because $Q = \union{k=1..m} I_i$} \\
&= \sum_{i=1..n} \sum_{I \in Q, i \in I} a_i \mu(\inter{k \in I} A_k \inter{} \inter{k \in I^c} A_k^c) 
& \text{By rearranging the sum}\\
&= \sum_{i=1..n} a_i \sum_{I \in Q, i \in I}  \mu(\inter{k \in I} A_k \inter{} \inter{k \in I^c} A_k^c) \\
&= \sum_{i=1..n} a_i \mu(A_i)
\end{align*}

\item
Let's first prove that if a set $A$ has measure 1, for all measurable sets $B$, $\mu(A \inter{} B) = \mu(B)$. This holds because 

$$\mu(B) \geq \mu(A \inter{} B) = 1 - \mu(A^c \union{} B^c) \geq 1 - \mu(B^c) = \mu(B)$$


Let's now prove that $f(\omega) = g(\omega)$.
Let $x$ in the set on the left

$\mu(\{g = x\}) \geq \mu(\{g = x\} \inter{} \{f=x\}) = \mu(\{f = x\} \inter{} \{f=g\}) = \mu(\{f = x\})$ 
Symmetrically, we prove that $\mu(\{f = x\}) \geq \mu(\{g = x\})$, and thus this two quantities are equal.

This proves that in the sum $\sum_{x \in f(\Omega)} x \mu(f = x)$ there is a term that is non zero, $\mu(\{f = x\}) = \mu(\{g=x\}) \neq 0$, and $x \in g(\Omega)$. Since all quantities are positive, this means, $\sum_{x \in f(\Omega)} x \mu(f = x) \leq \sum_{x \in g(\Omega)} x \mu(g = x)$, and by symmetry: $\sum_{x \in f(\Omega)} x \mu(f = x) = \sum_{x \in g(\Omega)} x \mu(g = x)$

\end{itemize}

\newpage

\Q{5}


Let's first suppose that $h := \sum_{i=1..n} a_i 1_{A_i}$ for $a_i > 0$ and $A_i \in \sigma(f)$. Let $h(\Omega_1) = \{x_1, ..., x_m\}$.


We can write $h = \sum_{j=1..m} x_j 1_{\{h=x_j\}}$
$$\{ h = x_j \} = f^{-1}(A_j)$$

$A_j$ are disjoint because for $i \neq j$, $f^{-1}(A_i \inter{} A_j) \subseteq \{h = x_i\} \inter{} \{h = x_j\} = \emptyset$


We define $g(x) = x_i$ on $A_i$, and any other value outside. 

Let $\omega \in \Omega_1$, and $x_i = h(\omega)$, then $\omega \in f^{-1}(A_i)$, so $f(\omega) \in A_i$ and $gof(\omega) = x_i = h(\omega)$. So $h = gof$.


Let $h$ be now just non negative.


We can approximate $h$ point wise by a sequence of non negative simple functions $h_n$ measurable w.r.t the $\sigma$-algebra generated by $f$. We know now that $h_n$ can be written as $g_n o f$.

By convergence of $h_n$, the sequence $g_n$ converges point wise in the image of $f$. Let call $g$ the limit, and set $g$ to 0 outside the image of $f$, then $h = gof$. and $g = (\text{limsup}_n \, g_n) 1_{f(\Omega_1)}$ is measurable as the product of two measurable functions.

Let $h$ be just measurable, $h = h^+ - h^- = g^+of - g^-of = (g^+ - g^-) o f$.

If we consider the trivial $\sigma$-algebra $\{\mathbb{R}, \emptyset \}$ instead of Borel, and $f$ constant equal to $0$, then any $h$ is measurable w.r.t the $\sigma$ algebra generated by $f$. If now $h$ is not constant, it cannot be written as $gof = g(0)$
\end{document}

%%% Local Variables:
%%% mode: latex
%%% TeX-master: t
%%% End:

