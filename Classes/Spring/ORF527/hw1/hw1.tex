
\documentclass[12pt]{article}

% packages
\usepackage{geometry}
\usepackage{amsmath}
\usepackage{amsfonts}
\usepackage{enumitem}

\newcommand{\Q}[1]{\subsubsection*{Q.#1}}
\newcommand{\union}[1]{\underset{#1}{\cup} }
\newcommand{\bigunion}[1]{\underset{#1}{\bigcup} \, }
\newcommand{\inter}[1]{\underset{#1}{\cap} }
\newcommand{\biginter}[1]{\underset{#1}{\bigcap} }

\newcommand{\minimize}[3]{\optimize{#1}{#2}{#3}{min}}
\newcommand{\maximize}[3]{\optimize{#1}{#2}{#3}{max}}

\DeclareMathOperator{\cov}{cov}
\DeclareMathOperator{\var}{var}


% parameters
\geometry{hmargin=1cm,vmargin=1cm}
\title{ORF527 - Problem Set 1}
\author{Bachir EL KHADIR }

\begin{document}
\maketitle
\Q{1}

Let $\varepsilon > 0, x \in [0, 1]$.

\begin{itemize}
\item $\sup |f^n - f| \rightarrow_n 0$, let $n \in \mathbb N$
  $\forall t \in [0, 1] \; |f^n(t) - f(t)| \le \frac{\varepsilon}3$.

\item $f^n$ is continuous, let $\delta > 0$ such that
  $\forall y \in [0, 1], |x-y| < \delta \Rightarrow |f^n(x) - f^n(y)|
  < \frac{\varepsilon}3$.  
\item
  Let $y \in [0, 1]$ such that $|x - y| < \delta$
  \begin{align*}
    |f(x) - f(y)|
    & \le |f(x) - f^n(x)| + |f^n(x) - f^n(y)| + |f^n(y) - f(y)|
    \\&\le 3 \frac{\varepsilon}3
    \\&\le \varepsilon 
  \end{align*}
  Which conclude the proof.
\end{itemize}

\Q{2}
\subsubsection*{3.2}

let $X \sim \mathcal N(0,1)$, and $\varepsilon \sim \mathcal B(-1, 1, \frac1 2)$ be two independant rv. And Let $Y = \varepsilon X$
\begin{itemize}
\item By symmetry of the distribution of $X$:
$F_Y(y) = P(Y \leq y) = P(\varepsilon X \leq y) = E[ P(\varepsilon X \leq y | \varepsilon) ] = \frac1 2 P(X \leq y) + \frac1 2 P(- X \leq y) = P(X \leq x)$
so $Y \sim \mathcal N(0, 1)$.

\item $cov(X, Y) = E[X^2 \varepsilon] = E[X^2]E[\varepsilon] = 0$
\item  $X, Y$ are not independent. Indeed, Let $\alpha := \mathbb P(|X| > 0.5)$.
  \begin{itemize}
  \item $P(|X| > 0.5, |Y| > 0.5) = P(|X| > 0.5) = \alpha$,
  \item $P(|X| > 0.5) P(|Y| > 0.5) = P(|X| > 0.5)^2 = \alpha^2$
  \end{itemize}
  But since $\alpha \not \in \{0, 1\}$, $\alpha \ne \alpha^2$.
\end{itemize}

\subsubsection*{3.3}
For a random variable $X$, let's call $\Phi_X$ its caracteristic function.
\begin{enumerate}[label=(\alph*)]
\item 
  For $t \in \mathbb R^n$, $\Phi_{AV}(t) = \mathbb E[ e^{i V^TA^Tt} ] = \Phi_{V}(A^Tt) = e^{(A\mu)^Tt - \frac12 (t^T A \Sigma A^T t)}$, so $V \sim \mathcal N(A\mu, A\Sigma A^T)$
\item By symmetry of the gaussisan distribution, $-Y$ has the same distribution as $Y$. So it suffices to show that the result holds for $X+Y$.
  By independence: $\Phi_{X+Y}(t) = \Phi_X(t) \Phi_Y(t) = \Phi_X(t)^2 = e^{i2\mu t - \frac12 2 t^2 }$, so $X+Y \sim \mathcal N(0, 2)$.
\item If $cov(X, Y) = 0$, the covariance matrix of the guassian process $(X, Y)$ is diagonal, therefore
  $$\forall t = (t_1, t_2) \in \mathbb R^2 \; \Phi_{(X, Y)}(t) = E[e^{i \mu_X t_1 + i \mu_Y t_2 - \frac12 \sigma_X^2t_1^2 - \frac12 \sigma_Y^2t_2^2}] = \Phi_X(t_1) \Phi_Y(t_2)$$
  Where $X \sim \mathcal N(\mu_X, \sigma_X^2), Y \sim \mathcal N(\mu_Y, \sigma_Y^2)$.
  Therefore $X, Y$ are independent.
\item
  We consider that the guaussian vector is non degenerate otherwise one of the variables is proportinal to the other.
  
  $Y | X = x$ is gaussian because it has the density $f_{Y | X=x}(y) = \frac{f(x, y)}{\int_R f(x, s) ds} \propto e^{-Q(y)}$ where is a quadratic form in $y$.
  
  Let $\alpha := \frac{\cov(X, Y)}{\var(X)}$. $(Y - \alpha X, X)$ is gaussian and $\cov(Y - \alpha X, X) = 0$. So $Y - \alpha X \perp X$. Therefore:
  $$\mu_{Y | X = x} = E[Y | X = x] = E[Y - \alpha X]+ E[\alpha X | X = x] = \mu_Y - \alpha \mu_X + \alpha x = \mu_Y + \frac{\cov(X,Y)}{\var X} (x - \mu_X)$$
  $$\sigma_Y^2 = \var(Y | X = x) = \var(Y - \alpha X) + \underbrace{\var(\alpha X | X = x)}_{=0} = \var Y + \alpha^2 \var X - 2\alpha \cov(X, Y) = \sigma_Y^2 - \frac{\cov(X,Y)^2}{\var X}$$
  
\end{enumerate}

\Q{3}
\begin{enumerate}[label=(\alph*)]
\item Since $\forall n > 0, \Delta_n(1) = 0$, $B_1 = \sum_{n=0}^{\infty} \lambda_n Z_n \Delta_n(1) = \lambda_0 Z_0 \Delta_0(1) = \lambda_0Z_0$.
  But $B_t - U_t = \lambda_0 Z_0 \Delta_0(t) = t\lambda_0 Z_0 = t B_t$, which proves the result.
\item
  If $s < t$, then $cov(B_s, B_t) = \cov(B_s - B_t, B_s) + \var(B_s) = s$.
  
  $\cov(U_s, U_t) = \cov(B_s - sB_1, B_t -tB_1) = \cov(B_s, B_t) + ts \cov(B_1, B_1) - t \cov(B_s, B_1) - s\cov(B_1, B_t) = s + ts - ts - ts = s - ts = s(1- t)$.

\item
  For $s < t$, $cov(X_t, X_s) = g(t)g(s) (h(s) \wedge h(t))$, should be equal to $s(1-t)$.
  We can take $g: t \rightarrow 1 - t$, $h: s \rightarrow \frac s{1 - s}$ defined on $(0, \frac12)$

\item
  $U_t$ is a gaussian because for all $t_1 \ldots t_n$, a linear combination oft $(U_{t_1}, \ldots U_{t_n})$ is a linear combination of  $(B_1, B_{t_1}, \ldots B_{t_n})$ which is guaussian.
  For the same reasons, $(U_{\frac t{1+t}})_{t > 1}$ is a guaussian process, and so is $Y_t$.

  For $s < t, \frac s{s+1} < \frac t{t+1} $: $\cov(Y_t, Y_s) = (1+t)(1+s)\cov(U_{\frac t{t+1}}, U_{\frac s{s+1}}) = (1+t)(1+s) \frac s{s+1} (1 - \frac t{t+1}) = s$
  
  $Y_t$ is a guaussian process with the same mean (0) and covariance as a Brownian Motion, therefore it has the same distribution  as a Borwnian Motion.
  
  In addition, $Y_t$ is continuous as a composition of two continuous functions, therefore it is a brownian motion.
  
\end{enumerate}
\end{document}

%%% Local Variables:
%%% mode: latex
%%% TeX-master: t
%%% End:













