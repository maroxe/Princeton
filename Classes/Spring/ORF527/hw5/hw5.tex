% Created 2016-03-09 Wed 10:46
% Intended LaTeX compiler: pdflatex
\documentclass[11pt]{article}
\usepackage[utf8]{inputenc}
\usepackage[T1]{fontenc}
\usepackage{graphicx}
\usepackage{grffile}
\usepackage{longtable}
\usepackage{wrapfig}
\usepackage{rotating}
\usepackage[normalem]{ulem}
\usepackage{amsmath}
\usepackage{textcomp}
\usepackage{amssymb}
\usepackage{capt-of}
\usepackage{hyperref}
\usepackage{minted}
\usepackage[margin=0.5in]{geometry}
\usepackage{amsmath}
\usepackage{amsfonts}
\newcommand{\Problem}[1]{\subsection*{Problem #1}}
\newcommand{\Q}[1]{\subsubsection*{Q.#1}}
\newcommand{\union}[1]{\underset{#1}{\cup} }
\newcommand{\bigunion}[1]{\underset{#1}{\bigcup} \, }
\newcommand{\inter}[1]{\underset{#1}{\cap} }
\newcommand{\biginter}[1]{\underset{#1}{\bigcap} }
\newcommand{\minimize}[3]{\optimize{#1}{#2}{#3}{min}}
\newcommand{\maximize}[3]{\optimize{#1}{#2}{#3}{max}}
\DeclareMathOperator{\cov}{cov}
\DeclareMathOperator{\var}{var}
\author{Bachir El khadir}
\date{\textit{<2016-03-09 Wed>}}
\title{Problem set 5, ORF527}
\hypersetup{
 pdfauthor={Bachir El khadir},
 pdftitle={Problem set 5, ORF527},
 pdfkeywords={},
 pdfsubject={},
 pdfcreator={Emacs 24.5.1 (Org mode )}, 
 pdflang={English}}
\begin{document}

\maketitle

\section{Q1 (7.2 in Steele)}
\label{sec:orgheadline1}

Let \(\tau_n = inf\{t, |X_t| \ge n\}\) be a localizing sequence of \((X_t)\), so that \(X_{t \wedge \tau_n}\) is bounded. \(\tau_n\) is non-decreasing and diverges to \(\infinity\) because the continuous function \(t \rightarrow X_t\) is bounded on every compact set \([0, T]\), and if \(n\) is larger than this bounde then \(\tau_n \ge T\).

Since \(\phi\) is continuous, \(Y_{t \wedge \tau_n} = \phi(X_{t \wedge n})\) is bounded.


So for \(s < t\), \(E[Y_{t \wedge \tau_n} | F_s] = E[\phi(X_{t \wedge \tau_n}) | F_s] \underbrace{\le}_{\text{Jensen}} \phi(E[X_{t \wedge \tau_n} | F_s]) = \phi(Y_{s \wedge n})\)
Counter example:
\begin{itemize}
\item \(\Omega = (0, 1)\), \(P\) the uniform measure.
\item \(\phi(x) = x^2\)
\item \(X(t, \omega) = \frac{1}{\omega^{\frac 12}}\) is a integrable constant in \(t\), so it is a martingale, but \(\phi(X) = \frac{1}{\omega}\) is not integrable.
\end{itemize}


\section{Q1 (7.3)}
\label{sec:orgheadline2}
Let \(X_t\) be a continuous local submartingale martingale verifying (7.35), and \(\tau_n\) a localizing sequence, and \(s < t < T\), then:

\begin{itemize}
\item \(E[|X_t|] \le E[\sup_{[0, T]} |X_s|] < \infty\)
\item \(E[X_{t \wedge \tau_n} | F_s] \ge X_{s \wedge \tau_n}\)
\end{itemize}

Using the fact that \(X_{s \wedge \tau_n}\) is uniformly bounded by the \(L_1\) function \(\sup_{[0, T]} |X_s|\), we use dominated convergence theorem to prove that
\(E[X_t | F_s] \ge X_s\), so \(X_s\) is a submartingale.

A bounded local martingale trivially verifies (7.35)

\section{Q2}
\label{sec:orgheadline3}
a. By Ito isometry and linearity:
$$E[ (\int_0^T X_s^n dW_s - \int_0^T X_s dW_s)^2 ] =  E[ \int_0^T (X_s^n - X_s)^2 ds ] \rightarrow 0$$

b.  \(\tau_n = \inf \{t ,  \int_0^t X_s^2 ds \ge n\} \wedge T\) is a localizing sequence.
By markov inequality:
$$P(|\int_0^T X_{t \wedge \tau_n} dW_t| \ge \varepsilon) \le \frac{E[|\int_0^T X_{t \wedge \tau_n} dW_t|^2]}{\varepsilon^2}$$
By Ito:
$$P(|\int_0^T X_{t \wedge \tau_n} dW_t| \ge \varepsilon) \le \frac{E[\int_0^T X^2_{t \wedge \tau_n} dt]}{\varepsilon^2}$$

\begin{itemize}
\item For \(0 < \delta < \varepsilon\)
\begin{align}
P(|\int_0^T X_t dW_t| \ge \varepsilon)
&\le P(|\int_0^{\tau_n} X_t dW_t| \ge \varepsilon-\delta) + P(|\int_{\tau_n}^T X_t dW_t| \ge \delta)
\\&\le \frac{E[|\int_0^{\tau_n} X_t dW_t|^2]}{(\varepsilon-\delta)^2} + P(|\int_{\tau_n}^T X_t dW_t| \ge \delta)
\\&\le \frac{E[\int_0^{\tau_n} X_t^2 dt]}{(\varepsilon-\delta)^2} + P(|\int_{\tau_n}^T X_t dW_t| \ge \delta)
\\&\le \frac{N}{(\varepsilon-\delta)^2} +  P(\tau_n < T)
\\&\le \frac{N}{(\varepsilon-\delta)^2} +  P(\int_0^T X_s^2 \ge N)
\end{align}

We get the result by taking  \(\delta\) to 0.
\end{itemize}

c. By b.
$$P(|\int_0^T (X_t - X^n_t) dW_t|^2 > \varepsilon) \le P(\int_0^T (X_t - X^n_t)^2 dt \ge \varepsilon) + \frac N{\varepsilon^2}$$
Taking the \(\lim\sup\) with respect to \(n\):
  $$\lim\sup_n P(|\int_0^T (X_t - X^n_t) dW_t|^2 > \varepsilon) \le \frac N{\varepsilon^2}$$
  And thus for all \(N > 0\). We conclude by taking the \(N \rightarrow 0\).

d. Let \(X \in \mathcal H^{loc}[0, T]\), let \(\tau_n\) be a localizing sequence, so that \(X1_{[0, \tau_n]} \in \mathcal H[0, T]\).

Since \(\mathcal H_0[0, T]\) is dense in \(\mathcal H[0, T]\) with respect to the \(L_2(\Omega \times [0, T])\) norm, there exist a sequence \(X_n \in \mathcal H_0\) such that:
\(E[\int_0^T (X1_{[0, \tau_n]}(s) - X_n(s))^2 ds] \rightarrow_n 0\).

\(P(\int_0^T (X(s)1_{[0, \tau_n]}(s) - X_n(s))^2 ds > \varepsilon) \le \frac{E[\int_0^T (X1_{[0, \tau_n]}(s) - X_n(s))^2 ds]}{\varepsilon} \rightarrow_n 0\)

So \(\int_0^T (X(s)1_{[0, \tau_n]}(s) - X_n(s))^2\) converges to 0 in probability.

We also know that \(\int_0^T X^2(s)1_{[0, \tau_n]}(s) ds \rightarrow \int_0^T X^2(s) ds\) alsmost surely, and thus in probability.

Now, \(\int_0^T (X(s) - X_n(s))^2 ds \le \int_0^T (X(s) - X(s)1_{[0, \tau_n]}(s))^2 + \int_0^T (X1_{[0, \tau_n]}(s) - X_n(s))^2 \overset{P}{\rightarrow} 0\), which gives the result.

Using c., We can definie the integral of  \(X \in \mathcal H^{loc}[0, T]\) as the limit in probability of a sequence of simple function that converge to \(X\) in the sense of c.

\section{Q3}
\label{sec:orgheadline4}

All functions considered here are \(\mathcal H^{loc}\) as continuous function / integrals of brownian motions.
a. \(d(e^tW_t) = e^tW_t dt + e^tdW_t\)

b. \(f x \rightarrow \frac1{1+x^2}\), \(f'(x) = \frac{-2x}{(1+x^2)^2}\), \(f''(x) = \frac{-2(1+x^2)^2 + 8x^2(1+x^2)}{(1+x^2)^4} = \frac{-2 + 6x^2}{(1+x^2)^3}\)
$$d(1+W_t^2)^{-1} = \frac{-2W_t}{(1+W_t^2)^2}dW_t + \frac{-1 + 3W_t^2}{(1+W_t^2)^3}dt$$
and at \(0\) the value is 1.

c.
\begin{itemize}
\item \(Y_t = \int_0^t \sqrt{|W_s|} dW_s\), \(dY_t = \sqrt{|W_s|} dW_s\)
\item \(d \cos(Y_t) = -\sin(Y_t) dY_t - \frac12 \cos(Y_t) d < Y >_t = -\sin(Y_t) dY_t - \frac12 \cos(Y_t) |W_s| ds\)

\item \(Z_t = e^{\alpha W_t + \sigma t}\)
\item \(dZ_t = \sigma Z_t dt + \alpha Z_t dW_t + \frac12 \alpha^2 Z_t dt = Z_t( (\sigma + \frac12 \alpha^2)dt + \alpha dW_t)\)
\item \(U_t = e^{\alpha W_t + \sigma t} \cos(\int_0^t \sqrt{|W_s|} dW_s)\)
\item \(V_t = e^{\alpha W_t + \sigma t} \sin(\int_0^t \sqrt{|W_s|} dW_s)\)
\item \(d\cos(Y_t) dZ_t = -\alpha sin(Y_t)\sqrt{|W_t|}Z_t dt = -\alpha \sqrt{|W_t|} V_t\)
\end{itemize}

\begin{align*}
d(e^{\alpha W_t + \sigma t} \cos(\int_0^t \sqrt{|W_s|} dW_s))
&= \cos(Y_t) dZ_t + Z_t d\cos(Y_t) + dZ_t dcos(Y_t)
\\&= \cos(Y_t)  Z_t( (\sigma + \frac12 \alpha^2)dt + \alpha dW_t) - Z_t (\sin(Y_t) dY_t + \frac12 \cos(Y_t) |W_t| dt)
- \alpha sin(Y_t)\sqrt{|W_t|}Z_t dt
\\&= U_t  (\sigma + \frac12 \alpha^2)dt + U_t \alpha dW_t - Z_t \sin(Y_t) dY_t - \frac12 U_t |W_t| dt
- \alpha sin(Y_t)\sqrt{|W_t|}Z_t dt
\\&= U_t(\sigma + \frac12 \alpha^2 - \frac{|W_t|}2) dt + (\alpha U_t - \sin(Y_t) \sqrt{|W_t|}Z_t) dW_t
- \alpha sin(Y_t)\sqrt{|W_t|}Z_t dt
\\&= \left(U_t(\sigma + \frac12 \alpha^2 - \frac{|W_t|}2) - \alpha  \sqrt{|W_t|} V_t \right) dt + (\alpha U_t -  \sqrt{|W_t|} V_t) dW_t
\end{align*}

at \(0\) the value is 1.
d.
\begin{itemize}
\item \(U_t = \int_0^t W_s d\tilde W_s\)
\item \(dU_t = W_s d\tilde W_s\)
\item \(V_t = W_t U_t\)
\item \(dV_t = W_t dU_t + U_t dW_t = W_t^2 d\tilde W_t + U_t dW_t\)
\item \(d\exp(V_t) = \exp(V_t)(dV_t + \frac12( W_t^4 + U_t^2)dt)\)
\end{itemize}


\begin{align*}
d(\exp(W_t \int_0^t W_s d\tilde W_s)W_t)
& = d(\exp(V_t)W_t)
\\& = W_t d(\exp(V_t)) + \exp(V_t) dW_t + d(\exp(V_t)) dW_t
\\& = W_t \exp(V_t)(dV_t + \frac12( W_t^4 + U_t^2)dt) + \exp(V_t) dW_t  + \exp(V_t) U_t dt
\\& = W_t \exp(V_t)(W_t^2 d\tilde W_t + U_t dW_t + \frac12(W_t^4 + U_t^2)dt) + \exp(V_t) dW_t  + \exp(V_t) U_t dt
\\& = \exp(V_t) W_t^3 d\tilde W_t + \exp(V_t) \left( W_t U_t + 1 \right) dW_t  + \exp(V_t) \left(\frac12 W_t^5 + \frac12 W_t U_t^2 +  U_t\right) dt
\end{align*}

at \(0\) the value is \(0\)

e.
Ito Forumla:
$$cos(W_t) = cos(0) - \underbrace{\int_0^t \sin(W_s) dW_s}_{\text{martingale}} - \frac12 \int_0^t cos(W_s)ds$$

Taking the expectation on both sides, and swapping \(E\) and \(\int\) because \(\cos\) is bounded:
$$E[cos(W_t)] = 1 - \frac12 \int_0^t E[cos(W_s)] ds$$

So \(t \rightarrow E[cos(W_t)]\) is solution the differential equation: \(f = 1 - \frac12 \int_0^t f\)
Since the solution is unique(\(e^{-\frac s2}\)):
$$\log(E[cos(W_t)]) = -\frac{t}2$$


$$\frac{\partial}{\partial t} \logE[cos(W_t)] = -\frac12$$
\end{document}