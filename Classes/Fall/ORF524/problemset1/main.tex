\documentclass[12pt]{article}

% packages
\usepackage{geometry}
\usepackage{amsmath}
\usepackage{amsfonts}

% custom commands
\newcommand{\Q}[1]{\subsubsection*{Question #1}}

% parameters
\geometry{hmargin=1cm,vmargin=1cm}
\title{ORF524 - Problem Set 1}
\author{Bachir EL KHADIR }

\begin{document}

\maketitle

\Q{1}
\begin{itemize}
\item [1)] Let $A_1, ..A_2 \in \Sigma$, so $\cup_i A_i = (\cap_i A_i^c)^c \in \Sigma$
\item [2)] $\Sigma \neq \emptyset$. Let $A \in \Sigma$, we have $\Omega = A \cup A^c$ and $\emptyset = A \cap A^c$ are both in $\Sigma$
\item [3)] Let $\Sigma_A$ be the smallest algebra containing $A$. By definiton, we have $\Omega, \emptyset, A, A^c \in \Sigma_A$. Conversely, it's easy the see that $\{\emptyset, \Omega, A, A^c\}$ is an algebra
\end{itemize}


\Q{2}
$P(A) \geq 0$ because $P$ is a measure.
$P(A) = P(\Omega) - P(A^c) = 1 - P(A^c) \leq 0 $


\Q{3}
\begin{itemize}
\item [1)] $mathbb{R} = \cap_{n \in \mathbb{Z}} [n, n+1]$ and $\mu([n, n+1]) = 1$
\item [2)] If $\Omega$ is countable, then there exist a sequence $(a_i)_{i \in \mathbb{N}}$ such that $\Omega = \cup_i {a_i}$. $\Omega$ is then $\sigma$-finite because $\mu({a_i}) = 1$. Conversly, if $\Omega$ is  $\sigma$-finite, there $\Omega$ can be written as a union of countably many finite sets, $\Omega$ is then countable.
\end{itemize}

\Q{4}
\begin{itemize}
\item [1)]
  Let's first suppose that $f \geq 0$.

  Let $\phi = \sum_i a_i 1_{A_i}$ be a simple function such that $0 \leq \phi \leq f$,
  then $\int \phi = \sum_i a_i P(A_i) = \sum_i a_i \sum_{\omega} 1_{A_i}(\omega) = \sum_{\omega} \phi(\omega)$
  
  $\int f d\mathbb{P} = \sup_{\phi \leq f} \int \phi = \sup_{\phi \leq f} \sum_{\omega} \phi(\omega) = \sum_{\omega}  \sup_{\phi \leq f} \phi(\omega) = \sum_{\omega} f(\omega)$

  In the general case, $f = f^+ - f^-$, it's easy to apply the previous proof to $f^+$ and $f^-$ (if well defined) and conclude because of linearity.
  
\item [2)] $P_f$ is a probability measure because:
  \begin{itemize}
  \item $P_f \geq 0$ becase $f \geq 0$.
  \item $P_f$ is $\sigma$-additive because if $\{A_i \subset \Omega, i \in \mathbb{N}\}$ is a set of disjoint sets, then $P_f(\cup_i A_i) = \sum_{\omega \in \cup_{A_i}} f(\omega) = \sum_i \sum_{\omega \in A_i} f(\omega) = \sum_i P(A_i)$. We could re-arrange the terms because they are all positive.
  \item $P(\omega) = \sum f(\omega) = 1$
  \end{itemize}

  
\end{itemize}

\end{document}

%%% Local Variables:
%%% mode: latex
%%% TeX-master: t
%%% End:
