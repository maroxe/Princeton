\documentclass[12pt]{article}

% packages
\usepackage{geometry}
\usepackage{amsmath}
\usepackage{amsfonts}
\usepackage[]{mcode}

% custom commands
\newcommand{\Q}[1]{\subsubsection*{Problem #1}}
\newcommand{\optimize}[4]
{
\begin{align*}
& \underset{#1}{\text{#4}}
& & #2 \\
& \text{subject to}
& & #3
\end{align*}
}



\newcommand{\union}[1]{\underset{#1}{\cup} }
\newcommand{\bigunion}[1]{\underset{#1}{\bigcup} \, }
\newcommand{\inter}[1]{\underset{#1}{\cap} }
\newcommand{\biginter}[1]{\underset{#1}{\bigcap} }

\newcommand{\minimize}[3]{\optimize{#1}{#2}{#3}{min}}
\newcommand{\maximize}[3]{\optimize{#1}{#2}{#3}{max}}

\newcommand{\Ab}[9]{
\left(
\begin{array}{ccc}
#1&#2&#3 \\
#4&#5&#6 \\
#7&#8&#9 \\
\end{array} 
\right)
}
\newcommand{\Dir}[5]{
\left(
\begin{array}{c}
#1\\#2\\#3\\#4\\#5
\end{array} 
\right)
}


\newcommand{\BDir}[8]{
\left(
\begin{array}{c}
#1\\#2\\#3\\#4\\#5\\#6\\#7\\#8
\end{array} 
\right)
}


% parameters
\geometry{hmargin=1cm,vmargin=1cm}
\title{ORF524 - Problem Set 2}
\author{Bachir EL KHADIR }

\begin{document}

\maketitle

\Q{1}

Let's consider the following optimization problem $P$:
\maximize{x}{0}{Ax \leq b}
and its dual $D$:
\minimize{y \geq 0}{y^T b}{y^TA = 0}

If $1.$ is feasible, then $0$ is the optimal solution. If $2.$ is feasible, then the optimal value of the dual problem is negative. By the duality theorem, we conclude that the two systems cannot be feasible at the same time.


If $1.$ is infeasable, it means that $P$ is unfeasable, which means that $D$ is either unbounded or unfeasble. But since $0$ is a feasable solution for $D$, $D$ must be unbounded, and there must exist $y$ such that $y^Tb < 0$, with $y \geq 0$ and $y^T A = 0$

\Q{2}
\begin{itemize}

\item
\maximize{f_{u,v} \geq 0}{ \sum_{v: (s,v) \in E} f_{s,v}}
{  \forall (u,v) \in E \, f_{u,v} \leq w(u, v) \\ &&& \forall v \in V \setminus \{s,t\} \sum_{u: (u,v) \in E} f_{u,v} = \sum_{u: (v, u) \in E} f_{v,u}}

If we set $w_{u,v}$ to 0 when $(u,v) \not \in E$, we can rewrite the problem as:
\maximize{f_{u,v} }{ \sum_{v: (s,v) \in E} f_{s,v}}
{  \forall u,v \in V \, 0 \leq f_{u,v} \leq w(u, v) \\ &&& \forall v \in V \setminus \{s,t\} \sum_{u \in V} f_{u,v} = \sum_{u \in E} f_{v,u}}


Or in vectorial form:

\maximize{f \in R^{|V|^2}, f \geq 0 }{ c^T f}
{  0 \leq f \leq w \\ &&& Af = 0}


%%% code matlab 1 %%%

The solution is: 
\[ f = 
\left(
\begin{array}{ccccccc}
  0&4&2&8&0&0&0\\
  0&0&0&1&3&0&0\\
  0&0&0&0&0&2&0\\
  0&0&0&0&4&5&0\\
  0&0&0&0&0&0&8\\
  0&0&0&0&1&0&6\\
  0&0&0&0&0&0&0
\end{array}
\right)
\]

$c^T f = 14$

\item Let's call $g, u$ the dual variables (for the first and second constraint resp.)

The dual can be written as
$$\text{max}_{f \geq 0 } \, { \text{min}_{g \geq 0, u} \, c^T f + g^T(w-f) + u^TAf}$$
e.g.
$$\text{max}_{f \geq 0 } \, { \text{min}_{g \geq 0, u} \, (c-g+u^TA)^T f + g^Tw }$$
e.g.
\minimize{g \geq 0, u}{g^Tw }{g - u^TA \geq c}



e.g with $y = (w, g)$, $v = (0 , w)$, $CC = (-c, 0)$\[ AA = \left(
  \begin{array}{cc} A^T&-I_{n^2}\\0&-I_{n^2}\end{array} 
\right)
\]

\minimize{y}{y^Tv }{AA y \leq CC}

We obtain the same solution (14):
$$u = (0, -1, -1, -1, 0, -1, 0)$$
\[ g = \left(
\begin{array}{ccccccc}
1&0&0&0&1&0&1\\
1&1&0&0&1&0&1\\
1&0&1&0&1&0&1\\
1&0&0&1&1&0&1\\
0&0&0&0&0&0&0\\
1&0&0&0&1&1&1\\
0&0&0&0&0&0&0
\end{array} \right)
\]

\lstinputlisting[language=Matlab, frame=single, breaklines=true]{matlabproblem2.m}

\end{itemize}

\Q{3}
$$c = (2, 1, 3, 1, -3)$$
\[ A = \left(
\begin{array}{r|r|r|r|r}
x_1&x_2&x_3&x_4&x_5\\
\hline
1&2&0&4&-3 \\
1&1&0&-3&4 \\
-1&-3&3&0&0 \\
\end{array} 
\right) \]

$\hat A := (A, I_3)$

\subsubsection*{First phase}
\minimize{x, y \geq 0}{e^Ty}{Ax + y = b}
$z = (x = 0, y = b)$ is a BFS
\[\left| 
\begin{array}{c|c|c|c|c|c|c|c|c}
  z & B & A_B & A_B^{-1} & e^Tx & \bar e & j & d & \theta\\
  \hline
   \BDir 0 0 0 0 0 2 2 1 & 6,7,8 & I_3 &
   I_3 &
   5 & \bar e_1 = -1 & 1 & \BDir 1 0 0 0 0 {-1} {-1} 1  & 2 \\
   \hline
   \BDir 2 0 0 0 0 0 0 3 & 1,2,8 & \Ab 1 2 0 1 1 0 {-1} {-3} 1 &
   \Ab {-1}  2 0  1 {-1} 0 2 {-1} 1 &
   3 & \bar e_3 = -3 & 3 & \BDir 0 0 1 0 0 0 0  {-3}  & 1 \\
   \hline
   \BDir 2 0 1 0 0 0 0 0 & 1,2,3 & - &- &
   0 \text{(optimal)} &  - & - & - & - \\
   \hline
\end{array} 
\right| \]

We found a BFS $x = (2, 0, 1, 0, 0)$ to our problem


\subsubsection*{Second phase}
\[\left| 
\begin{array}{c|c|c|c|c|c|c|c|c}
  x & B & A_B & A_B^{-1} & c^Tx & \bar c & j & d & \theta\\
  \hline
   \Dir 2 0 1 0 0 & 1, 3, 5 & \Ab 1 0 {-3} 1 0 4 {-1} 3 0 &
   \frac{1}{21} \Ab {12} 9 0 4 3 7 {-3} 3 0 &
   7 & \bar c_2 =  -\frac{8}{7}, \bar c_4 = -5 & 4 & \Dir {-1} 0 {-\frac{1}{3}} 1 1  & 2 \\
   \hline
   \Dir 0 0 {\frac{1}{3}} 2 2
   & 3,4,5 & \Ab 2 0 {-3} 1 0 4 {-3} 3 0  &
   \frac{1}{33} \Ab {12} 9 0 4 3 7 {-3} 3 0 &
   -3 & \hat c_1 = 5, \hat c_2 = \frac{47}{3} & - & - & -\\
   \hline
\end{array} 
\right| \]

The optimal value is then -3 obtained for $x = \Dir 0 0 {\frac{1}{3}} 2 2$.




\Q{4}

\begin{itemize}
\item 

  Let $x_i$ be the number of times that the manager uses the process $i$. The revenue for the manager is then $38 (4x_1 + x_2 + 3x_2) + 33(3x_1 + x_2 + 4x_3) - 111 x_1 - 11 x_2 - 100 x_3 = w^Tx$ where $w = (140, 60, 146)$


  \maximize{x \geq 0}{ w x}
    {2x_1 + x_2 + 4x_3 \leq 8M \\ &&& 4x_1 + x_2 + 2x_3 \leq 5M}
which can be put to standard form:
    \maximize{x \geq 0}{ x^Tw}
    {2x_1 + x_2 + 4x_3 + x_4 = 8 \\ &&& 4x_1 + x_2 + 2x_3 +x_5 = 5}
\[A = \left(
\begin{array}{r|r|r|r|r}
x_1&x_2&x_3&x_4&x_5\\
\hline
2&1&4&1&0\\
4&1&2&0&1
\end{array} 
\right) \]

$$b = (5, 8)^T$$

We rewrite the program:

    \maximize{x \geq 0}{  x^Tw}{Ax = b}

Simplex tableau

\[
\begin{array}{c|c|c|c|c|c|c|}
x_1 & x_2 & x_3 & x_4 & x_5 &p \\
2 & 1 & 4 & 1 & 0 & 0 & 8 \\
4 & 1 & 2 & 0 & 1 & 0 & 5 \\
-140 &  -60 & -146& 0 & 0 & 1 & 0 \\
\end{array}
\]

\[
\begin{array}{c|c|c|c|c|c|c|}
x_1 & x_2 & x_3 & x_4 & x_5 & p \\
1/2 & 1/4 & 1 & 1/4 &  0 & 0 & 2 \\
3 & 1/2& 0 & -1/2 &  1 & 0 & 1 \\ 
-67& -47/2 & 0 & 73/2 &  0 & 1 & 292    \\
\end{array}
\]


\[
\begin{array}{c|c|c|c|c|c|c|c}
x_1 & x_2 & x_3 & x_4 & x_5 & p \\
0 & 1/6 & 1 & 1/3  &  -1/6 &  0 & 11/6\\   
1 & 1/6 & 0 & -1/6 &  1/3 &   0 & 1/3  \\  
0 & -37/3 & 0 & 76/3 &  67/3 &  1 & 943/3  \\
\end{array}
\]


\[
\begin{array}{c|c|c|c|c|c|c|c}
x_1 & x_2 & x_3 & x_4 & x_5 &p \\
-1 &  0 & 1 & 1/2 & -1/2  & 0 & 3/2    \\
6 & 1 & 0 & -1  & 2 & 0 & 2 \\
74 & 0 & 0 & 13&47  &   1 & 339    
\end{array}
\]

The optimal BFS is then $x = (2, \frac32)$ corresponding to the basis $B = (2,3)$ and the optimal solution $339$

\item
The increase in price (that we call $\Delta$) can change the optimal solution if one of the new reduced costs becomes positive.

The only parameter that changes is $w$, it becomes $w' := w + \Delta (4, 1, 3, 0, 0)$
It is easy to see that $\bar w_4'$ and $\bar w_5'$  will not be affected.


$$\bar w_1 = -74$$


Let's call $\bar w'$ the reduced cost after the price increase.
$$\bar w_1' :=  w_1' - w_B'^T A_B^{-1} A_1 =  \bar w_1 + \Delta (4 - (1, 3)^T A_B^{-1} A_1) = \bar w_1 + \Delta$$


$\bar w_2'$ becomes positive when $\Delta > -\bar w_1 = 74$,


\item
The new constraints are:

$$4x_1 + 3x_2 + 5x_3 \leq 14$$
The optimal solution $x^*$ verifies this constraint
$4 *0 + 3*2 + 5*3/2 = 13.5 < 14$

So the optimal solution doesn't change since we are optimizing over a smaller set.

\item Maximizing $U(r)$ is the same as maximizing $U(r)^2 = r$ when $r$ is positive because $U$ is increasing.

\end{itemize}


\Q{5}

\begin{itemize}

\item
Let's first prove that if we perturbate $a_{i,j}$, there is a basic feasible optimal solution that stays optimal. If that was not the case, it means that some BFS $y$ wasn't optimal and become optimal after the perturbation. Said differently, the reduced costs corresponding to $y$ were all positive and became non positive after each small perturbation. That is not possible because the reduced costs depend continuesly on $A$ and therefore on $a_{i,j}$

Indeed, if $B$ the corresponding basis, $\bar c_j = c_j - c_B^T A_B^{-1} A_j$ seen as a function of $f(a_{i,j})$ is a rational fraction. Therefore if $\bar c_j$ is positive for $a_{i,j}$, there is an openset centered on $a_{i,j}$ where it stays positive.


Let $B$ be the basis corresponding to the BFS that stays optimal after the perturbation, then

$V(A) = c^T A_B^{-1} b$ being linear in A, we have: 

\[ \frac{\partial}{\partial a_{i,j}} V(A) = c^T\frac{\partial}{\partial a_{i,j}} A_B^{-1} b 
= \left\{ \begin{array}{cc}
-c^T A_B^{-1} E_{i,j} A_B^{-1}b & \text{if $j \in B$}\\
0 & \text{if $j \not \in B$}
\end{array}
\right.
\]

Where $E_{ij}$ is the $|A| \times |A|$-matrix with only 0s except on the case $(i,j)$ where there is 1.
In the following we write only the entries of $\nabla V$ that are in the basis, ie $\nabla V := (\frac{\partial V}{\partial a_{ij}})_{i, j \in B}$
\item 

\begin{align}
\nabla V_{i,j} &= -c_B^T A_B^{-1} E_{i,j} A_B^{-1}b \\
&=- c_B^T A_B^{-1} E_i E_j^T A_B^{-1}b \\
&=- (c_B^T A_B^{-1})_i (A_B^{-1}b)_j
\end{align}


$$\nabla V = (c^T A_B^{-1})(A_B^{-1} b)^T$$

\item

\begin{itemize}

\item
If the change in $a_{i,j}$ is for a number in a non-basic column, say $A_N$ , then the original optimal solution is still feasible, the only change is to the reduced cost of $N-$th variable. 
Recompute $\bar c_N = c_N - c_B^T(A_B^{-1})^TA_N = f(a_{i,j})$, the solution doesn't change as long as  $f(a_{ij}) >= 0$, we can solve this linear inequality to get the range.

\item If the change is for a number in a basic column, the solution doesn't change iff $A_B^{-1}$ remains invertible, $A_B^{-1}b$ remains positive and all the reduced costs remain positive. eg one has to solve the following inequalities for $a_{ij}$ ($N$ is the set of columns not in $B$):

$$ A_B^{-1} b \geq 0$$
$$ \bar c = c_N - c_B^T(A_B^{-1})^TA_N \geq 0$$

\end{itemize}


\end{itemize}
\Q{6}

Let $x_1,..,x_n$ is the set of BFS corresponding to the basis $B_1,..B_n$
$\mathbb{P}( \exists i \, x_i \text{ degenrate}) \leq \sum_i \mathbb{P}( x_i \text{ degenrate})$

Let $i \in \{1, ..., n\}$, $x_i$ is degenerate iff there exist a $j$ such that $x_i^{(j)} = \sum_{k=1..n} {A_B^{-1}}_{i,k} b_k = 0$
$({A_B^{-1}}_{i,k})_k$ cannot be all 0 because $A_B^{-1}$ is invertible. Let's suppose for example that ${A_B^{-1}}_{i,1} \neq 0$

But $\mathbb{P}(x_i^{(j)} = 0) = \mathbb{P}( b_1 = -\frac{1}{{A_B^{-1}}_{i,1}} \sum_{k > 0}{A_B^{-1}}_{i,k} b_k) = \mathbb{E} \left[ \mathbb{P}( b_1 = -\frac{1}{{A_B^{-1}}_{i,1}} \sum_{k > 0}{A_B^{-1}}_{i,k} b_k | A, b_2, ... b_k) \right] = 0$ because $b_1$ has a density w.r.t. lebesgue measure.



\end{document}

%%% Local Variables:
%%% mode: latex
%%% TeX-master: t
%%% End:

