\documentclass[12pt]{article}

% packages
\usepackage{geometry}
\usepackage{amsmath}
\usepackage{amsfonts}
\usepackage{enumitem}


% custom commands
\newcommand{\Q}[1]{\subsubsection*{Question #1}}

% parameters
\geometry{hmargin=1cm,vmargin=1cm}
\title{ORF526 - Problem Set 6}
\author{Bachir EL KHADIR }

\begin{document}

\maketitle

\Q{1} 
The pre image of open sets by a continous function is open. 
Let's call $\mathcal O$ the set of open sets.

We know that $\sigma(\mathcal O) = B(R^k)$.


Let's consider $B = \{ A \in B(R^d), f^{-1}(A) \in B(R^k)\}$
we know that $\mathcal O \subseteq B$ because of the definition of continuity, so (using the fact that B is sigma algebra from a question from previous problem set) $B(R^k) = \sigma(\mathcal O) \subset B$, and therefore $B = B(R^k)$.
Therefore $f$ is measurable.

\Q{2}

\begin{enumerate}[label=\alph*.]

\item
$f$ is positive, and integrates to one. 
Indeed, let $I$ denote $\int_R e^{-\frac{x^2}{2}} dx = \int e^{-(\frac x {\sqrt 2})^2} = \sqrt{2} \int_R e^{-u^2} du$.
\begin{align}
I^2 &= 8 \int_0^\infty \int_0^\infty e^{-(x^2 + y^2)} dy\,dx &\text{By Fubuni-Tonelli}\\
&= 8 \int_0^\infty \left( \int_0^\infty e^{-(x^2 + y^2)} \, dy \right) \, dx \\
&= 8 \int_0^\infty \left( \int_0^\infty e^{-x^2(1+s^2)} x\,ds \right) \, dx & s \rightarrow \frac{y}{x}  \text{ being a diffeomorphism} \\
&= 8 \int_0^\infty \left( \int_0^\infty e^{-x^2(1 + s^2)} x \, dx \right) \, ds \\
&= 8 \int_0^\infty \left[ \frac{1}{-2(1+s^2)} e^{-x^2(1+s^2)} \right]_{x=0}^{x=\infty} \, ds \\
&= 8 \left (\tfrac{1}{2} \int_0^\infty \frac{ds}{1+s^2}  \right ) \\
&= 4 \left [ \arctan s \frac{}{} \right ]_0^\infty \\
&= 2 \pi
\end{align}

Since $I \geq 0$,  $\int_R f = \frac{1}{\sqrt{2\pi}} I = 1$

\item 

$\frac{x^n f(x)}{x^2} \rightarrow_{\infty} 0$, so $Z^n$ is integrable for all $n \in \mathbb{N}$


$E[Z^i] = 0$ by symetry of $x \rightarrow x^i f(x)$ for $i = 1, 3$.

$t \rightarrow  x^2$ is a diffeomorphisme from $R^{*+}$ to itself, $dx = \frac {dt} {2\sqrt t} $.
$$E[Z^2] = 2 \int_0^{\infty} x^2 f(x) dx = 2 \int t e^{-\frac {t} 2} dt = - [e^{-\frac t 2}]_0^{\infty} = 1$$

$$E[Z^4] =  \int_R x^4 \frac{e^{-\frac{x^2}{2}}}{\sqrt{2\pi}} dx = -\frac 1 {\sqrt{2\pi}} \left([x^3 e^{-\frac{x^2}{2}}]_{-\infty}^{\infty} + 3 \int_R  x^2 e^{-\frac{x^2}{2}} \right) = 3 E[Z^2] = 3$$


\item 

\begin{itemize}

\item
$E[|Z|] = \frac{2}{\sqrt{2\pi}} \int_0 z e^{-\frac {z^2} {2}} = [-\frac{2}{\sqrt{2\pi}} e^{-\frac {z^2} {2}}]_0^{\infty} = \sqrt{\frac2 \pi}$

\item 
$E[|Z|^2] = Var(Z) = 1$

\item
$E[|Z|^3] = \frac{2}{\sqrt{2\pi}} \int_0 z^2  ze^{-\frac {z^2} {2}}
= -\frac{2}{\sqrt{2\pi}} \int_0 z^2  (e^{-\frac {z^2} {2}})' = [-\frac{2}{\sqrt{2\pi}} z^2 e^{-\frac {z^2} {2}}]_0^{\infty} + \frac{4}{\sqrt{2\pi}} \int_0 z (e^{-\frac {z^2} {2}})
= \frac{4}{\sqrt{2\pi}}$

\item
$E[|Z|^4] = E[Z^4] = 3$
\end{itemize}

\item 

$E[\exp(aZ)] = \int e^{az} e^{-\frac{z^2}{2}} = \int e^{-\frac{(z+a)^2}{2}} e^{\frac{a^2}{2}} = e^{\frac{a^2}{2}}$

\item

$Z = (Z_1, ... Z_n)$, by independence: $\Phi_Z(u) = \prod_i \Phi_{Z_i}(u_i) = (2\pi)^{-\frac n 2}e^{-\frac{||u||^2}{2}}$

By linearity of $E$ and bilinearity of $Cov$ for centered rv:


$$E[X] = \mu + AE[Z] = \mu$$
$$Cov(X) = Cov(X - \mu) = Cov(AZ) = ACov(Z)A^T =  AA^T$$
$$\Phi_X(u) = E[e^{i u^T X}] = E[e^{i u^T \mu}e^{i u^TAZ}] = e^{i u^T \mu} \Phi_Z(u^TA) =  (2\pi)^{-\frac n 2}e^{i u^T \mu -\frac{||u^TA||^2}{2}}$$
\end{enumerate}


\Q{3}


let $X \sim \mathcal N(0,1)$, and $\varepsilon \sim \mathcal B(-1, 1, \frac1 2)$ be two independant rv. And Let $Y = \varepsilon X$

By symmetry of the distribution of $X$:
$$F_Y(y) = P(Y \leq y) = P(\varepsilon X \leq y) = E[ P(\varepsilon X \leq y | \varepsilon) ] = \frac1 2 P(X \leq y) + \frac1 2 P(- X \leq y) = P(X \leq x)$$
so $Y \sim \mathcal N(0, 1)$.

$(X, Y)$ is not  normal because $(1, 1) (X, Y)^T = X+Y$ is not normal because $P(X+Y = 0) = P(\varepsilon = -1) = \frac1 2$.


\Q{4}
\begin{enumerate}[label=\alph*)]
\item 
Let $h: (x, y) \rightarrow (\sqrt{x^2 + y^2}, \phi(x, y))$,
$h^{-1}: (r, \theta) \rightarrow (r \cos(\theta),r \sin(\theta)))$.


$h$ is a diffeomorphisme from from $R^2 \setminus \{(0,0)\}$ to $R^{+*} \times [0, 2\pi[$

$$\det(Dh^{-1}) = r$$

Since $\cos(\theta)^2 + \sin(\theta)^2 =1 $ we have that:
$$f_{X,Y}(r\cos(\theta),r\sin(\theta)) = \frac{1}{2\pi} e^{-\frac 1 2 r^2}$$

For $g$ a continous bounded function, we have that:
\begin{align*}E[g(\sqrt{X^2 + Y^2}, \phi(X, Y))] 
&= \int g(\sqrt{x^2 + y^2}, \phi(x, y)) f_{X,Y}(x,y) dx dy
\\&= \int g(r, \theta) f_{X,Y}(r\cos(\theta),r\sin(\theta)) \frac{dr}{2\pi} d\theta
\\&= \int g(r, \theta) r e^{-\frac1 2 r^2} dr \frac{d\theta}{2\pi}
\\&= E[g(R, \Theta)]
\end{align*}

Where $(R, \Theta)$ has a density $f(r, \theta) = r e^{-\frac1 2 r^2}  \frac{1}{2\pi} 1_{r>0} 1_{\theta \in [0, 2\pi)}$.


\item $X^2 + Y^2 \sim R^2$.
Let $g$ be bounded continuous. Using the change of variable $s = r^2$
$$E[g(X^2 + Y^2)] = E[g(R^2)] = \int_{R^+} g(r^2) r  e^{-\frac1 2 r^2}dr = \int_{R^+} g(s) \frac{1}{2} e^{-\frac 1 2 s} ds$$
So $X^2 + Y^2 \sim \mathcal Exp(\frac 1 2)$


\item 

$h(\sqrt{-2\log U} cos(2\pi V), \sqrt{-2\log U} sin(2\pi V)) = (-2 \log U, 2\pi V) =: (A, B)$

$P(-2 \log U < x) = P(U > e^{-\frac {x} 2}) = (1 - e^{-\frac {x} 2}) 1_{x>0}$

Since $U, V$ are independent, $(A, B)$ has the same distribution as $(R, \Theta)$. 


Using a) we get that $(\sqrt{-2\log U} cos(2\pi V), \sqrt{-2\log U} sin(2\pi V))$ has the same distribution as $X, Y$ because of the following:
For $g$ continuous bounded:
$$E[g(U, V)] = E[goh^{-1}(A, B)] = E[ goh^{-1}(R, \Theta) ] = E[ g(X, Y)]$$


\end{enumerate}



\Q{5}
\begin{enumerate}
\item For next assignment

\item
\[
C 
= 
\left(\begin{array}{cc}C_1& C_2\\ C_3& C_4\end{array}\right)
, C^{-1} = \frac{1}{|C|} \left(\begin{array}{cc}C_4& -C_2\\ -C_3& C_1\end{array}\right) =: \left(\begin{array}{cc}a & b\\ c & d\end{array}\right) 
\]




Let's call $u = y - \mu_x$ 
\begin{align}
N(y) &= \int_R f(x, y) dx 
\\&= \int_R f(x+\mu_x, y) dx 
\\&= \frac{1}{2\pi \sqrt{|C|}} \int  \exp( -\frac 1 2 (x,u)C^{-1}(x,u)^T ) dx
\\&= \frac{1}{2\pi \sqrt{|C|}} \int  \exp(-\frac 1 2 [a x^2 + (b+c)xu + d u^2 ) dx
\\&= \frac{1}{2\pi \sqrt{|C|}} \int  \exp(-\frac {a} 2 [x^2 + (1 + \frac{b+c}{a}) xu] ) dx  \exp(- \frac1 2 d u^2 )
\\&= \frac{1}{2\pi \sqrt{|C|}} \int  \exp\left(-\frac {a} 2 [x + (1 + \frac{b+c}{2(a)})u]^2 \right) dx \, \exp(- \frac1 2 d u^2 + \frac{(b+c)^2}{8a}u^2)
\\&= \frac{1}{2\pi \sqrt{|C|}} \int  \exp\left(-\frac {a} 2 x^2 \right) dx \, \exp( (-\frac1 2 d  + \frac{(b+c)^2}{8a}) u^2)
\\&= \frac{1}{\sqrt{2\pi |C| a}}  \, \exp( (-\frac1 2 d  + \frac{(b+c)^2}{8a}) u^2)
\\&= \frac{1}{\sqrt{2\pi C_4}}  \, \exp( (-\frac {C_1}{2|C|}  + \frac{(C_2+C_3)^2}{8C_4|C|}) u^2)
\\&= \frac{1}{\sqrt{2\pi C_4}}  \, \exp( -\frac1 {2|C|}(C_1  - \frac{(C_2+C_3)^2}{4C_4}) (y-\mu_y)^2)
\end{align}

So
$$f_y(x) = \sqrt{\frac{C_4}{2\pi |C|}} \exp( -\frac 1 2 ((x,y)-\mu)C^{-1}((x,y)-\mu)^T + \frac1 {2|C|}(C_1  - \frac{(C_2+C_3)^2}{4C_4}) (y-\mu_y)^2 ) $$

\end{enumerate}

\Q{6}
Since $X$ is bounded. Let $a$ be such that $|X| < a$. $c[X > x]$ is equal to 1 for $x$ small enough and to 0 for $x$ large enough. The integral is then well defined and equal to $\int_{-a}^0 c[X > x] - c[\Omega] + \int_0^a c[X > x]$
\begin{enumerate}
\item 
Let $\Omega' := \{ X \ge Y \}$. 

We know that $P(\Omega') = 1$ so for every measurable set $A$, $P(A \cup \Omega') = P(A)$.


 $c[X > x] \geq c[Y > x]$ by monotonicity of $c$ because $ \{Y > x\} \cap \Omega' \subseteq \{X > x\} $  and $P(Y > x) = P(\{Y > x\} \cap \Omega')$. And this true because:

$P(Y > x) = P(\{Y > x\} \cap \Omega') + P(\{Y > x\} \cap \Omega'^c)$ and $P(\{Y > x\} \cap \Omega'^c) \le P(\Omega'^c) = 0$



\item 

When $a = 0$ the result is trivial.
When $a \neq 0$:
\begin{align}
\int a X dc &= \int^0 c[X > \frac x a] dx - c[\Omega] + \int_0 c[X > \frac x a]\\
&= \int^0 c[X > u]  - c[\Omega] (a du)+ \int_0 c[X > u] adu & u = \frac x a \\
&= a \int X dc
\end{align}


\item 

\begin{align*}
\int (a + X) dc &= \int^0 c[ X >  x - a]  - c[\Omega] dx + \int_0 c[X > x - a]\\
&= \int^{-a} (c[X > u]  - c[\Omega]) du+ \int_{-a} c[X > u] du \, \, (u = x - a)\\
&= \int^{0} (c[X > u]  - c[\Omega]) du - \int_{-a}^0 (c[X > u]  - c[\Omega]) du + \int_{-a}^0 c[X > u] du + \int_0 c[X > u] du\\
&=  \int X dc + \int_{-a}^0 c[\Omega] du  \\
&=  \int X dc + a c[\Omega]
\end{align*}


\end{enumerate}

\end{document}

%%% Local Variables:
%%% mode: latex
%%% TeX-master: t
%%% End:



