\documentclass[12pt]{article}

% packages
\usepackage{geometry}
\usepackage{amsmath}
\usepackage{amsfonts}

% custom commands
\newcommand{\Q}[1]{\subsubsection*{Question #1}}
\newcommand{\salgebra}{$\sigma$-algebra }

\newcommand{\union}[1]{\underset{#1}{\cup} }
\newcommand{\bigunion}[1]{\underset{#1}{\bigcup} \, }
\newcommand{\inter}[1]{\underset{#1}{\cap} }
\newcommand{\biginter}[1]{\underset{#1}{\bigcap} }

\newcommand{\norm}[1]{||{#1}|| }
\newcommand{\abs}[1]{|{#1}| }

% parameters
\geometry{hmargin=1cm,vmargin=1cm}
\title{ORF526 - Problem Set 1}
\author{Bachir EL KHADIR }

\begin{document}

\maketitle



$\Omega = [0, 1]$, $\mathcal F = \mathcal B([0, 1])$, $\mathbb{P}$ is the restriction of the lebesgue measure on $\Omega$. This is a probability space.


Let's consider the sequence:

$$X_k = k 1_{\{0 < x < \frac 1 k\}}$$


\begin{itemize}
\item $\mathbb{E}[X_k] = 1$
\item $X_k \rightarrow_{k \infty} 0$ a.s., because for all $x \in (0,1)$, $X_k(x) = 0$ for all $k > \frac{1}{x}$
\end{itemize}

\Q{1} 

\begin{itemize}
\item $\sup_k ||X_k||_1 = 1 < \infty$
\item For any $C > 0$, for any $k > C$, $\mathbb{E}[|X_k| 1_{\{X_k > C\}}] = \mathbb{E}[X_k] = 1$. Which means the sequence is not uniformly integrable.
\end{itemize}

\Q{2}
the $(X_k)$ satisfy the conditions

\Q{3}
$\mathbb{E}(\lim \inf X_k) = \mathbb{E}(\lim X_k) = \mathbb{E}(0) = 0 < 1 = \lim_k \mathbb{E}(X_k) = \lim \inf \mathbb{E}(X_k)$


\Q{4}

Let's define 

$\mu_1(A_1, ..., A_m) = \prod_i \mathbb{P}(X_i \in A_i)$

$\mu_2(A_1, ..., A_m) = \mathbb{P}(X_1 \in A_1,...,X_m \in A_m)$

$\mu_1$ is a measure (as the product measure of $\mathbb{P}oX^{-1}$)

$\mu_2$ agrees with $\mu_1$ on sets of the form $(-\infty, x_1] \times ... \times (-\infty, x_m]$


Since $\mu_1$ is $\sigma$-finite (it's a probability measure), by Cathedory extension theorem, $\mu_1 = \mu_2$ on ...


\Q{5}
\begin{enumerate}
\item $i \Rightarrow iii$


Let $\epsilon > 0$,$A_n = \union{m \geq n} \{ \omega , |X_n(\omega) - X(\omega)| > \epsilon\}$
and $A_{\infty} = \inter{n} A_n$
$A_n$ is a decreasing sequence.

If $\omega \in A_\infty$, 
for infinitely many $m \in \mathbb{N}$, 
$|X_n(\omega) - X(\omega)| > \epsilon$. 
Which means that $\omega \in N$. Therefore $\mathbb{P}(A_\infty) \leq \mathbb{P}(N) = 0$


By continuty from above:
$$\mathbb{P}(|X_n - X| > \epsilon) \leq \mathbb{P}(A_n) \rightarrow \mathbb{P}(A_\infty) = 0$$


\item $ii \Rightarrow iii$
By Markov inequality
$\mathbb{P}(|X_n - X| > \epsilon) \leq \frac{E{|X_n - X|}}{\epsilon} \rightarrow 0$


\item $iii \Rightarrow iv$ 

\item blabla
\end{enumerate}


\Q{6}
\begin{enumerate}

\item
Every cdf is right continous and admits$F$ a left limit everywhere. (Let's call it $F(x-)$)

A point of discontinuty if where $F(x-) \neq F(x)$.


Let $A$ be the set of discontinuties of $F$.
\[
f: \left\{ 
\begin{array}{cc}
      A \rightarrow &\mathbb{Q}\\
      x \rightarrow &\text{some arbitrary } r \in (F(x^-), F(x))\\
\end{array}
\right.
\]


This application is an injection. So $A$ is countable.

\item
\end{enumerate}



\end{document}

%%% Local Variables:
%%% mode: latex
%%% TeX-master: t
%%% End:



