\documentclass[12pt]{article}

% packages
\usepackage{geometry}
\usepackage{amsmath}
\usepackage{amsfonts}


\geometry{hmargin=1cm,vmargin=1cm}


% custom commands
\newcommand{\Q}[1]{\subsubsection*{Question #1}}
\newcommand{\salgebra}{$\sigma$-algebra }

\newcommand{\union}[1]{\underset{#1}{\cup} }
\newcommand{\bigunion}[1]{\underset{#1}{\bigcup} \, }
\newcommand{\inter}[1]{\underset{#1}{\cap} }
\newcommand{\biginter}[1]{\underset{#1}{\bigcap} }

\newcommand{\norm}[1]{||{#1}|| }
\newcommand{\abs}[1]{|{#1}| }



% parameters
\title{ORF526 - Problem Set 4}
\author{Bachir EL KHADIR }

\begin{document}

\maketitle

\Q{1}
\begin{enumerate}

\item
Let $\mathbb{F}$ be a field which is either $\mathbb{R}$ or $\mathbb{C}$.  A normed vector space over $\mathbb{F}$ is a pair $(V,\norm{\cdot})$ where $V$ is a vector space over $\mathbb{F}$ and $\norm{\cdot}\colon V\to\mathbb{R}$ is a function such that
\begin{enumerate}
\item $\norm{v}\geq 0$ for all $v\in V$ and $\norm{v}=0$ if and only if $v=0$ in $V$ (\emph{positive definiteness})
\item $\norm{\lambda v} = \abs{\lambda} \norm{v}$ 

for all $v\in V$ and all $\lambda\in\mathbb{F}$
\item $\norm{v+w}\leq\norm{v}+\norm{w}$ for all $v,w\in V$ (the \emph{triangle inequality})
\end{enumerate}


\item Inner product space

\item A metric space $M$ is called complete if every Cauchy sequence of points in $M$ has a limit that is also in $M$ or, alternatively, if every Cauchy sequence in $M$ converges in $M$.

\item A Banach space is a vector space X over the field R of real numbers, or over the field C of complex numbers, which is equipped with a norm and which is complete with respect to that norm.

\item A Hilbert space is a vector space $H$ with an inner product $<f,g>$ such that the norm defined by
 $\norm{f}=\sqrt{<f,f>}$
turns $H$ into a complete metric space. If the metric defined by the norm is not complete, then $H$ is instead known as an inner product space.
\end{enumerate}

\Q{2}
\begin{itemize}

\item
$$(a_1, b_1] \times (a_2, b_2] = (-\infty, b_1] \times (-\infty, b_2]
 \setminus 
\left( (-\infty, b_1] \times (-\infty, a_2] 
 \union{} (-\infty, a_1] \times (-\infty, b_2] \right)$$ 


\begin{align*}
\mu (a_1, b_1] \times (a_2, b_2] 
&= \mu (-\infty, b_1] \times (-\infty, b_2]
 -
\mu \left( (-\infty, b_1] \times (-\infty, a_2] 
 \union{} (-\infty, a_1] \times (-\infty, b_2] \right)\\
&= \mu (-\infty, b_1] \times (-\infty, b_2] 
-\mu  (-\infty, b_1] \times (-\infty, a_2] 
- \mu ((-\infty, a_1] \times (-\infty, b_2]) \\
&+ \mu \left( (-\infty, b_1] \times (-\infty, a_2] 
 \inter{} (-\infty, a_1] \times (-\infty, b_2] \right) \\
&= F(b_1, b_2) - F(b_1, a_2) - F(b_2, a_1) + F(a_1, a_2)
\end{align*}


\item
The following intersection is decreasing:
$$(-\infty, x_1] \times (-\infty, x_2] = \biginter{k \in \mathbb{N}} (-\infty, x^k_1] \times (-\infty, x^k_1]$$
By continuity from above $F(x_k) \rightarrow F(x)$


\item

$$\mathbb{R} = \union{k \in \mathbb{N}} (-\infty, x_1^k] \times (-\infty, x_2^k]$$

The union is increasing, by continuity from below we have the equality.


\item $(-\infty, x_1] \times (-\infty, x_2] \subseteq (-\infty, y_1] \times (-\infty, x_2] $ so $F(x_1, x_2) \leq F(y_1, x_2)$



\item 
\end{itemize}

\Q{3}
Let's write $f$ and $g$ as:
$f = \sum_i a_i 1_{A_i}$, $g = \sum_k b_k 1_{B_k}$
$$\int (f+g) = \sum_i a_i \mu(A_i) + \sum_k \mu(B_k) = \int f + \int g$$

\Q{4}
\begin{itemize}
\item
If $f = \sum a_i 1_{A_i}$ a simple function, then $cf = \sum (c a_i) 1_{A_I}$, $\int c f = \sum c a_i \mu(A_i) = c \sum a_i \mu(A_i) = c \int f$.


If $f_n$ a sequence of increasing simple function converging to $f$, then $(c f_n)$ is an monotonous sequence converging to $c f$, and therefore by monotnous convergence, $\int c f = \lim \int c f_n = c \lim \int f_n = c \int f$.

\item

\item


\end{itemize}

\Q{5}

\begin{itemize}
\item
\begin{itemize}
\item $\mu_f(\emptyset) = \mu f^{-1}(\emptyset) = \mu \emptyset = 0$
\item $\mu_f(B^c) = \mu f^{-1}(B^c) = \mu (f^{-1}B)^c = 1 - \mu(f^{-1}B) = 1 - \mu_f(B)$
\item if $\{B_k | k \in \mathbb{N}\}$ a set of pairwise disjoint sets, so is$\{ f^{-1} B_k | k \in \mathbb{N}\}$ and therefore $$\mu_f(\union{k} B_k) = ...$$
\end{itemize}

\item
If $g$ is simple, eg $g = \sum a_i 1_{A_i}$: $gof = \sum a_i 1_{f^{-1}(A_i)}$

$\int_{\Omega} g o f \rm{d}\mu = \sum_i a_i \mu(f^{-1}A_i) = \sum_i a_i \mu_f(A_i) = \int_E g  \rm{d}\mu_f$

If $g_n \rightarrow g$, $g_nof \rightarrow gof$ so:
$$\int gof = \lim\int g_nof  = \lim \int g_n \rm{d}\mu_f = \int g \rm{d}\mu_f$$
\end{itemize}


\Q{6}


\end{document}

%%% Local Variables:
%%% mode: latex
%%% TeX-master: t
%%% End:

