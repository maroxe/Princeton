\documentclass{article} \usepackage[utf8]{inputenc}
\usepackage[english]{babel} \usepackage{graphicx}
\usepackage{amsfonts} \usepackage{amsmath} \usepackage{amsthm}
\usepackage[a4paper, total={7in, 10in}]{geometry}
\newtheorem{theorem}{Theorem}
\newtheorem{definition}{Definition}
\newtheorem{example}{Example}
\newtheorem{remark}{Remak}
\newtheorem{lemma}{Lemma}

\newenvironment{class}[1]
{\section{Class #1}}
{ ----------------------}
  
\begin{document}

$X = \underbrace{M}_{\text{local martingale}} + \underbrace{A}_{\text{bounded variation process}}$

Ito: $f \in \mathcal C^2, df(X_t) = f'(X_s)dX_s + \frac12 f''(X_s)d<M>_s$
\section{Basic concepts of SPT}
Starting point: semimartingale market models, ie:

\begin{equation}
  dB(t) = r(t) B(t) dt
\end{equation}

\begin{equation}
dX_i(t) = X_i(t) \left(b_i(t)dt + \sum_{\nu} \sigma_{i,\nu} dW_{\mu}(t)\right)
\end{equation}



Here:
\begin{itemize}
\item $B(t)$ is the value of the bank accound if we start from 1 dollar today.
\item $X_i(t)$ stands for the price of one share of stock of company $i$.
\item $r(t)$ is the short rate.
\item $b_i(t)$ rate of return of stock $i$.
\item $\sigma_{i, \nu}(t)$ volatility of stock $i$ with respect to $W_{\nu}$.
\end{itemize}

\begin{theorem}[Solutions]
  (1) and (2) admist solutions (as long as we know the ?)
  $B(t) = e^{\int_0^t r_s ds}$
  $$X_i(t) = X_i(0) \exp({\int_0^t \gamma_i(s) ds + \int \Sigma_{\nu=1}^d \sigma_{i,\nu}(s) dW_{\nu}(s)})$$
  
  where
  $$\gamma_i(t) = b_i(t) - \frac12 a_{ii}(t) = b_i(t) - \frac12 \sum_{\mu=1}^d \sigma_{i\mu}(t)$$
  is called the growth rate.
\end{theorem}
\begin{proof}
  \begin{itemize}
  \item $e^{\int_0^t r(s) ds}$ is a process of bounded variations.
    $( \int_0^t r(s)ds = \int_0^t r(s)^+ds - r(s)^-ds )$ By Ito's
    formula for the semi martingale $\int_0^t r(s)ds$ and $f = \exp$
    $\rm de^{\int_0^t r(s)\rm ds} = e^{\int_0^t r(s)\rm ds} \rm
    d(\int_0^t r(s) \rm ds) = e^{\int_0^t r(s)\rm ds} r(t) dt$.
  \item 
    $$X_i(t) = X_i(0) e^{\int_0^t \gamma_i(s) ds + \int \Sigma_{\nu=1}^d \sigma_{i,\nu}(s) dW_{\nu}(s)}$$
    $$d \, log(X_i(t)) = d(\int_0^t \gamma_i(s) ds + \int \Sigma_{\nu=1}^d \sigma_{i,\nu}(s) dW_{\nu}(s)) = \gamma_i(t) dt + \Sigma_{\nu=1}^d \sigma_{i,\nu}(t) dW_{\nu}(t)$$
    \begin{align*}
      d \, log(X_i(t)) &= \frac{dX_i(t)}{X_i(t)} - \frac12 \frac1{X_i(t)^2} \underbrace{X_i(t)^2 \sum \sigma_{i\mu}^2(t) dt}_{d<X_i>(t)}\\
                       &= \frac{dX_i(t)}{X_i(t)} - \frac12 \sum \sigma_{i\mu}^2(t) dt
    \end{align*}

  \end{itemize}

\end{proof}

\begin{remark}
  [growth rate]
  $$\frac1T \log X_i(t) - \frac1T \int_0^T \gamma_i(t)dt \rightarrow 0$$
  Whenever $\sigma$ does not grow too fast in $T$.
\end{remark}
\begin{proof}
  $$\frac1T \log X_i(t) - \frac1T \int_0^T \gamma_i(t)dt = \frac1T \int_0^T \sum_{\nu} \gamma_{i\nu}(t) dW_\nu(t)$$
\end{proof}

\begin{theorem}[Time change martingale]
  Every stochastic integral $I_t = \sum \int h_{\nu} dW_{\nu}(s)$ can be written as a time change of a brownian motion $\beta$ where
  $$\beta(s) = I_{\tau_s}$$
  $$\tau_s = \inf \{ t : \int_0^t \sum h_{\nu}(s)^2 ds \}$$
  $I_t = \beta(<I>_t)$
\end{theorem}


\begin{class}{Portfolios old theory}
  
  \begin{definition}[Portfolios]
    Fix a filtration $(\mathcal F_t)_{t \ge 0}$ such that $B, X_i, r, b, \sigma$ are adapated to it.
    A portfolio $\Pi(t) = (\Pi_1(t), \ldots, \Pi_n(t))$is a bounded progressively measurable process with respect to $(\mathcal F_t)_t$ such that:
    $$\sum_i \Pi_i(t) = 1 \; \forall t$$
    We $\Pi$ call long-only portfolio if $\pi_i(t) \ge 0 \forall i$ 
  \end{definition}

\begin{definition}[Progessively measurable]
  $\Pi(t)$ measurable with respect to $\cup_{s < t} \mathcal F_s$
\end{definition}
\begin{example}
  \begin{itemize}
  \item Equal weigted portfolio: $\Pi_1(t) = \ldots = \Pi_n(t) = \frac1n$.
  \item Market portfolio: Suppose company $i$ has $N_i(t)$ shares at time $t$
    $\Pi_i(t) = \frac{X_i(t)V_i(t)}{\sum X_j(t)V_j(t)}$
  \end{itemize}
\end{example}

\textbf{Assumption:} All portfolios $\Pi$ are self financing( $\iff$ we immediately re investing all gain from traind).
Mathematically, the portfolio value $V^{(\pi)}(t) = \sum \Pi_i(t) X_i(t)$ satisfies the equation $\frac{dV^{\pi}(t)}{V_i^{pi}(t)} = \sum_{i} \pi_i(t) \frac{dX_i(t)}{X_i(t)}$.
\begin{theorem}
  Has an explicit solution
  $$V^{(\pi)}(t) = V^{(\pi)}(0) \exp( \int_0^t \gamma_{\pi}(u)du + \int_0^t \sum_{\nu} \sigma_{\pi\nu}(u)dW_{\nu}(u))$$
  $\gamma_\pi(t) = \sum_i \pi_i(t) \gamma_i(t) + \gamma_{\pi}^*(t)$
  $\gamma_{\pi}^*(t) = \frac12 (\sum \pi_i(t) a_{ii}(t) - \sum_{i, j} \pi_i(t)\pi_j(t)a_{i,j}(t))$
  $$\sigma_{\pi\nu}(t) = \sum_i \pi_i(t)\sigma_{i\mu}(t)$$
\end{theorem}
\end{class}


\begin{definition}[Portfolio]
  \begin{itemize}
  \item Classical portfolios: $$\zeta(t) = (\underbrace{\zeta_i(t)}_{\text{(\# of share)}})_i$$
  \item Self financing condition:
    portfolio value $V(t) = \zeta.X$ satisfies $dV = \zeta.dX$
  \item in SPT, we wwwwant to think about weights. $\Pi_i(t) = \frac{\zeta_i(t)X_i(t)}{\zeta.X}$
  \item It only make sens to think of $V$ in relative terms:
    $$\frac{dV^{(\pi)}(t)}{V^{\pi}(t)} = \sum_i \pi_i(t) \frac{dX_i(t)}{X_i(t)}$$
  \end{itemize}
\end{definition}
\begin{theorem}
  Has an explicit solution
  $$V^{(\pi)}(t) = V^{(\pi)}(0) \exp( \int_0^t \gamma_{\pi}(u)du + \int_0^t \sum_{\nu} \sigma_{\pi\nu}(u)dW_{\nu}(u))$$
  $$\gamma_\pi(t) = \sum_i \pi_i(t) \gamma_i(t) + \underbrace{\gamma_{\pi}^*(t)}_{\text{excess growth rate}}$$

  $$\gamma_{\pi}^*(t) = \frac12 (\sum \pi_i(t) a_{ii}(t) - \sum_{i, j} \pi_i(t)\pi_j(t)a_{i,j}(t))$$
  $$\sigma_{\pi\nu}(t) = \sum_i \pi_i(t)\sigma_{i\mu}(t)$$
  We can prove that $\frac1T \log(V^{\pi}(t)) - \frac1T \int_0^T\gamma^{\pi}(u)du \rightarrow 0$

\end{theorem}
\begin{remark}[Market portfolios and market weights]
  \textbf{Disclaimer:} Fron now on, think of $X_i(t)$ as the market capitalization of company $i$ ($\#$ shares . price per share).
  
\end{remark}

\subsubsection{The market portfolio}

\textbf{Recall:} the market portfolio has weights $\pi_i(t) = \frac{X_i(t)}{\sum X_j} = \mu_i(t)$.
  For the market portfolio:
  $$\frac1T \int_0^T \gamma^{\mu} du = \frac1T \int_0^T \sum \gamma_i(u)\mu_i(u)du + \frac1T \int_0^T \gamma_{\mu}^*(u)du$$
  If in the original model for $X_i$ the coefficients only depend on the $\mu_i$s:
  $b_i(t) = \bar b_i.\mu, \sigma_{i\nu}(t) = \bar\sigma_{i\nu}.\mu$
  then we are taking the average of a function on $\mu$:
  $$\frac1T \int_0^T f(\mu_1(t), \ldots, \mu_n(t))dt$$

  $\mu \rightarrow \int_0^T f(\mu(t))dt$ is aclled an additive functional.
  To understand market portfolio:
  \begin{itemize}
  \item Need to understand how $\mu$ begaves in the real world.
  \item Select a class of models compatible with that.
  \item Study the assymptotics of the additive functional, which will give us the asymptotic growth of market portfolio.
  \end{itemize}

  Main observation (Fernholz): rank the market weights: $\mu_{(1)} \ge \ldots \ge \mu_{(n)}$
  \begin{itemize}
  \item the curve $\log k \rightarrow \log \mu_{(k)}(t)$ is very stable over time.
  \item shape is close to linear (weights decay poly)
  \item $\implies$ look for models of $(\mu_1(t), \ldots, \mu_n(t))$ so that $(\mu_{(1)}(t), \ldots, \mu_{(n)}(t))$ is stochastically stable. e.g. there exist an initial distribution of $(\mu_{(1)}(0), \ldots, \mu_{(n)}(0)) \overset{d}{=} (\mu_{(1)}(t), \ldots, \mu_{(n)}(t))$
    Such a distribution is a called a stationary / invariant distribution of the process.
  \item Simplest model of this kind: first model of Fernholz. 
  \end{itemize}
  \begin{definition}[First order model]
    Fix parameters $b_1, \ldots, b_n$ and $\sigma_1, \ldots, \sigma_n > 0$.
    Define the evolution of capitalizations:
    $$dX_i(t) = \sum_{k = 1}^n 1_{\{X_i(t) = X_{(k)}(t)\}} b_k + \sum_{k = 1}^n 1_{\{X_i(t) = X_{(k)}(t)\}} \sigma_k dW_i(t)$$
    \textbf{Warning:} Not so easy to make sens of the solution.
    We know:
    \begin{itemize}
    \item There exist a unique \textit{weak} solution:
      \begin{itemize}
      \item given a probability space on which $W_1, \ldots, W_n$ are defined, I can find a larger probability space on which  there are processes $X_1, \ldots X_n$ solving the equation.
      \item No matter how I do it, the law $(X_1, \ldots, X_n)$ will be the same.(Bass Pardoux '87)

      \end{itemize}
    \item There exist a unique \textit{strong} solution if no more than 2 $X_i$'s collide $\iff k\rightarrow \sigma_k^2$ is concave. (Ichiba Karatzav, Misha '15)
    \end{itemize}
  \end{definition}

  \textbf{Goal:} Derive a SDE for the ranked caps $X_{(1)}(t) \le \ldots \le X_{()}(t)$

  \begin{theorem}[Only two processse]
    $X = M^X + A^X, Y = M^Y + A^Y$ semi martingales.
    Then $\max(X, Y)$ and $\min(X, Y)$ are semi martingales.
    \[
      \left\{
        \begin{array}{c}
          d \max(X, Y)_t = 1_{\{\max = X\}}  dX + 1_{ \{ \max = Y\} } dY + \frac12 dL_0^{\max(X,Y) - \min(X,Y)}(t)\\
          d \min(X, Y)_t = 1_{\{\max = X\}}  dX + 1_{ \{ \max = Y\} } dY - \frac12 dL_0^{\max(X,Y) - \min(X,Y)}(t)\\
        \end{array}
      \right.
    \]
    
  \end{theorem}
  \begin{proof}
    Key identity: $max(X, Y) =: X \vee Y = \frac{X+Y}2 + \frac{|X-Y|}2$

    Ito Tanaka:
    \begin{align*}
      d X \vee Y &= \frac{dX + dY}2 + \frac12 \left(sign(X-Y) d(X-Y) + dL^{|X-Y|}_0(t) \right)\\
                   &= \underbrace{\frac12(1 + sign(X-Y))}_{1_{X \ge Y}}dX + \frac12(1 - sign(X-Y))dY + \frac12 dL^{|\max-\min|}_0\\
    \end{align*}

  \end{proof}

  \begin{theorem}[Back to the first order model]
    Consider a first order model with 2 stocks:
    $$dX_1(t) = 1_{\{X_1(t) = X_{(1)}(t)\}} (b_1 dt+ \sigma_1 dW_1(t)) + 1_{\{X_1(t) = X_{(2)}(t)\}} (b_2 dt+ \sigma_2 dW_2(t))$$
    $$dX_2(t) = 1_{\{X_2(t) = X_{(1)}(t)\}} (b_2 dt+ \sigma_2 dW_2(t)) + 1_{\{X_2(t) = X_{(2)}(t)\}} (b_2 dt+ \sigma_2 dW_2(t))$$

    There exist independent standard Brownian Motions $\beta_1, \beta_2$ such that:
    $$dX_{(1)}(t) = b_1 dt + \sigma_1 d\beta_1(t) - \frac12 dL_0^{X_{(1)}-X_{(2)}}$$
    $$dX_{(2)}(t) = b_2 dt + \sigma_2 d\beta_2(t) - \frac12 dL_0^{X_{(1)}-X_{(2)}}$$

    ($X_{(1)} = \min$)
  \end{theorem}
  \begin{lemma}[Levy's caracterization of BM]
    If $M_1, \ldots, M_n$ are continuous local martingales and $<M_i, M_j>(t) = t 1_{i = j}$, then:
    $(M_1, \ldots, M_n)$ is a standard $n$-dimensional BM.
  \end{lemma}
  \begin{proof}
    \begin{align*}
      d X_{(1)}  &= d X_1 \vee X_2 \\
                 &= 1_{X_1 = X_{(1)}} dX_1 + 1_{X_2 = X_{(1)}} dX_2 - \frac12 dL_0^{X_{(2)} - X_{(1)}}\\
                 &= 1_{X_1 = X_{(1)}} (b_1 dt + \sigma_1 dW_1) + 1_{X_1 = X_{(1)} = X_2} (b_1 dt + \sigma_1 dW_2)
      \\&+ 1_{X_2 = X_{(1)}} (b_1 dt + \sigma_2 dW_2) + 1_{X_2 = X_{(1)} = X_1} (b_1 dt + \sigma_1 dW_1)
      \\&- \frac12 dL_0^{X_{(2)} - X_{(1)}}
      \\&= 1_{X_1 = X_{(1)}} (b_1 dt + \sigma_1 dW_1) + 1_{X_2 = X_{(1)}} (b_1 dt + \sigma_1 dW_2) - \frac12 dL_0^{X_{(2)} - X_{(1)}}
                 &(\{ t, 1_{X_1 = X_2}\} \text{has measure 0})
      \\&= b_1 dt + \sigma_1 1_{X_1 = X_{(1)}} dW_1 + \sigma_2 1_{X_2 = X_{(1)}} dW_2
    \end{align*}

    $$dX_{(2)} = b_2 dt + \sigma_1 1_{X_1 = X_{(2)}} dW_1 + \sigma_2 1_{X_2 = X_{(2)}} dW_2 $$
    $$d\beta_{(1)} = 1_{X_{(1)} = X_1} dW_1 + 1_{X_{(1)} = X_2} dW_2$$
    $$d\beta_{(2)} = 1_{X_{(2)} = X_1} dW_1 + 1_{X_{(2)} = X_2} dW_2$$
    
    \textbf{Claim:} $\beta_1, \beta_2$ are independent standard BM. By the lemma. 
    \begin{itemize}
    \item a stochastic integral is continuous and a local martingale
    \item Ito isometry
    \end{itemize}
  \end{proof}

  \begin{theorem}[Banner, Fernholz, Karatzan '05]
    Start with the first order model with $n$ companies:
    $$dX_i(t) = \sum_{k = 1}^n 1_{\{X_i(t) = X_{(k)}(t)\}} b_k + \sum_{k = 1}^n 1_{\{X_i(t) = X_{(k)}(t)\}} \sigma_k dW_i(t)$$
    Then there exist independent standard BM $\beta_1, \ldots, \beta_n$ such that
    $dX_{(k)} = b_k dt + \sigma_k d\beta_k(t) - \frac12 dL_0^{X_{(k+1)} - X_{(k)}}(t) + \frac12 dL_0^{X_{(k)} - X_{(k-1)}}(t)$
  \end{theorem}
  \begin{proof}
    Difficuties
    \begin{itemize}
    \item Why are there no loca times of the form $L^{X_{(l)} - X_{(k)}}$ for $l \ge k+2$?
    \item Why is local time coefficient $\frac12$?
    \end{itemize}
  \end{proof}
\end{document}

  



















