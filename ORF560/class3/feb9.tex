\documentclass{article} \usepackage[utf8]{inputenc}
\usepackage[english]{babel} \usepackage{graphicx}
\usepackage{amsfonts} \usepackage{amsmath} \usepackage{amsthm}
\usepackage[a4paper, total={7in, 10in}]{geometry}
\newtheorem{theorem}{Theorem}
\newtheorem{definition}{Definition}
\newtheorem{example}{Example}
\newtheorem{remark}{Remak}
\newtheorem{lemma}{Lemma}

\newenvironment{class}[1]
{\section{Class #1}}
{ ----------------------}
  
\begin{document}

$X = \underbrace{M}_{\text{local martingale}} + \underbrace{A}_{\text{bounded variation process}}$

Ito: $f \in \mathcal C^2, df(X_t) = f'(X_s)dX_s + \frac12 f''(X_s)d<M>_s$
\section{Basic concepts of SPT}
Starting point: semimartingale market models, ie:

\begin{equation}
  dB(t) = r(t) B(t) dt
\end{equation}

\begin{equation}
dX_i(t) = X_i(t) \left(b_i(t)dt + \sum_{\nu} \sigma_{i,\nu} dW_{\mu}(t)\right)
\end{equation}



Here:
\begin{itemize}
\item $B(t)$ is the value of the bank accound if we start from 1 dollar today.
\item $X_i(t)$ stands for the price of one share of stock of company $i$.
\item $r(t)$ is the short rate.
\item $b_i(t)$ rate of return of stock $i$.
\item $\sigma_{i, \nu}(t)$ volatility of stock $i$ with respect to $W_{\nu}$.
\end{itemize}

\begin{theorem}[Solutions]
  (1) and (2) admist solutions (as long as we know the ?)
  $B(t) = e^{\int_0^t r_s ds}$
  $$X_i(t) = X_i(0) \exp({\int_0^t \gamma_i(s) ds + \int \Sigma_{\nu=1}^d \sigma_{i,\nu}(s) dW_{\nu}(s)})$$
  
  where
  $$\gamma_i(t) = b_i(t) - \frac12 a_{ii}(t) = b_i(t) - \frac12 \sum_{\mu=1}^d \sigma_{i\mu}(t)$$
  is called the growth rate.
\end{theorem}
\begin{proof}
  \begin{itemize}
  \item $e^{\int_0^t r(s) ds}$ is a process of bounded variations.
    $( \int_0^t r(s)ds = \int_0^t r(s)^+ds - r(s)^-ds )$ By Ito's
    formula for the semi martingale $\int_0^t r(s)ds$ and $f = \exp$
    $\rm de^{\int_0^t r(s)\rm ds} = e^{\int_0^t r(s)\rm ds} \rm
    d(\int_0^t r(s) \rm ds) = e^{\int_0^t r(s)\rm ds} r(t) dt$.
  \item 
    $$X_i(t) = X_i(0) e^{\int_0^t \gamma_i(s) ds + \int \Sigma_{\nu=1}^d \sigma_{i,\nu}(s) dW_{\nu}(s)}$$
    $$d \, log(X_i(t)) = d(\int_0^t \gamma_i(s) ds + \int \Sigma_{\nu=1}^d \sigma_{i,\nu}(s) dW_{\nu}(s)) = \gamma_i(t) dt + \Sigma_{\nu=1}^d \sigma_{i,\nu}(t) dW_{\nu}(t)$$
    \begin{align*}
      d \, log(X_i(t)) &= \frac{dX_i(t)}{X_i(t)} - \frac12 \frac1{X_i(t)^2} \underbrace{X_i(t)^2 \sum \sigma_{i\mu}^2(t) dt}_{d<X_i>(t)}\\
                       &= \frac{dX_i(t)}{X_i(t)} - \frac12 \sum \sigma_{i\mu}^2(t) dt
    \end{align*}

  \end{itemize}

\end{proof}

\begin{remark}
  [growth rate]
  $$\frac1T \log X_i(t) - \frac1T \int_0^T \gamma_i(t)dt \rightarrow 0$$
  Whenever $\sigma$ does not grow too fast in $T$.
\end{remark}
\begin{proof}
  $$\frac1T \log X_i(t) - \frac1T \int_0^T \gamma_i(t)dt = \frac1T \int_0^T \sum_{\nu} \gamma_{i\nu}(t) dW_\nu(t)$$
\end{proof}

\begin{theorem}[Time change martingale]
  Every stochastic integral $I_t = \sum \int h_{\nu} dW_{\nu}(s)$ can be written as a time change of a brownian motion $\beta$ where
  $$\beta(s) = I_{\tau_s}$$
  $$\tau_s = \inf \{ t : \int_0^t \sum h_{\nu}(s)^2 ds \}$$
  $I_t = \beta(<I>_t)$
\end{theorem}


\begin{class}{Portfolios old theory}
  
  \begin{definition}[Portfolios]
    Fix a filtration $(\mathcal F_t)_{t \ge 0}$ such that $B, X_i, r, b, \sigma$ are adapated to it.
    A portfolio $\Pi(t) = (\Pi_1(t), \ldots, \Pi_n(t))$is a bounded progressively measurable process with respect to $(\mathcal F_t)_t$ such that:
    $$\sum_i \Pi_i(t) = 1 \; \forall t$$
    We $\Pi$ call long-only portfolio if $\pi_i(t) \ge 0 \forall i$ 
  \end{definition}

\begin{definition}[Progessively measurable]
  $\Pi(t)$ measurable with respect to $\cup_{s < t} \mathcal F_s$
\end{definition}
\begin{example}
  \begin{itemize}
  \item Equal weigted portfolio: $\Pi_1(t) = \ldots = \Pi_n(t) = \frac1n$.
  \item Market portfolio: Suppose company $i$ has $N_i(t)$ shares at time $t$
    $\Pi_i(t) = \frac{X_i(t)V_i(t)}{\sum X_j(t)V_j(t)}$
  \end{itemize}
\end{example}

\textbf{Assumption:} All portfolios $\Pi$ are self financing( $\iff$ we immediately re investing all gain from traind).
Mathematically, the portfolio value $V^{(\pi)}(t) = \sum \Pi_i(t) X_i(t)$ satisfies the equation $\frac{dV^{\pi}(t)}{V_i^{pi}(t)} = \sum_{i} \pi_i(t) \frac{dX_i(t)}{X_i(t)}$.
\begin{theorem}
  Has an explicit solution
  $$V^{(\pi)}(t) = V^{(\pi)}(0) \exp( \int_0^t \gamma_{\pi}(u)du + \int_0^t \sum_{\nu} \sigma_{\pi\nu}(u)dW_{\nu}(u))$$
  $\gamma_\pi(t) = \sum_i \pi_i(t) \gamma_i(t) + \gamma_{\pi}^*(t)$
  $\gamma_{\pi}^*(t) = \frac12 (\sum \pi_i(t) a_{ii}(t) - \sum_{i, j} \pi_i(t)\pi_j(t)a_{i,j}(t))$
  $$\sigma_{\pi\nu}(t) = \sum_i \pi_i(t)\sigma_{i\mu}(t)$$
\end{theorem}
\end{class}


\begin{definition}[Portfolio]
  \begin{itemize}
  \item Classical portfolios: $$\zeta(t) = (\underbrace{\zeta_i(t)}_{\text{(\# of share)}})_i$$
  \item Self financing condition:
    portfolio value $V(t) = \zeta.X$ satisfies $dV = \zeta.dX$
  \item in SPT, we wwwwant to think about weights. $\Pi_i(t) = \frac{\zeta_i(t)X_i(t)}{\zeta.X}$
  \item It only make sens to think of $V$ in relative terms:
    $$\frac{dV^{(\pi)}(t)}{V^{\pi}(t)} = \sum_i \pi_i(t) \frac{dX_i(t)}{X_i(t)}$$
  \end{itemize}
\end{definition}
\begin{theorem}
  Has an explicit solution
  $$V^{(\pi)}(t) = V^{(\pi)}(0) \exp( \int_0^t \gamma_{\pi}(u)du + \int_0^t \sum_{\nu} \sigma_{\pi\nu}(u)dW_{\nu}(u))$$
  $$\gamma_\pi(t) = \sum_i \pi_i(t) \gamma_i(t) + \underbrace{\gamma_{\pi}^*(t)}_{\text{excess growth rate}}$$

  $$\gamma_{\pi}^*(t) = \frac12 (\sum \pi_i(t) a_{ii}(t) - \sum_{i, j} \pi_i(t)\pi_j(t)a_{i,j}(t))$$
  $$\sigma_{\pi\nu}(t) = \sum_i \pi_i(t)\sigma_{i\mu}(t)$$
  We can prove that $\frac1T \log(V^{\pi}(t)) - \frac1T \int_0^T\gamma^{\pi}(u)du \rightarrow 0$

\end{theorem}
\begin{remark}[Market portfolios and market weights]
  \textbf{Disclaimer:} Fron now on, think of $X_i(t)$ as the market capitalization of company $i$ ($\#$ shares . price per share).
  
\end{remark}

\subsubsection{The market portfolio}

\textbf{Recall:} the market portfolio has weights $\pi_i(t) = \frac{X_i(t)}{\sum X_j} = \mu_i(t)$.
  For the market portfolio:
  $$\frac1T \int_0^T \gamma^{\mu} du = \frac1T \int_0^T \sum \gamma_i(u)\mu_i(u)du + \frac1T \int_0^T \gamma_{\mu}^*(u)du$$
  If in the original model for $X_i$ the coefficients only depend on the $\mu_i$s:
  $b_i(t) = \bar b_i.\mu, \sigma_{i\nu}(t) = \bar\sigma_{i\nu}.\mu$
  then we are taking the average of a function on $\mu$:
  $$\frac1T \int_0^T f(\mu_1(t), \ldots, \mu_n(t))dt$$

  $\mu \rightarrow \int_0^T f(\mu(t))dt$ is aclled an additive functional.
  To understand market portfolio:
  \begin{itemize}
  \item Need to understand how $\mu$ begaves in the real world.
  \item Select a class of models compatible with that.
  \item Study the assymptotics of the additive functional, which will give us the asymptotic growth of market portfolio.
  \end{itemize}

  Main observation (Fernholz): rank the market weights: $\mu_{(1)} \ge \ldots \ge \mu_{(n)}$
  \begin{itemize}
  \item the curve $\log k \rightarrow \log \mu_{(k)}(t)$ is very stable over time.
  \item shape is close to linear (weights decay poly)
  \item $\implies$ look for models of $(\mu_1(t), \ldots, \mu_n(t))$ so that $(\mu_{(1)}(t), \ldots, \mu_{(n)}(t))$ is stochastically stable. e.g. there exist an initial distribution of $(\mu_{(1)}(0), \ldots, \mu_{(n)}(0)) \overset{d}{=} (\mu_{(1)}(t), \ldots, \mu_{(n)}(t))$
    Such a distribution is a called a stationary / invariant distribution of the process.
  \item Simplest model of this kind: first model of Fernholz. 
  \end{itemize}
  \begin{definition}[First order model]
    Fix parameters $b_1, \ldots, b_n$ and $\sigma_1, \ldots, \sigma_n > 0$.
    Define the evolution of capitalizations:
    $$dX_i(t) = \sum_{k = 1}^n 1_{\{X_i(t) = X_{(k)}(t)\}} b_k + \sum_{k = 1}^n 1_{\{X_i(t) = X_{(k)}(t)\}} \sigma_k dW_i(t)$$
    \textbf{Warning:} Not so easy to make sens of the solution.
    We know:
    \begin{itemize}
    \item There exist a unique \textit{weak} solution:
      \begin{itemize}
      \item given a probability space on which $W_1, \ldots, W_n$ are defined, I can find a larger probability space on which  there are processes $X_1, \ldots X_n$ solving the equation.
      \item No matter how I do it, the law $(X_1, \ldots, X_n)$ will be the same.(Bass Pardoux '87)

      \end{itemize}
    \item There exist a unique \textit{strong} solution if no more than 2 $X_i$'s collide $\iff k\rightarrow \sigma_k^2$ is concave. (Ichiba Karatzav, Misha '15)
    \end{itemize}
  \end{definition}

  \textbf{Goal:} Derive a SDE for the ranked caps $X_{(1)}(t) \le \ldots \le X_{()}(t)$

  \begin{theorem}[Only two processse]
    $X = M^X + A^X, Y = M^Y + A^Y$ semi martingales.
    Then $\max(X, Y)$ and $\min(X, Y)$ are semi martingales.
    \[
      \left\{
        \begin{array}{c}
          d \max(X, Y)_t = 1_{\{\max = X\}}  dX + 1_{ \{ \max = Y\} } dY + \frac12 dL_0^{\max(X,Y) - \min(X,Y)}(t)\\
          d \min(X, Y)_t = 1_{\{\max = X\}}  dX + 1_{ \{ \max = Y\} } dY - \frac12 dL_0^{\max(X,Y) - \min(X,Y)}(t)\\
        \end{array}
      \right.
    \]
    
  \end{theorem}
  \begin{proof}
    Key identity: $max(X, Y) =: X \vee Y = \frac{X+Y}2 + \frac{|X-Y|}2$

    Ito Tanaka:
    \begin{align*}
      d X \vee Y &= \frac{dX + dY}2 + \frac12 \left(sign(X-Y) d(X-Y) + dL^{|X-Y|}_0(t) \right)\\
                   &= \underbrace{\frac12(1 + sign(X-Y))}_{1_{X \ge Y}}dX + \frac12(1 - sign(X-Y))dY + \frac12 dL^{|\max-\min|}_0\\
    \end{align*}

  \end{proof}

  \begin{theorem}[Back to the first order model]
    Consider a first order model with 2 stocks:
    $$dX_1(t) = 1_{\{X_1(t) = X_{(1)}(t)\}} (b_1 dt+ \sigma_1 dW_1(t)) + 1_{\{X_1(t) = X_{(2)}(t)\}} (b_2 dt+ \sigma_2 dW_2(t))$$
    $$dX_2(t) = 1_{\{X_2(t) = X_{(1)}(t)\}} (b_2 dt+ \sigma_2 dW_2(t)) + 1_{\{X_2(t) = X_{(2)}(t)\}} (b_2 dt+ \sigma_2 dW_2(t))$$

    There exist independent standard Brownian Motions $\beta_1, \beta_2$ such that:
    $$dX_{(1)}(t) = b_1 dt + \sigma_1 d\beta_1(t) - \frac12 dL_0^{X_{(1)}-X_{(2)}}$$
    $$dX_{(2)}(t) = b_2 dt + \sigma_2 d\beta_2(t) - \frac12 dL_0^{X_{(1)}-X_{(2)}}$$

    ($X_{(1)} = \min$)
  \end{theorem}
  \begin{lemma}[Levy's characterization of BM]
    If $M_1, \ldots, M_n$ are continuous local martingales and $<M_i, M_j>(t) = t 1_{i = j}$, then:
    $(M_1, \ldots, M_n)$ is a standard $n$-dimensional BM.
  \end{lemma}
  \begin{proof}
    \begin{align*}
      d X_{(1)}  &= d X_1 \vee X_2 \\
                 &= 1_{X_1 = X_{(1)}} dX_1 + 1_{X_2 = X_{(1)}} dX_2 - \frac12 dL_0^{X_{(2)} - X_{(1)}}\\
                 &= 1_{X_1 = X_{(1)}} (b_1 dt + \sigma_1 dW_1) + 1_{X_1 = X_{(1)} = X_2} (b_1 dt + \sigma_1 dW_2)
      \\&+ 1_{X_2 = X_{(1)}} (b_1 dt + \sigma_2 dW_2) + 1_{X_2 = X_{(1)} = X_1} (b_1 dt + \sigma_1 dW_1)
      \\&- \frac12 dL_0^{X_{(2)} - X_{(1)}}
      \\&= 1_{X_1 = X_{(1)}} (b_1 dt + \sigma_1 dW_1) + 1_{X_2 = X_{(1)}} (b_1 dt + \sigma_1 dW_2) - \frac12 dL_0^{X_{(2)} - X_{(1)}}
                 &(\{ t, 1_{X_1 = X_2}\} \text{has measure 0})
      \\&= b_1 dt + \sigma_1 1_{X_1 = X_{(1)}} dW_1 + \sigma_2 1_{X_2 = X_{(1)}} dW_2
    \end{align*}

    $$dX_{(2)} = b_2 dt + \sigma_1 1_{X_1 = X_{(2)}} dW_1 + \sigma_2 1_{X_2 = X_{(2)}} dW_2 $$
    $$d\beta_{(1)} = 1_{X_{(1)} = X_1} dW_1 + 1_{X_{(1)} = X_2} dW_2$$
    $$d\beta_{(2)} = 1_{X_{(2)} = X_1} dW_1 + 1_{X_{(2)} = X_2} dW_2$$
    
    \textbf{Claim:} $\beta_1, \beta_2$ are independent standard BM. By the lemma. 
    \begin{itemize}
    \item a stochastic integral is continuous and a local martingale
    \item Ito isometry
    \end{itemize}
  \end{proof}

  \begin{theorem}[Banner, Fernholz, Karatzan '05]
    Start with the first order model with $n$ companies:
    $$dX_i(t) = \sum_{k = 1}^n 1_{\{X_i(t) = X_{(k)}(t)\}} b_k dt + \sum_{k = 1}^n 1_{\{X_i(t) = X_{(k)}(t)\}} \sigma_k dW_i(t)$$
    Then there exist independent standard BM $\beta_1, \ldots, \beta_n$ such that
    $dX_{(k)} = b_k dt + \sigma_k d\beta_k(t) - \frac12 dL_0^{X_{(k+1)} - X_{(k)}}(t) + \frac12 dL_0^{X_{(k)} - X_{(k-1)}}(t)$
  \end{theorem}
  \begin{proof}
    Difficuties
    \begin{itemize}
    \item Why are there no loca times of the form $L^{X_{(l)} - X_{(k)}}$ for $l \ge k+2$?
    \item Why is local time coefficient $\frac12$?
    \end{itemize}

    From induction Hypothesis:
    $$dX_{(k)}(t) = \sum_i^n 1_{X(k) = X_i(t)} \frac{1}{N_k(t)} dX_i(t) + \sum_l^{k-1}\frac{1}{N_k(t)} dL_0^{X_{(k)} - X_{(l)}}(t)
    - \sum_{l=k+1}^n \frac{1}{N_k(t)} dL_0^{X_{(l)} - X_{(k)}}(t)
    $$

    Idea $X_{(1)} = \min(X_1, \ldots, X_n) = \min(X_1, \min(X_2, \ldots, X_n))$

    Tasks at this point
    \begin{itemize}
    \item $N_k(t) =1$ for all k and lebesgue a.e.t with probability 1.
    \item $L_0^{X_{(k)} - X{(k)}} = 0$ for all $|l - k| \ge 1$ with probability 1.
    \item $N_k(t) = 2$ under $dL_0^{X_{(k+1) - X_{(k)}}}$ with probability 1, ie
      $\mathbb P(\int_0^\infty 1_{\{N_k(t) \ne 2 \}} dL_0^{X_{(k+1) - X_{(k)}}} = 0)$
    \end{itemize}
  \end{proof}

  \subsubsection{Skrodhod problems and reflected Brownian motions}

  \begin{definition}[Skorokhod problem in 1D]
    Given a continuous path $\phi: [0, \infty) \rightarrow \mathcal{R}$ with $\phi(0) > 0$, want to find a non-decreasing path
    $\eta: [0, \infty)\rightarrow \mathcal{R}^+$ s.t
    \begin{itemize}
    \item $\psi(t) = \phi(t) + \eta(t) \ge 0$ for all $t \ge 0$.
    \item $\int_0^{\infty} 1_{\psi(t) \ne 0} d\eta(t)= 0$
    \end{itemize}

    \begin{theorem}[Skorokhold]
      There exists a unique solution of the skorokhod problem for any continuous $\phi$ s.t $\phi(0) > 0$.
    \end{theorem}
    \begin{proof}
      \begin{itemize}
      \item $\eta(t) = \sup_{0 \le s \le t} \phi^-(s)$
        $\int_0^{\infty} 1_{\psi(t) \ne 0} d\eta(t) = 0$?

        Note that a point of increase t of $\eta$($\iff$ the support of the corresponding measure $d\eta$) have the property
        $\phi(t)^- = \sup_{s \le t} \phi(s)^- = \eta(t)$.
        
        We need the show that $\psi(t) = 0$ for such a point $t$.
        
      \item Uniquenss: $(\eta, \psi), (\hat \eta, \hat \psi)$ two solutions.
        
        $\hat \psi - \psi = \hat \eta - \hat \eta$ BV process.

        Ito: $\frac12 (\hat \psi - \psi)^2 = \int_0^t (\hat \psi - \psi)d(\hat \eta - \eta) =
        - \int_0^t \hat \psi d\eta - \int_0^t  \psi d \hat \eta$
      \end{itemize}
    \end{proof}
  \end{definition}
  \begin{definition}[Reflected Brownian motion in 1D]
    \begin{align*}
      \Phi & : C([0, \infty), \mathbb R) &\longrightarrow &C([0, \infty), \mathbb R)\\
           & : \phi &\longrightarrow & \psi 
    \end{align*}
    A reflected Brownian motion with drift $\mu$ and diffusion coefficient $\sigma$ is the process $\Phi(\mu t + \sigma B(t))$
  \end{definition}
  \begin{remark}

    Consider a first order model with 2 companies:
    $dX_i = 1_{X_i = X_(1)} b_1 dt + 1_{X_i = X_(2)} b_2 dt + 1_{X_i = X_(1)} \sigma_1 dW_1 + 1_{X_i = X_(2)} \sigma_2 dW_2$

    Claim: $|X_1 - X_2| = X_{(2)} - X_{(1)}$ is a RBM with drift $b_2 - b_1$ and drift coefficient $\sqrt{\sigma_1^2 + \sigma_2}$.
    $\underbrace{X_{(2)}(t) - X_{(1)}(t)}_{\psi(t)} = \underbrace{X_{(2)}(0) - X_{(1)}(0) + (b_2 - b_1)t + \sigma_2 \beta_2(t) - \sigma_1 \beta_1(t)}_{\phi(t)} + L_0^{X_{(2)} - X_{(1)}}(t)$
    
  \end{remark}
  \begin{definition}[Skorokhod problem in $R_+^m$ (SP)]
    Consider a continuous path $\phi: \mathbb R_+ \rightarrow \mathbb R^m$ st $\phi(0) > 0$, and a matrix $R \in \mathbb R^{m \times n}$
    . We want to find a continuous path $\eta \mathbb R_+ \rightarrow \mathbb R^m$ such that:
    \begin{itemize}
    \item all componoents of $\eta$ are non-decreasing.
    \item $\psi(t) = \phi(t) + R \eta(t)$
    \item  $\int_0^{\infty} 1_{t, \psi_k(t) \ne 0} d\eta_k(t) = 0$
    \end{itemize}
  \end{definition}
  \begin{theorem}[Existence and Uniqueness of the Solution]
    $R = I - Q$ and $Q$ has non negative entries, zero diagonal entries and spectral radius $<1$, then the skorokhod problem has unique solution. 
  \end{theorem}
  \begin{proof}
    Define $C_0 = \{ \eta: [0, \infty) \rightarrow (R^+)^m \text{ non decreasing componenents starting at } 0\}$
    $\Pi: C_0 \rightarrow C_0; \eta \rightarrow \sup_{0 \le s \le t} (\phi(s) - Q\eta(s))$
    \textbf{Claim:}
    $$\eta \text{ Solution to SP } \iff \Pi(\eta) = \eta$$

    $\implies$ existence and uniqueness Banach fixed point theorem. (starting from 0)
  \end{proof}
  \begin{proof}[Fixed point]
    \begin{itemize}
    \item[$\Rightarrow$] If $\Pi(\eta) = eta$, then $\eta$ solves SP.
      \begin{itemize}
      \item $\eta(t) = \sup_{s \le t}(\phi(s) - Q\eta(s))$, $\eta$ is  non-decresaing, $\eta(0) = 0$. 
      \item Non negativity:
        $$\psi(t) = \phi(t) + \eta(t) - Q \eta(t) \ge \eta(t) + (\phi(t) - Q\eta(t)) \ge 0$$
      \item $\{ \text{points of increase of } \eta_i \} \subseteq \{ t; \psi_i(t) = 0\}$

        Let $t^*$ be a point of increast of $\eta_i$. so we have:
        $$\sup_{0 \le s \le t^*} (\phi_i(s) - \sum_{j=1}^m Q_{ij} \eta_j(s))^- = (\phi_i(t^*) - \sum_{j=1}^m Q_{ij} \eta_j(t^*))^- $$

        
        \begin{align*}
          0 \le \psi(t^*) &= \phi(t^*) + (I-Q)\eta(t^*) \\
                    &= \underbrace{(\phi_i(t^*) - Q \eta(t^*))^+}_0 + (\phi_i(t^*) - Q \eta(t^*))^- + \eta(t^*)\\
                    &\le 0
        \end{align*}

      \end{itemize}
    \item[$\Leftarrow$]
      Steps:
      \begin{itemize}
      \item $v$ a fixed point , $\eta$ a solution, then $v \le \eta$
      \item Suppose $\eta_i(t) > v_i(t)$, for i, t, then $\eta_i(t^*) > v_i(t^*)$ for a $t^*$ which is a point of increase of $\eta_i$
      \item $0 = \psi(t^*) = \phi(t^*) - (Q\eta)_i(t^*) + \eta_i(t^*) =  \underbrace{( \phi(t^*) - (Q\eta)_i(t^*) )^+}_0 - \underbrace{(\phi(t^*) - (Q\eta)_i(t^*))^-}_{\le v_i(t^*)} + \eta_i(t^*)$
        so $v_i(t^*) \ge  + \eta_i(t^*)$.
      \end{itemize}
    \end{itemize}
  \end{proof}
  \begin{theorem}[Banach Fixed Point Theorem]
    If $(S, d)$ a complete metric space, and $\Pi: S \rightarrow S$ $\alpha$-contractant, $\alpha < 1$.
    Then there exists a unique fixed point of $\Pi$
  \end{theorem}
  \begin{proof}[Application]
    Define
    $C_0^T = \{ \eta: [0, T] \rightarrow (R^+)^m \text{ non decreasing
      componenents starting at } 0\}$ Consider the space $C_0^T$ with
    the distance:  $d(\eta, \tilde \eta) = sup_{[0, T]} ||\eta(t) - \tilde
    \eta(t)||_{\infty}$
    
    $\Pi: C_0^T \rightarrow C_0^T$ is not a contraction unfortunately.
    
    Instead:
    
    for a matrix $Q$ with non negative entries and $spec(Q) < 1$ then there exist a diagonal $D$ with positive entries s.t. $DQD^{-1}$ has column sums of the new matrix < 1.
    
    If we have a solution to SP with $DQD^{-1}$, then the image under $D^{-1}$ of the new reflected path is the desired $\Psi$.
    $\Pi: C_0^T \rightarrow C_0^T$ is a contraction for $DQD^{-1} = \tilde Q$.
  \end{proof}

  Goal: Banner Fernholz Karatzad theorem (*):

  $$dX_{(k)} = b_k dt + \sigma_k d\beta_k + \frac12 dL_0^{X_{(k)} - X_{(k-1)}} - \frac12 dL_0^{X_{(k+1)} - X_{(k)}}$$

  \[
    \left(
      \begin{array}{c}
        X_{(2)}(t) - X_{(1)})(t)\\
        \cdots\\
        X_{(n)}(t) - X_{(n-1)})(t)\\
      \end{array}
    \right) = 
    \left(
      \begin{array}{c}
        X_{(2)}(t) - X_{(1)})(0)\\
        \cdots\\
        X_{(n)}(t) - X_{(n-1)})(0)\\
      \end{array}
    \right)
    +
    \left(
    \begin{array}{c}
      b_2 - b_1
      \cdots\\
      b_n - b_{n-1}
    \end{array}
  \right)t
  +
  \left(
    \begin{array}{c}
      \beta_{2}(t) - \beta_{1})(t)\\
      \cdots\\
      \beta_{n}(t) - \beta_{n-1}))(t)\\
    \end{array}
  \right)
  +
  (I - Q)
  ...
\]

Recall that to get (*) we need
\begin{itemize}
\item $\int_0^{\infty} 1_{X_{(i)} = X_{(i+1)} }dt = 0$ as
\item $L_0^{X_{(l)} - X_{(k)}} = 0$ if $|k - l| \ge 2$
\item ($|\{ i, X_{(k)}(t) = X_{(l)} \}| = 2$ $dL^{X_{(k+1)} - X_{(k)}}$ae ) $\mathbb P$-as
\end{itemize}
\begin{theorem}[Reiman, Williams PTRF '88]
  Take a RBM $Z$ on $\mathbb R_+^{n-1}$ with reflection matrix $R = I - Q$, then $\int_0^{\infty} 1_{Z_l(t) = 0} dL_0^{Z_k}(t) = 0 \; \mathbb P$ -as for $l \ne k$
\end{theorem}
\begin{proof}
  By definition $Z(t) = Z(0) + \mu t + W(t) + R(L_0^{(Z_i(t))})_i$ where $\mu \in R^{n-1}$, $W$ is a BM with some covariance matrix( In this case $A = 2R$).

  By induction on $|J| \ge 2$, $J \subseteq \{ 1, \ldots, m \}$, $\int_0^{\infty} 1_{Z_l(t) = 0, l \in J} dL_0^{Z_k}(t) = 0$
  
\end{proof}
\begin{proof}[Proof of BFK]
  \begin{itemize}
  \item to $b_k dt + \sigma_k d\beta_k(t)$ in (*) Try to use Levy's characerization of BM.
    $\forall k \; Leb(\{ t, X_{(k)} = X_{(k+1)}(t)  \}) = 0$ $\iff \forall i,j \; Leb(X_{i}(t) = X_{j}(t)) = 0$
    Use Girsanov, wlog $b_1 = \ldots = b_n$
    $$dX_i(t) = \sum_{k = 1}^n 1_{\{X_i(t) = X_{(k)}(t)\}} \sigma_k dW_i(t)$$
    $$d(X_i - X_j)(t) = \underbrace{\sum_{k = 1}^n 1_{\{X_i(t) = X_{(k)}(t)\}} dW_i(t) - \sum_k 1_{\{X_j(t) = X_{(k)}(t)\}} \sigma_k dW_j(t)}_{\text{martingale}}$$
    $$<X_i - X_j>(t) = \int_0^t (\sum_{k = 1}^n 1_{\{X_i(t) = X_{(k)}(t)\}})^2 + \int (\sum_k 1_{\{X_j(t) = X_{(k)}(t)\}} \sigma_k)^2 $$

    All $\sigma_k > 0$, so $<>_t \ge ct$ for some $c > 0$, By Montonous class argument $\int_A d<>_t \ge c \; Leb(A)$

    \begin{theorem}
      If that property holds for a continuous martingale $M$ then one can find $\theta$ st:
      
      $$\forall t > 0, \forall I \text{ open, finite, interval} P(M(t)
      \in I) \le |I|^{\theta}$$
      
      which would imply that $P(M(t) = m) = 0$
    \end{theorem}
    Counter example: Statement not true with $\Theta |I|$

    Fabes, Kenig, Duke MK
    
    $$dY_t = \sigma(t, Y_t)dB_t$$
    satisfies: law of $Y_t$ zero abs continuous part.
    
  \item Step 2: Kang Williams AAP' 07

  \end{itemize}
\end{proof}
\end{document}

  



















