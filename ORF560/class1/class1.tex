%%% Local Variables:
%%% mode: latex
%%% TeX-master: "chapter1.tex"
%%% End:

\documentclass{article}
\usepackage[utf8]{inputenc}
\usepackage[english]{babel}
\usepackage{graphicx}
\usepackage{amsfonts}
\usepackage{amsmath}
\usepackage{amsthm}
\usepackage[a4paper, total={7in, 10in}]{geometry}
\newtheorem{theorem}{Theorem}
\newtheorem{definition}{Definition}


\begin{document}
\section{measure theory}
  
\begin{definition}[(Continuous) Semimartingale]
  $X_t, t \ge 0$ is a stochastic process with continuous paths which admits the decomposition
  $$X_t = M_t + A_t$$
  \begin{itemize}
  \item $M_t$ is a local martingale
  \item $A_t$ is a processof bounded variation.
  \end{itemize}
\end{definition}
\begin{definition}[Process of bounded variation(BV)]
  $A_t$ has BV if it is a difference of two non-decreasing prices.
  $A_t = A_t^{up} - A_t^{down}$ Where $A_t^{up/down}$ are non-decreasing.
  Alternatively
  $$A(\omega) = \int_0^t \mu(ds, w) = \mu([0, t], \omega)$$
  where $\mu$ is a random signed non-atomic measure on $R$.
\end{definition}
\begin{definition}[Local Martingale]
  $M_t$ is local martingale iff there exists a sequence $\tau_n \rightarrow \infty$ of stopping times s.t $M_{t \wedge \tau_n}$ is a martingale for every $n$.
\end{definition}
\subsection{Question}
\begin{itemize}
\item What happens if we take $f \in C^2(R)$. Is it true that $f(X_t)$
  is still a semi martingale?
\item What can we say about the maximum of two semi martingalse.
\end{itemize}

\begin{theorem}[Ito formula for semi martingals]
  If $X$ is a semimartingale and and $f \in C^2$, then $f(X_t)$ is a semimartingale and
  $$f(X_t) = f(X_0) + \int_0^t f'(X_s) dM_s + \int_0^t f'(X_s)dA_s + \int_0^t \frac{f''(X_s)}2 d<M>_s$$
  Where
  \begin{itemize}
  \item $X_t = M_t + A_t$
  \item $<M>_t$ is the quandratic variation of $M$. It is the only BV process s.t $M_t^2 - <M>_t$ is a local martingale.
  \item Existence: $<M>_t = \lim^{\mathbb{P}}_{\text{mesh} \rightarrow 0} \sum_{i=1..k+1} (M_{t_i} - M_{t_{i-1}})^2$
  \end{itemize}
\end{theorem}
Example: $X_t$ brownian motion.
\begin{itemize}
\item $$X_t^2 = X_0^2 + \int_0^t 2X_s dX_s + \int_0^t 1 d<X>_s$$
\item $$d<X>_s = ds$$
\end{itemize}

\begin{proof}
  $ $\newline
\begin{itemize}
\item Localization:
  Introduce the stopping times $\tau_n = \inf \{ t: |X_t| \ge n\}$. Note that $\tau_n$ is non-decreasing $\rightarrow \infty$.
  Suffice th show that $f(X_{t \wedge \tau_n}) = f(X_0) + \int_0^{t \wedge \tau_n} f'(X_s)ds + \int_0^{t \wedge \tau_n} f'(X_s)dA_s + \int_0^{t \wedge \tau_n} \frac{f''(X_s)}2 d<M>_s$ (Because it converge in probability to the original formula)
\item On $[-n, n]$, we can approximate $f$ by polynomials uniformly ($p_k^{(j} \rightarrow f^{(j)}, j = 0, 1, 2$). (Stones-Weierestrass theorem applied to $f''$). 
\item It is enough to show Ito's formula for polynomials. By linearity it is enough to show it for $f(x) = x^m$
  $$X_{t \wedge \tau_k} = X_0^m + \int_0^{t \wedge \tau_n} mX_s^{m-1} dM_s + \int_0^{t \wedge \tau_n} m X_s^{m-1} dA_s + \int_0^{t \wedge \tau_n} \frac{m(m-1)}2 X_s^{m-2} d<M>_s$$
\item Induction over $m$, $X_t^m = X_t X_t^{m-1} \overset{\text{Ito applied to product}}{=} ..$
\end{itemize}
\end{proof}

\begin{theorem}[Ito applied to product]
  if $X$ and $Y$ are continuous semi martingales, then $XY$ is also a consitnuous semi-martingale and we have the decomposition:
  $$X_t Y_t = X_0 Y_0 + \int_0^t X_s dY_s + \int_0^t Y_s dX_sds + <X, Y>_t$$ where:
  \begin{itemize}
  \item $<X, Y>_t = \lim \sum \Delta X_k \Delta Y_k$, is the BV part of $X_t Y_t$, 
  \end{itemize}
\end{theorem}


\begin{itemize}
\item Polarization $$X_tY_t = \frac{(X_t + Y_t)^2 - (X_t - Y_t)^2}4$$
  So it is enough to show that the square of semi-martingale is also a
  semi martingale.
\item If $Z_t$ a semi martingale, let's show that $$Z_t^2 = Z_0^2 + \int_0^t2Z_s dM_s^Z + \int_0^t 2 Z_s dA_s^Z + \int_0^t 1 d<M^Z>_s$$:
  \begin{align*}
    <M^Z>_t &= \lim \sum (M^Z_{t_i} - M^Z_{t_{i-1}})^2
    \\&= \lim \sum (Z_{t_i} - Z_{t_{i-1}})^2 & \text{(Because $A^Z$ has BV)}
    \\&= \lim Z_t^2 - Z_0^2 -2 \sum Z_{t_{i-1}}(Z_{t_i} - Z_{t_{i-1}})
    \\&= Z_t^2 - Z_0^2 -2 \int Z_s dZ_s
  \end{align*}
\end{itemize}


\subsection{Next question}
What if $f$ is not $C^2$?
Motivation: if you have a market with two companies where log stocj prices are modelled by the semimartingales X and Y we will be interested ni
$\max(X_t, Y_t)$ and $\min(X_t, Y_t)$
Note: $\max, \min$ are not $C^2$ but at least convex.

\end{document}

  
