\documentclass{article}
\usepackage[utf8]{inputenc}
\usepackage[english]{babel}
\usepackage{graphicx}
\usepackage{amsfonts}
\usepackage{amsmath}
\usepackage{amsthm}
\usepackage[a4paper, total={7in, 10in}]{geometry}
\newtheorem{theorem}{Theorem}
\newtheorem{definition}{Definition}

\begin{document}
Main motivation:
$max(X_t, Y_t)$. We want to know if it is semimartingale.

Claim:
This Q is related to the following Q:
Take a brownian motion $B$. We know that $\int_0^t 1_{B(s) = 0}ds = 0$ for all $t$ with probability 1.
What is the right way of measuring the time spent at 0 by a BM.

Relation between the two questions:
Suppose $B_1, B_2$ are independent standard BMs and we want to stdy $max(B_1, B_2)$


Note: $max(x, y) = \frac{x+y + |x-y|}{2}$, so we only need to analyse $|B_1 - B_2|$. $f(x) = |x|$, $f''(x) = 2\delta_0(x)$
Try to apply Ito: $|B(t)| = \int_0^t sgn(B_s) dB_s + \int_0^t \delta_0(B_s) ds$


Three natural ways of measuring time spent at 0 by BM:
\begin{itemize}
\item Take $\varepsilon > 0$, consider $\int_0^t 1_{|B_s| < \varepsilon} ds$, take the limit $\varepsilon \downarrow 0$ in some way.
\item Define time spent at 0 as: $2(|B(t)| - \int_0^t sign(B_s) dB_s)$
\item Recall that a BM is a limit of random walks, $B(t) = \lim_{\text{distribution}} \frac{S_{\lfloor Kt \rfloor}}{\sqrt K}$
  Take $\lim_K \frac1{\sqrt K}\# \{ i, S_i = 0, i \le \lfloor Kt \rfloor \}$
\end{itemize}

Luckily $1 \iff 2 \iff 3$

\begin{definition}[Local time]
  For a BM $B$ and a point $a \in R$, call the almost sure limit $\lim_{\epsilon 0} \frac{1}{2\epsilon} \int_0^t 1_{|B_s - a| < \epsilon}ds$ the local time of $B$ at $a$ and write $L_t^a$.
  With probability 1 $(a, t) \rightarrow L_t^a$ is continuous.(Trotter '57)
\end{definition}

Remarks:
\begin{enumerate}
\item Need to justify the a.s limit and continuity.
\item $t \rightarrow L_t^a$ is non decreasing wp 1.
\item $B(R) \ni A \rightarrow \int_0^t 1_{B_s \in A}ds$  has $a \rightarrow L_t^a$ as its density.
\end{enumerate}

\begin{definition}[Local Time]
  Define the local time $L_t^a$ by $|B_t-a| - \int_0^t sgn(B_s-a)ds =: \frac12 L_t^a$
\end{definition}
Proof:

Bump function $\rho(x) = c e^{\frac1{x^2-1}}1_{|x| < 1}$, $\int \rho = 1$, $\rho_n(x) = n\rho(nx)$
$h_n(x) = \int_{-\infty}^x \rho_n(y - a)dy, h_n(x) = \int_{-\infty}^x \rho_n(y)dy$
Easy to check: $h_n(x) \rightarrow sgn(x-a), H_n(x) \rightarrow |x - a|$
Clear: $\rho_n \in C^{\infty}$
$\Rightarrow$ Ito:  $H_n(B_t) = H_n(B_0) + \int_0^t h_n(B_s)dB_s + \int_0^t \frac12 \rho_n(B_s)ds$

$n \rightarrow \infty$:
\begin{itemize}
\item $H_n(B_t) \rightarrow |B_t - a|$, as
\item 
\end{itemize}

\end{document}

  


















